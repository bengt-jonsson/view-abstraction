\section{Conclusions}
\label{sec:conclusions}
% \vspace*{-6pt}
We have presented a novel shape abstraction, called fragment abstraction,
for automatic verification of
concurrent data structure implementations that operate different forms of
dynamically allocated heap structures, including singly-linked lists, skiplists,
and arrays of singly-linke lists.
Our approach is the first framework that
can automatically verify concurrent data structure implementations that employ
skiplists and arrays of singly linked lists,
at the same time as handling an unbounded
number of concurrent threads, an unbounded domain of data values
(including timestamps), and an unbounded shared heap.
We showed fragment abstraction allows to combine local and global reachability
information to allow verification of the functional behavior
of a collection of threads.

As future work, we intend to investigate whether fragment abstraction can be
applied also to other heap structures, such as concurrent binary search trees.
%% \begin{itemize}
%% \item
%%   We present a novel formalism for specifying the occurrence of linearization
%%   points, called {\it controllers}.
%% Controllers  can handle, in a simple and uniform manner, complex patterns for
%% We have presented a framework for proving linearizability
%% of algorithms for concurrent data structures such as 
%% % 
%% To that end, we provide a specification language that
%% allows defining complex linearization patterns in a
%% simple and concise manner.
%% %
%% The systems we consider are infinite in multiple dimensions, namely,
%% in the size of the data, the number of threads, and the size
%% of the heap.
%% %
%% Therefore, we define a collection of powerful
%% abstraction schemes that allow both precision
%% and efficiency of the analysis.
%

%%  \paragraph{Acknowledgments}
%% We thank the reviewers for helpful comments.
%% This work was supported in part by
%% the Swedish Foundation for Strategic Research within the
%% ProFuN project, and by the Swedish Research Council within the UPMARC 
%% centre of
%% excellence.
