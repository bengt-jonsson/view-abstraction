\section{Textbank}

\section{Old Overview Text defining the Fragment Abstraction}

\begin{itemize}
  \item
    Our {\em data abstraction} abstracts from actual values variables and
    fields of heap cells that represent data (they range over $\tt int$
    in Figure~\ref{ts-stack:fig}),
%%     of $\intgrs$-valued variables and $\intgrs$-fields of heap cells,
    into possible relations
    (in the set $\set{<,=,>}$) with global data variables and values of
    observer registers. 
    In order to verify that the observer cannot reach an accepting state,
    it is necessary to
    keep track of heap cells whose ${\tt data}$ field has the same value as an
    observer register. Thus, 
    for an observer register $\reg_i$, let $\cell$ be a {\em $\reg_i$-cell} if
    the ${\tt data}$ field of $\cell$ has the same value as $\reg_i$.
    Then, for timestamps, our data abstraction represents
    possible relations with timestamps of $\reg_i$- cells.
%% between values of ${\tt Data}$ fields of 
%% heap cells, local ${\tt Data}$ variables, and registers of the observer.
%% We similarly abstract from concrete values of timestamps by representing
%% only relations between values of timestamps in heap cells and in
%% local variables.
  \item
We cope with an unbounded number of threads by
adapting the {\em thread-modular approach}~\cite{BLMRS:cav08},
representing only the part of the global configuration that is visible to
a single thread $\thread$; this part includes the local state of $\thread$ and
the part of the heap that $\thread$ can access through its local
and global pointer variables.
\item
  We cope with an unbounded heap by a novel shape abstraction, 
called {\em fragment abstraction}. This abstraction 
represents a heap by a set of (heap) {\em fragments}, where each fragment is
a pair of node types (called {\em tags}) that are connected by a pointer.
A tag is an abstraction of a heap cell, which contains both 
\begin{inparaenum}[(i)]
\item
  local information about its non-pointer fields, obtained
   using our data abstraction,
 \item
   global information, saying how
it can be reached from and reach to (by following a chain of pointers)
other heap cells that are pointed to by global variables, or cells whose
    data field has  the same value as some observer register.
\end{inparaenum}
A set of fragments represents the set of heap structures in which each
pair of pointer-connected cells are represented by some fragment in the set.
By construction, our fragment abstraction is finitary, since there is
a bounded set of tags.
\end{itemize}
%% just a pair of cells that are connected by a ${\tt next}$ pointer. Thus, the
%% representation of a particular heap is obtained as the set of pairs of cells
%% that are connected by a ${\tt next pointer}$. Conversely,
%% a set of (heap) fragments represents the set of heaps that can be formed by
%% putting together fragments from this set, where each fragment may be used
%% multiple times. In order to obtain a representation which is both expressive
%% and bounded, cells are first abstracted to a finite set of cell types, called
%% {\em tags}. A tag represents both local information about the values of
%% fields of the cell
%% and about which pointer variables point to the cell, as well as global
%% information about how the cell is positioned in the overall projected heap.
%% The global information records from which global and thread-local pointer
%% variables the cell can be reached by a sequence of pointers, as well as
%% which global variables can be reached from the cell.
%% The postcondition computation must consider how this projection can
%% be changed by execution steps of the program.
%% Such a step can be performed
%% either by the thread $t$ (a local step) or by 
%% some other thread $t'$ (an interference step).
%% Local steps can be handled by adapting standard techniques
%% for computing postconditions.
%% The corresponding postcondition computation must then take into account that
%% the projection onto a particular thread can be changed either by step of that
%% thread, or by a step of another thread.
%% interference steps involve to first compute a new abstract configuration
%% over two thread identifiers
%% which represents the combined state of both threads $t$ and $t'$,
%% and only therafter
%% performing the step of $t'$; thereafter the resulting abstract post-state
%% is projected back onto $t$.

Let us illustrate our fragment abstraction on the TS stack algorithm.
Its heap consists of a set of singly linked lists (SLLs), each of which
is accessed from a pointer in the array ${\tt pools[maxThreads]}$.
Figure~\ref{fig:tsshape} shows an example heap state,
in a configuration when % Quy write from here
it is accessed concurrently by three threads $\thread_1$, $\thread_2$, and $\thread_3$. The heap consists of three SLLs accessed from the three pointers $\tt pools[1]$, $\tt pools[2]$, and $\tt pools[3]$ respectively. Each heap cell is
shown with the values of its fields, using the layout shown to the right in
Figure~\ref{fig:tsshape}.
In addition, each cell is labeled by the
pointer variables that point to it. We use ${\tt lvar[i]}$ to denote the local
variable ${\tt lvar}$ of thread $\thread_i$.
%\todo[inline]{Quy: Use a better scheme for indexing local variables. Previously,
%  we used ${\tt var[2]}$, for variable ${\tt var}$ of thread 2. Maybe use it also here?}
%\todo[inline]{You must start by explaining the notation better. E.g., what means
%  a pointer variable that labels a node?}

Thread $\thread_1$ is trying to push a new node with data value $4$, pointed by its local variable $\tt new$, having reached line 3.
Thread $\thread_3$ has just called the ${\tt push}$ method.
Thread $\thread_2$ has reached line 12 in the execution of the ${\tt pop}$ method,  and has just assigned ${\tt youngest}$ to the first node in the list
pointed to by $\tt pools[3]$ which is not logically removed (in this case it is the last node of that list).
The local state of thread $\thread_2$
%% which executes the ${\tt pop}$ method
consists of the values of its program counter, the boolean variable ${\tt success}$, the pointer variables ${\tt youngest}$, ${\tt myTop}$, and ${\tt n}$,
and the timestamp variable ${\tt maxTS}$.
In verification, we must also consider the observer. In this case, the added
observer is the one in Figure~\ref{fig:lifostack:fig}, which has two registers
$\tt x_1$ and $\tt x_2$, which are assigned the values $4$ and $2$,
respectively.
%\todo[inline]{to Quy: what about the variable ${\tt k}$?}




%\quycom{I am trying to improve text from here}
%\todo[inline]{I do not quite understand that this is the result: Thread 2 is doing a pop from list 3: so it sees ${\tt top}_3$. But why does it see the
%  list 2? and why does it need to see the list 1? Does thread 2 really see all the nodes and variables in Figure 5?}
%As can be seen the the figure, the node pointed by $\tt new[1]$ is not visible with $\thread_3$ because it is still the private variable of $\thread_2$. We use three observer registers $\tt d_1$, $\tt d_2$, and $\tt d_3$. The data value $\tt 9$ is equal to $\tt d_1$, $\tt 8$ is equal to $\tt d_2$ and all other data values are equal to $\tt d_3$. To abstract timestamp of a node, we keep the ordering between its timestamp and nodes whose data values are equal to $\tt d_1$ or $\tt d_2$ which are described by $\tt to_1$,...,$\tt to_6$ in the figure.
% Quy write here
%\todo[inline]{Quy: make a picture of the correspondingq
%  thread view with data abstraction  Complete this description}.
%\begin{figure}
%	\input tsshapeb
%\caption{The projection on thread 2 after data abstraction}
%\label{fig:tsdabsshape}
%\end{figure} 

%In order to represent the projection in Fig. ~\ref{fig:tsdabsshape} by

Our symbolic representation represents the reachable local states of a thread by
a finite set of {\em local symbolic configurations}, using our
data abstraction.
It maps each local symbolic configuration to a set of fragments, which
is a symbolic representation of the set of corresponding reachable heaps.
Thus, our symbolic representation is a mapping from a set of local symbolic
configurations, which maps each local symbolic configuration in its domain
to a set of fragments.
A global configuration satisfies a symbolic representation $\symbrep$
if the local stat of each thread $\thread$ satisfies some local symbolic
configuration in the domain of $\symbrep$, which is mapped to a set
of fragments, which represents the heap that is accessible to $\thread$ in
the global configuration. In the following, we explain more precisely the
local symbolic configurations, and our fragment abstraction.

A local configuration of a thread which executes the ${\tt pop}$ method consists
of the values of the program counter, the boolean variable ${\tt success}$, the
pointer variables ${\tt youngest}$, ${\tt myTop}$, ${\tt n}$,
and the timestamp variable ${\tt maxTS}$.
A {\em local symbolic configuration} is defined by applying our data abstraction
to the local non-pointer variables.  For a ${\tt pop}$ method, it maps
the program counter and ${\tt success}$ to their current values. It also maps
${\tt maxTS}$ to a mapping from each observer register $\reg_i$
to $\set{<,=,>}$, such that if $t$ is the value of some $\reg_i$-node's
$\tt ts$-field, then ${\tt maxTS} \sim t$ for some $\sim$ in this set.

In the configuration of Figure~\ref{fig:tsshape}, thread $\thread_2$
satisfies the local symbolic configuration shown in the green towards the
bottom, 
which maps the program counter to line 12, ${\tt success}$ to ${\tt false}$ and ${\tt maxTS}$ to the mapping from observer registers which maps
$\tt x_1$ to $\set{>}$ and $\tt x_2$ to $\set{=}$
(recall that the values of $\tt x_1$ and $\tt x_2$ are $4$, $2$ respectively).

\todo[inline]{To Quy: what is the value of ${\tt MaxTS}$?}

%% Secondly, we identify the current working list and abstract the other lists where we do not distinguish the differences between variables in the other list.

Let us now show how our fragment abstraction represents the heap that is
accessible to $\thread_2$.
%% We must first define the notion of tag.
%% Finally, we define the information conveyed by node tags. We define one tag for push operations and one for pop operations, since
%% their methods use different sets of local variables.
The local pointer variables of $\thread_2$ are 
${\tt youngest}$, ${\tt myTop}$, and ${\tt n}$. Its global variables are of form
${\tt pools[i]}$, where ${\tt i}$ is a thread index, which we abstract into the
two variables ${\tt pools[me]}$, representing ${\tt pools[3]}$
which points to the list which $\thread_2$ is currently accessing,
and ${\tt pools[other]}$ representing ${\tt pools[i]}$ for $\tt i \neq 3$.
%\todo[inline]{Quy: here, we need a good way to explain how to represent ${\tt pools[i]}$
%  and ${\tt pools[k]}$}
%\todo[inline]{Thus, which are the global variables?}
%% For a thread which executes a $\tt pop$ operation, 
Define a {\em tag} as a tuple
$\tuple{\pvars,\vals,\reachfrom,\reachto,\private}$, where
\begin{itemize}
\item
  $\pvars$ is a set of local and global pointer variables
%%   of the ${\tt pop}$ method and (i.e., a subset of ${\tt top, ctop, youngest, n, myTop}$
  %\bjcom{Quy: Check exactly what this set should be}
\item
  $\vals$ is an abstraction of the non-pointer fields of a cell,
  i.e., it maps
  \begin{inparaenum}[(i)]
  \item
    the ${\tt mark}$ field to a boolean value,
  \item
    the ${\tt data}$ field to a mapping from observer registers to $\set{=, \neq}$,
  \item
    the ${\tt ts}$ field to a mapping from observer registers to subsets of $\set{<,=,>}$.
  \end{inparaenum}
\item
  $\reachfrom$ and $\reachto$ are sets of global variables and observer registers, and %\bjcom{Quy: Check what is really needed here: Maybe we need only reachfrom?}
\item
  $\private$ is a boolean value.
\end{itemize}
%% For an observer register $\reg_i$, let $\cell$ be a {\em $\reg_i$-cell} if
%% the ${\tt data}$ field of $\cell$ has the same value as $\reg_i$.
A heap cell $\cell$ which is accessible to $\thread_2$ satisfies a tag
with respect to $\thread_2$,
denoted $\cell \satfrag{{\thread_2}} \tagtuple$ if
\begin{itemize}
\item
  $\pvars$ is the set of global and local (to $\thread_2$)
  pointer variables that point to $\cell$,
  %\bjcom{Quy: Or Should it be contains?}
\item
the non-pointer fields of $\cell$ satisfies $\vals$ under our data abstraction, i.e.
  \begin{inparaenum}[(i)]
  \item
    $\vals({\tt mark})$ is the value of its ${\tt mark}$ field,
  \item
    for each observer register $\reg_i$, if $\cell$ is (is not) a
    $\reg_i$-cell, then $= \in \vals({\tt data})$ ($\neq \in \vals({\tt data})$),
  \item
    let $ts$ be the value of its ${\tt ts}$ field, then
    for each observer register $\reg_i$, if
    the value $t$ of the ${\tt ts}$ field of some $\reg_i$-cell
    satisfies $ts \sim t$, then $\sim \in \vals({\tt ts})$,
  \end{inparaenum}
\item $\reachfrom$ is the set of global variables from which $\cell$ is 
  reachable (by a chain of ${\tt next}$) pointers), and the set of
  observer registers $\reg_i$ such that $\cell$ is reachable from some $\reg_i$-cell,
\item $\reachto$ is the set of global variables that point to a cell that
  is reachable from $\cell$ and  the set of
  observer registers $\reg_i$ such that some $\reg_i$-cell is reachable from $\cell$,
\item $\private$ is $\true$ only if $\cell$ cannot be accessed from any
  other thread than $\thread_2$.
\end{itemize}

We can now define the concept of fragment.
A {\em fragment} $\frag$ is a pair of form $\tuple{\inputtag,\outputtag}$, 
of form $\tuple{\inputtag,\nullconst}$, or 
of form $\tuple{\inputtag,\dangconst}$.
We write  
$\cell \satfrag{{\thread_2}} \tuple{\inputtag,\outputtag}$ if
the ${\tt next}$ field of $\cell$ points to a cell $\cell'$ such that
$\cell \satfrag{{\thread_2}} \inputtag$ and
$\cell' \satfrag{{\thread_2}} \outputtag$.
For fragments of form $\tuple{\inputtag,\nullconst}$ 
or $\tuple{\inputtag,\dangconst}$, the definition is adapted in the obvious way.
A global configuration satisfies a set of fragments wrp.\ to $\thread_2$ if
for any cell $\cell$ that is accessible to $\thread_2$, it contains a fragment
$\frag$ such that $\cell \satfrag{{\thread_2}} \frag$.

%\todo[inline]{Quy: can you revise from here, and explain carefully the
%  local symbolic configuration, and set of fragments in 
%  Figure~\ref{fig:tsviewshape}
%}

%\todo[inline]{to Quy: Comments on Figure~\ref{fig:tsviewshape}:
%  I suggest to let the data field be a SET, use []- notation for the mappings of
%  the $\lambda_i$, you had some nice notation before for reachfrom and
%  reachto}
%\todo[inline]{Quy: make a picture with all the fragments obtained from the
 % thread projection}
Figure~\ref{fig:tsviewshape} shows a set of fragments that is
satisfied wrp.\ to $\thread_2$ by configuration in Figure~\ref{fig:tsshape}.
Each tag in a fragment is depicted with the same layout as heap cells.
The non-pointer fields are replaced by their corresponding data abstractions. 
For the $\tt data$ field, a register $\reg_i$ denotes that $\reg_i$ is mapped to
$\set{=}$, and a $\circ$ that both registers are mapped to $\set{\neq}$.
The ${\tt ts}$ field has can be mapped to four different mappings: $\lambda_1$, $\lambda_2$, $\lambda_3$ and $\lambda_4$, which are shown in the
yellow box at the bottom.
%% which are the mappings from observers registers to subsets of $\set{<,=,>}$ as described in the figure.
The set $\pvars$ is shown above the tag.
The sets $\reachfrom$ and $\reachto$ are shown below each tag, with
$\reachfrom$ is in the first row and $\reachto$ in the second row.

The fragments $\tt v_1$, $\tt v_2$, and $\tt v_3$ represent cells
in the first two lists, whereas, $\tt v_4$, $\tt v_5$, $\tt v_6$, and $\tt v_7$
represent cells in the list pointed to by $\tt pools[3]$
%% The data field of a cell is presented by $\tt x_1$ if its value is equal to $\tt x_1$, $\tt x_2$ if its value is equal to $\tt x_2$ and $\circ$ if its value is different from both $\tt x_1$ and $\tt x_2$. 


%% The local symbolic configuration maps the program counter to the value 12, ${\tt success}$ to ${\tt false}$ and ${\tt maxTS}$ to a mapping from observer registers $\tt x_1$ to $\set{>}$, $\tt x_2$ to $\set{=}$. 

%% A thread-abstracted configuration satisfies a symbolic configuration if
%% \begin{itemize}
%% \item it satisfies program counter, observer state, and local variables %\bjcom{Say that program counter, and local variables ...}
%% \item
%%   For each cell $\cell$ in the heap of the configuration, there is
%%   a fragment $\tuple{\inputtag,\outputtag}$ in the set of fragments, such
%%   that $\cell \lhd \inputtag$ and ${\tt next}(\cell) \lhd \outputtag$.
%%   \bjcom{check the case when $\cell$ points to null etc.}
%% \end{itemize}

In the  verification, we must compute a symbolic representation
that is satisfied by all reachable program configurations (recall that
program configurations include the state of the observer).
This invariant is obtained by an abstract-interpretation-based
fixpoint procedure, which starts
from a representation of the set of initial configurations, and
thereafter repeatedly performs
postcondition computations that extend the
symbolic representation by the effect of any execution step of the program,
until convergence.
This procedure is presented in Section~\ref{subsect:postcond}.

\subsection{Old text from beginning of Section~\ref{sec:fragment-abstraction}}

By Theorem~\ref{thm:soundness}, linearizability can be verified by
establishing a 
reachability property for the product $\system=\augprog\compose\monitor$.
This verification must handle the challenges of
an unbounded number of threads, an unbounded heap, and unbounded data domain.
We have therefore developed a symbolic representation, which
is a careful combination of
data abstraction, thread abstraction, and shape abstraction. 

Our {\em data abstraction} abstracts from concrete data values in
$\intgrs$ by representing only {\em relations}
(in the set $\set{\prec,\equiv,\succ}$) between values of {\tt Data} fields of 
heap cells, local {\tt Data} variables, and registers of the observer.
%% We similarly abstract from concrete values of timestamps by representing
%% only relations between values of timestamps in heap cells and in
%% local variables.
In order to cope with an unbounded number of threads,
we extend the {\em thread-modular approach}~\cite{BLMRS:cav08}, in which
the symbolic representation characterizes
the projection of the configuration onto a single thread $\thread$.
%% The postcondition computation must consider how this projection can
%% be changed by execution steps of the program.
%% Such a step can be performed
%% either by the thread $\thread$ (a local step) or by 
%% some other thread $\thread'$ (an interference step).
%% Local steps can be handled by adapting standard techniques
%% for computing postconditions.
The corresponding postcondition computation must then take into account that
the projection onto a particular thread can be changed either by step of that
thread, or by a step of another thread.
%% interference steps involve to first compute a new abstract configuration
%% over two thread identifiers
%% which represents the combined state of both threads $\thread$ and $\thread'$,
%% and only therafter
%% performing the step of $\thread'$; thereafter the resulting abstract post-state
%% is projected back onto $\thread$.
In order to cope with an unbounded heap, we have developed a novel symbolic
representation, called {\em fragment abstraction}.
The idea of this abstraction is
to represent a heap by a set of (heap) {\em fragments}, where each fragment is
just a pair of nodes that are connected by a pointer.
In this section, we describe how this is done
for singly linked lists, i.e., where each heap cell contains exactly one
pointer field, named {\tt next}. In the following sections XXX and YYY, we describe
how fragment abstraction is applied to skiplists and to arrays of singly linked
lists.

\todo[inline]{Maybe we include timestamps from the very start?} 


\subsection{Old text on Intersection}

\paragraph{Fragment Intersection:} 
Let us now describe how to form the intersection of two sets of fragments
$\fragset_1$ and $\fragset_2$, where $\fragset_i$ characterizes the projection of the heap onto thread $\thread_i$ for $i=1,2$. In order to distinguish
between local variables of $\thread_1$ and $\thread_2$, we assume that local variable $\lvarof{x}$ in $\thread_i$ is named as $\lvarof{x[i]}$.
Let $\fragset$ denote the result of intersecting $\fragset_1$ and $\fragset_2$.
Then, for each heap cell $\cell$, which is
accessible to either $\thread_1$ or $\thread_2$, the set $\fragset$ must
contain a fragment $\frag$ with $\cell \satfrag{1,2} \frag$.
\bjcom{This notation to be defined}
There are  here two possibilities.
\begin{itemize}
\item If $\cell$ is accessible to both $\thread_1$ and $\thread_2$, then
  there are fragments $\frag_1 \in \fragset_1$ and $\frag_2 \in \fragset_2$
  such that $\cell \satfrag{1} \frag_1$ and $\cell \satfrag{2} \frag_2$.
  We use the notation $\frag_1 \sqcap \frag_2$ to denote a set of views such that
  whenever $\cell \satfrag{1} \frag_1$ and $\cell \satfrag{2} \frag_2$ then
  $\cell \satfrag{1,2} \frag$ for some $\frag \in (\frag_1 \sqcap \frag_2)$.
\item If $\cell$ is accessible to only one of $\thread_1$ and $\thread_2$,
  say $\thread_1$, then $\fragset$ should contain some fragment
  $\frag_1 \in \fragset_1$ with $\cell \satfrag{1,2} \frag_1$
  \bjcom{Check that we need not change $\frag_1$}
\end{itemize}
For a fragment $\frag$, define
$\greachfromat{\inat{\frag}}$ as the set of global variables in
$\reachfromat{\inat{\frag}}$. Define
$\greachtoat{\inat{\frag}}$,
$\greachfromat{\outat{\frag}}$,
$\greachtoat{\outat{\frag}}$,
$\gpvarsat{\inat{\frag}}$, and
$\gpvarsat{\outat{\frag}}$
analogously.
Define $\gtagat{\outat{\frag}}$ as the tuple
$\tuple{\gpvarsat{\outat{\frag}},\valsat{\outat{\frag}},\greachfromat{\outat{\frag}},\greachtoat{\outat{\frag}},\privateat{\outat{\frag}}}$.

\todo[inline]{Question to Quy: You have also used the notation
  $\gvalsat{\inat{\frag}}$. Will you need it, and if so what does it mean?} \quycom{Because in the data, I add data constraint between data fields and local data variable. When we do intersection, we do not need to care about this constraint because its local constraint}

Let us now describe how to compute  $\frag_1 \sqcap \frag_2$ for
two views $\frag_1 \in \fragset_1$ and $\frag_2 \in \fragset_2$.
Firstly, we consider the case where both $\frag_1$ and $\frag_2$ have size 2.
Let us consider some different cases.
They all take into account the  observation that if a cell $\cell$ satisfies
$\cell \satfrag{1} \frag_1$ and $\cell \satfrag{2} \frag_2$, then
the information about global variables in $\frag_1$ and $\frag_2$ must coincide.
 \begin{itemize}
 \item if $\greachfromat{\inat{\frag_1}} \neq \emptyset$ and $\greachfromat{\inat{\frag_2}} \neq \emptyset$ then the global information in $\frag_1$ and $\frag_2$ must coincide. We hence obtain:
  \begin{itemize}
  \item if $\gtagat{\inat{\frag_1}} = \gtagat{\inat{\frag_2}}$ and $\gtagat{\outat{\frag_1}} = \gtagat{\outat{\frag_2}}$ then $\frag_1 \sqcap v_2 = \set{\frag_{12}}$ where $\frag_{12}$ is identical to $\frag_1$ except that 
  \begin{itemize}
  \item $\pvarsat{\inat{\frag_{12}}} = \pvarsat{\inat{\frag_1}} \cup \pvarsat{\inat{\frag_2}}$ 
  \item  $\pvarsat{\outat{\frag_{12}}} = \pvarsat{\outat{\frag_1}} \cup \pvarsat{\outat{\frag_2}}$
  \item $\reachfromat{\inat{\frag_{12}}} = \reachfromat{\inat{\frag_{1}}} \cup \reachfromat{\inat{\frag_{2}}}$
   \item $\reachfromat{\outat{\frag_{12}}} = \reachfromat{\outat{\frag_{1}}} \cup \reachfromat{\outat{\frag_{2}}}$
     \todo[inline]{Question to Quy: Why do this union only for $\pvars$ and $\reachfrom$, and not for $\reachto$ and $\vals$?}\quycom{we fixed this in the disscussion}
    \end{itemize}
  \todo[inline]{Question to Quy: Should we not have an ``else'' here, with
    $\frag_1 \sqcap \frag_2 = \emptyset$?}\quycom{we fixed this in the disscussion}
 \end{itemize}

 \item if $\greachfromat{\inat{\frag_1}} = \emptyset$, $\greachfromat{\inat{\frag_2}} = \emptyset$, $\greachfromat{\outat{\frag_1}} \neq \emptyset$ and $\greachfromat{\outat{\frag_2}} \neq \emptyset$ then 
  \begin{itemize}
  \item if $\gtagat{\outat{\frag_1}} = \gtagat{\outat{\frag_2}}$, $\privateat{\inat{\frag_1}} = \false$ and $\privateat{\inat{\frag_2}} = \false$ then $\frag_1 \sqcap \frag_2 = \set{\frag_1', \frag_2', \frag_{12}}$ where
      %$\frag_{12}$ is same as $\frag_1$ except that 
%    \begin{itemize}
%  \item $\pvarsat{\inat{\frag_{12}}} = \pvarsat{\inat{\frag_1}} \cup \pvarsat{\inat{\frag_2}}$ 
%  \item  $\pvarsat{\outat{\frag_{12}}} = \pvarsat{\outat{\frag_1}} \cup \pvarsat{\outat{\frag_2}}$
%  \item $\reachfromat{\inat{\frag_{12}}} = \reachfromat{\inat{\frag_{1}}} \cup \reachfromat{\inat{\frag_{2}}}$
%   \item $\reachfromat{\outat{\frag_{12}}} = \reachfromat{\outat{\frag_{1}}} \cup \reachfromat{\outat{\frag_{2}}}$
%    \end{itemize} and
      \begin{itemize}
      \item
        $\frag_1'$ is same as $\frag_1$ except that 
    \begin{itemize}
  \item  $\pvarsat{\outat{\frag_1'}} = \pvarsat{\outat{\frag_1}} \cup \pvarsat{\outat{\frag_2}}$
  \item $\reachfromat{\outat{\frag_1'}} = \reachfromat{\outat{\frag_{1}}} \cup \reachfromat{\outat{\frag_{2}}}$
    \end{itemize}
  \item
    $\frag_2'$ is same as $\frag_2$ except that 
    \begin{itemize}
  \item  $\pvarsat{\outat{\frag_2'}} = \pvarsat{\outat{\frag_1}} \cup \pvarsat{\outat{\frag_2}}$
  \item $\reachfromat{\outat{\frag_2'}} = \reachfromat{\outat{\frag_{1}}} \cup \reachfromat{\outat{\frag_{2}}}$
    \end{itemize}
  \item
    $\frag_{12}$ is as in the previous case.
    \bjcom{to Quy: I added this, is it correct?}
      \end{itemize}
    \item if $\gtagat{\outat{\frag_1}} = \gtagat{\outat{\frag_2}}$ and $\privateat{\inat{\frag_1}} = \true$ or $\privateat{\inat{\frag_2}} = \true$ then $\frag_1 \sqcap \frag_2 = \set{\frag_1', \frag_2'}$ where
$\frag_1'$ and $\frag_2'$ are as above.
 \end{itemize}
 
 \item if $\greachfromat{\inat{\frag_1}} = \emptyset$, $\greachfromat{\inat{\frag_2}} = \emptyset$, $\greachfromat{\outat{\frag_1}} = \emptyset$ and $\greachfromat{\outat{\frag_2}} = \emptyset$ then 
 \begin{itemize}
 \item if $\tt gtag(\frag_1, o) = gtag(\frag_2, o)$, $\privateat{\inat{\frag_1}} = false$, $\privateat{\outat{\frag_1}} = false$, $\privateat{\outat{\frag_1}} = false$ and $\privateat{\outat{\frag_2}} = false$ then  $\frag_1 \sqcap \frag_2 = \{\frag_1,\frag_2,\frag_1',\frag_2',\frag_{12}\}$ %where $\frag_{12}$ is same as $\frag_1$ except that 
   % \begin{itemize}
 % \item $\pvarsat{\inat{\frag_{12}}} = \pvarsat{\inat{\frag_1}} \cup \pvarsat{\inat{\frag_2}}$ 
 % \item  $\pvarsat{\outat{\frag_{12}}} = \pvarsat{\outat{\frag_1}} \cup \pvarsat{\outat{\frag_2}}$
 % \item $\reachfromat{\inat{\frag_{12}}} = \reachfromat{\inat{\frag_{1}}} \cup \reachfromat{\inat{\frag_{2}}}$
 %  \item $\reachfromat{\outat{\frag_{12}}} = \reachfromat{\outat{\frag_{1}}} \cup \reachfromat{\outat{\frag_{2}}}$
 %   \end{itemize} and 
 %   $\frag_{1'}$ is same as $\frag_1$ except that 
 %   \begin{itemize}
 % \item  $\pvarsat{\outat{\frag_1'}} = \pvarsat{\outat{\frag_1}} \cup \frag_2.vars(j_o)$
 % \item $\reachfromat{\outat{\frag_1'}} = \reachfromat{\outat{\frag_{1}}} \cup \reachfromat{\outat{\frag_{2}}}$
 %   \end{itemize}
 %  and $\frag_2'$ is same as $\frag_2$ except that 
 %   \begin{itemize}
 % \item  $\pvarsat{\outat{\frag_2'}} = \pvarsat{\outat{\frag_1}} \cup \frag_2.vars(j_o)$
 % \item $\reachfromat{\outat{\frag_2'}} = \reachfromat{\outat{\frag_{1}}} \cup \reachfromat{\outat{\frag_{2}}}$
 %   \end{itemize}
    
    
   \item if $\tt gtag(\frag_1, o) = gtag(\frag_2, o)$, ($\privateat{\inat{\frag_1}} = true$ or $\privateat{\inat{\frag_2}} = true$) and $\privateat{\outat{\frag_1}} = false$ and $\privateat{\outat{\frag_2}} = false$ then  $\frag_1 \sqcap \frag_2 = \{\frag_1,\frag_2,\frag_1',\frag_2'\}$  

  \item if $\tt gtag(\frag_1, o) = gtag(\frag_2, o)$ and ($\privateat{\outat{\frag_1}} = true$ or $\privateat{\outat{\frag_1}} = true$) then  $\frag_1 \sqcap \frag_2 = \{\frag_1,\frag_2\}$ 

    \item if $\tt gtag(\frag_1, o) \neq gtag(\frag_2, o)$ then $\frag_1 \sqcap \frag_2 = \{\frag_1,\frag_2\}$
 \end{itemize}
 
 \end{itemize}



\subsection{Old text on Skiplists}

%\paragraph{Index Consistency:} Let $\tt i_1 \equiv i_2$ denotes that $\tt i_1$ and $\tt i_2$ are equivalent. Intuitively, $\tt i_1 \equiv i_2$ iff  ${\tt Val}^{\tt v}(\tt i_1) = {\tt Val}^{\tt v}(\tt i_2)$, ${\tt Val}^{\tt d}(\tt i_1) = {\tt Val}^{\tt d}(\tt i_2)$, $\tt private(i_1) = private(i_2)$ and $\tt \Phi(i_1.d, z) = \Phi(i_2.d, z)$ for every $\intgrs$-field $\tt d$ and $\intgrs$-term $\tt z$.  The indices $\tt i_1$ and $\tt i_2$ are consistent if they satisfy one of those conditions below
%\begin{itemize}
%	\item $\tt i_1$ is an {\em input} index, $\tt i_2$ is an {\em output} index, $\tt i_1 \equiv i_2$ and ${\tt Val}^{\tt r}(\tt i_1) = {\tt Val}^{\tt r}(\tt i_2)$, or
%	\item $\tt i_1$ is an {\em output} index, $\tt i_2$ is an {\em input} index, $\tt i_1 \equiv i_2$ and ${\tt Val}^{\tt r}(\tt i_1) = {\tt Val}^{\tt r}(\tt i_2)$, or
%	\item both $\tt i_1$ and $\tt i_2$ are output indices, $\tt i_1 \equiv i_2$, $\tt X_1 \cap X_2 = \emptyset$ and $\tt Y_1 = Y_2$ where $\tt {\tt Val}^{\tt r}(\tt i_1) = X1 \mapsto Y1$,  $\tt {\tt Val}^{\tt r}(\tt i_2) = X2 \mapsto Y2$
%\end{itemize}
\paragraph{Fragment Consistency:} For two fragments $\tt v_1$ and $\tt v_2$ with sizes of 2, let $\tt i$ and $\tt o$ be the {\em input} and {\em output} indices of $\tt v_1$, let $\tt j_i$ and $\tt j_o$ be the {\em input} and {\em output} indices of $\tt v_2$.
 \begin{itemize}
 \item   $\tt v_1$ and $\tt v_2$ are forward consistent, denoted as $\tt v_1 \hookrightarrow_S v_2$ if $\tt tag( v_1,\tt o) = tag(v_2,\tt j_i)$.   
 \item $\tt v_1$ and $\tt v_2$ is branch consistent, denoted as $\tt v_1 \leftrightarrow_S v_2$ if $\tt tag( v_1,\tt o) = tag(v_2,\tt j_o)$, $\tt G_1 \cap G_2 = \emptyset$ and $\tt v_1.{\tt reachto}(\tt i) = v_2.{\tt reachto}(\tt j_i)$ where $\tt G_1$ is the set of global variables in $\tt v_1.{\tt reachfrom}(\tt o)$,  $\tt G_2$ is the set of global variables in $\tt v_2.{\tt reachfrom}(\tt j_o)$. 
 
    \end{itemize}
For two fragments $\tt v_1$ and $\tt v_2$ with sizes of 1, let $\tt i$ be the {\em input} index of $\tt v_1$, let $\tt j_i$ be the {\em input} index of $\tt v_2$.
 \begin{itemize}

 \item $\tt v_1$ and $\tt v_2$ is branch consistent, denoted as $\tt v_1 \leftrightarrow_S v_2$ if $\tt v_1.next(i) = v_2.next(i)$ and $\tt G_1 \cap G_2 = \emptyset$ where $\tt G_1$ is the set of global variables in $\tt v_1.{\tt reachfrom}(\tt i)$,  $\tt G_2$ is the set of global variables in $\tt v_2.{\tt reachfrom}(\tt j_i)$.
 \end{itemize}
For two fragments $\tt v_1$ and $\tt v_2$, let we define several more relations between them as following:
\begin{itemize}
\item  $\tt v_1 \overset{+}\hookrightarrow_S v_2$ if there exists $\tt v$ such that $\tt v_1 \overset{*}\hookrightarrow_S v$ and $\tt v \hookrightarrow_S v_2$  where  $\tt \overset{*}\hookrightarrow_S$ is the transitive closure of $\tt \hookrightarrow_S$. 
\item $\tt v_1  \overset{**}\leftrightarrow_S  v_2$ if there exist $\tt v_1'$, $\tt v_2'$ such that ($\tt v_1  \overset{*}\hookrightarrow_S v_1'$ or $\tt v_1 = v_1'$), and ($\tt v_2  \overset{*}\hookrightarrow_S v_2'$ or $\tt v_2 = v_2'$) as well as $\tt v_1' \leftrightarrow_S v_2'$. 
\item $\tt v_1  \overset{*+}\leftrightarrow_S v_2$ if there exist $\tt v_1'$, $\tt v_2'$ such that ($\tt v_1  \overset{*}\hookrightarrow_S v_1'$ or $\tt v_1 = v_1'$), and $\tt v_2  \overset{*}\hookrightarrow_S v_2'$ as well as $\tt v_1' \leftrightarrow_S v_2'$. 
\item $\tt v_1  \overset{*\circ}\leftrightarrow_S v_2$ if there exist $\tt v_1'$ such that ($\tt v_1  \overset{*}\hookrightarrow_S v_1'$ or  $\tt v_1 = v_1'$) and  $\tt v_2 \leftrightarrow_S v_1'$.
\item $\tt v_1 \overset{++}\leftrightarrow_S v_2$ if there exist $\tt v_1'$, $\tt v_2'$ such that $\tt v_1  \overset{*}\hookrightarrow_S v_1'$, and $\tt v_2  \overset{*}\hookrightarrow_S v_2'$ as well as $\tt v_1' \leftrightarrow_S v_2'$.

\item $\tt v_1 \overset{+*}\leftrightarrow_S v_2$ if there exist $\tt v_1'$, $\tt v_2'$ such that $\tt v_1  \overset{*}\hookrightarrow_S v_1'$, and ($\tt v_2  \overset{*}\hookrightarrow_S v_2'$ or $\tt v_2 = v_2'$) as well as $\tt v_1' \leftrightarrow_S v_2'$.

\item $\tt v_1 \overset{+\circ}\leftrightarrow_S v_2$ if there exist $\tt v_1'$ such that $\tt v_1  \overset{*}\hookrightarrow_S v_1'$, and $\tt v_1' \leftrightarrow_S v_2'$.
\end{itemize}
 
\paragraph{Fragment Intersection:} Let us describe the intersection of two fragments $\tt v_1 \in S_1$ and $\tt v_2 \in S_2$ denoted as $\tt v_1 \sqcap v_2$. Firstly, we consider the case where both $\tt v_1$ and $\tt v_2$ have size 2.
 \begin{itemize}
 \item if $\tt v_1.greachfrom(i) \neq \emptyset$ and $\tt v_2.greachfrom(i) \neq \emptyset$ then 
  \begin{itemize}
  \item if $\tt gtag(v_1, i) = gtag(v_2, i)$ and $\tt gtag(v_1, o) = gtag(v_2, o)$ then $\tt v_1 \sqcap v_2 = \{v_{12}\}$ where $\tt v_{12}$ is same as $\tt v_1$ except that 
  \begin{itemize}
  \item $\tt v_{12}.vars(i) = v_1.vars(i) \cup v_2.vars(i)$ 
  \item  $\tt v_{12}.vars(o) = v_1.vars(o) \cup v_2.vars(o)$
  \item $\tt v_{12}.i.reachfrom = v_{1}.i.reachfrom \cup v_{2}.i.reachfrom$
   \item $\tt v_{12}.o.reachfrom = v_{1}.o.reachfrom \cup v_{2}.o.reachfrom$
    \end{itemize}
 \end{itemize}

 \item if $\tt v_1.greachfrom(i) = \emptyset$, $\tt v_2.greachfrom(i) = \emptyset$, $\tt v_1.greachfrom(o) \neq \emptyset$ and $\tt v_2.greachfrom(o) \neq \emptyset$ then 
  \begin{itemize}
  \item if $\tt gtag(v_1, o) = gtag(v_2, o)$, $\tt v_1.private.(i) = false$ and $\tt v_2.private.(i) = false$ then $\tt v_1 \sqcap v_2 = \{v_1', v_2', v_{12}\}$ where %$\tt v_{12}$ is same as $\tt v_1$ except that 
%    \begin{itemize}
%  \item $\tt v_{12}.vars(i) = v_1.vars(i) \cup v_2.vars(i)$ 
%  \item  $\tt v_{12}.vars(o) = v_1.vars(o) \cup v_2.vars(o)$
%  \item $\tt v_{12}.i.reachfrom = v_{1}.i.reachfrom \cup v_{2}.i.reachfrom$
%   \item $\tt v_{12}.o.reachfrom = v_{1}.o.reachfrom \cup v_{2}.o.reachfrom$
%    \end{itemize} and
    
     $\tt v_1'$ is same as $\tt v_1$ except that 
    \begin{itemize}
  \item  $\tt v_1'.vars(o) = v_1.vars(o) \cup v_2.vars(o)$
  \item $\tt v_1'.o.reachfrom = v_{1}.o.reachfrom \cup v_{2}.o.reachfrom$
    \end{itemize}
   and $\tt v_2'$ is same as $\tt v_2$ except that 
    \begin{itemize}
  \item  $\tt v_2'.vars(o) = v_1.vars(o) \cup v_2.vars(o)$
  \item $\tt v_2'.o.reachfrom = v_{1}.o.reachfrom \cup v_{2}.o.reachfrom$
    \end{itemize}
     \item if $\tt gtag(v_1, o) = gtag(v_2, o)$ and ($\tt v_1.private.(i) = true$ or $\tt v_2.private.(i) = true$) then $\tt v_1 \sqcap v_2 = \{v_1', v_2'\}$ 
 \end{itemize}
 
 
 
 
 
 
 
 
 \item if $\tt v_1.greachfrom(i) = \emptyset$, $\tt v_2.greachfrom(i) = \emptyset$, $\tt v_1.greachfrom(o) = \emptyset$ and $\tt v_2.greachfrom(o) = \emptyset$ then 
 \begin{itemize}
 \item if $\tt gtag(v_1, o) = gtag(v_2, o)$, $\tt v_1.private.(i) = false$, $\tt v_1.private.(o) = false$, $\tt v_1.private.(o) = false$ and $\tt v_2.private.(o) = false$ then  $\tt v_1 \sqcap v_2 = \{v_1,v_2,v_1',v_2',v_{12}\}$ %where $\tt v_{12}$ is same as $\tt v_1$ except that 
   % \begin{itemize}
 % \item $\tt v_{12}.vars(i) = v_1.vars(i) \cup v_2.vars(i)$ 
 % \item  $\tt v_{12}.vars(o) = v_1.vars(o) \cup v_2.vars(o)$
 % \item $\tt v_{12}.i.reachfrom = v_{1}.i.reachfrom \cup v_{2}.i.reachfrom$
 %  \item $\tt v_{12}.o.reachfrom = v_{1}.o.reachfrom \cup v_{2}.o.reachfrom$
 %   \end{itemize} and 
 %   $\tt v_{1'}$ is same as $\tt v_1$ except that 
 %   \begin{itemize}
 % \item  $\tt v_1'.vars(o) = v_1.vars(o) \cup v_2.vars(j_o)$
 % \item $\tt v_1'.o.reachfrom = v_{1}.o.reachfrom \cup v_{2}.o.reachfrom$
 %   \end{itemize}
 %  and $\tt v_2'$ is same as $\tt v_2$ except that 
 %   \begin{itemize}
 % \item  $\tt v_2'.vars(o) = v_1.vars(o) \cup v_2.vars(j_o)$
 % \item $\tt v_2'.o.reachfrom = v_{1}.o.reachfrom \cup v_{2}.o.reachfrom$
 %   \end{itemize}
    
    
   \item if $\tt gtag(v_1, o) = gtag(v_2, o)$, ($\tt v_1.private.(i) = true$ or $\tt v_2.private.(i) = true$) and $\tt v_1.private.(o) = false$ and $\tt v_2.private.(o) = false$ then  $\tt v_1 \sqcap v_2 = \{v_1,v_2,v_1',v_2'\}$  

  \item if $\tt gtag(v_1, o) = gtag(v_2, o)$ and ($\tt v_1.private.(o) = true$ or $\tt v_1.private.(o) = true$) then  $\tt v_1 \sqcap v_2 = \{v_1,v_2\}$ 

    \item if $\tt gtag(v_1, o) \neq gtag(v_2, o)$ then $\tt v_1 \sqcap v_2 = \{v_1,v_2\}$
 \end{itemize}
 
 \end{itemize}


%For the case when $\tt v_1$ and $\tt v_2$ have size 1, let $\tt i$ be the {\em input} index of $\tt v_1$, let $\tt j_i$ be the {\em input} index of $\tt v_2$.
% \begin{itemize}
%
% \item if $\tt reachfrom(i) \neq \emptyset$ and $\tt reachfrom(j_i) \neq \emptyset$ then 
%  \begin{itemize}
%  \item if $\tt gtag(v_1, i) = gtag(v_2, j_i)$ then $\tt v_1 \cap v_2 = v_{12}$ where $\tt v_{12}$ is same as $\tt v_1$ except that $\tt v_{12}.vars(i) = v_1.vars(i) \cup v_2.vars(j_i)$.
%  \item if $\tt gtag(v_1, i) \neq gtag(v_2, j_i)$ then $\tt v_1 \cap v_2 = \emptyset$
% \end{itemize}
% 
%  \item if $\tt v1.greachfrom(i) = \emptyset$ and $\tt greachfrom(j_i) = \emptyset$ then 
%  \begin{itemize}
%  \item if $\tt gtag(v_1, i) = gtag(v_2, j_i)$ then $\tt v_1 \cap v_2 = v_{12}$ where $\tt v_{12}$ is same as $\tt v_1$ except that $\tt v_{12}.vars(i) = v_1.vars(i) \cup v_2.vars(j_i)$.
%  \item if $\tt gtag(v_1, i) \neq gtag(v_2, j_i)$ then $\tt v_1 \cap v_2 = \{v_1,v_2\}$
% \end{itemize}
% 
% \end{itemize}
\paragraph{Local Abstract Transformers:} Firstly, let us show the abstract transformer on the set of fragment $\tt S$ in the fragment of the concurrent thread. Let $\tt kill(x,v)$ be the procedure to remove $\tt x$ if it exists from $\tt v.{\tt reachfrom}(\tt i)$, $\tt v.{\tt reachfrom}(\tt o)$, $\tt v.{\tt reachto}(\tt i)$, $\tt v.{\tt reachto}(\tt o)$, $\tt v.{\tt vars}(\tt i)$, and $\tt v.{\tt vars}(\tt o)$. For each program statement, let $\tt S_{post}$ be the set of fragments after executing the statement. Let $\tt S_{post}$ be initialized as the empty set.
\begin{itemize}

\item ${\tt x := y}$: The transformer is performed as following: For each fragment $\frag_y \in S$ where $\tt y \in \frag_y.{\tt vars}(\tt i)$,
\begin{enumerate}
\item  for each fragment $\tt v$ where $\tt v \hookrightarrow_S \frag_y$, create $\tt v'$ which is same as $\tt v$ except that
\begin{itemize}
\item $\pvarsat{\inat{\frag'}} = \pvarsat{\inat{\frag}} \setminus \set{x}$,
\item $\pvarsat{\outat{\frag'}} = \pvarsat{\outat{\frag}} \cup \set{x}$,
\item $\reachfromat{\inat{\frag'}} = \reachfromat{\inat{\frag}} \setminus \set{x}$ if $\tt x$ is a global variable,
\item $\reachtoat{\inat{\frag'}} = \reachtoat{\inat{\frag}} \cup \set{x}$ if $\tt x$ is a global variable,
 \item $\reachfromat{\outat{\frag'}} = \reachfromat{\outat{\frag}} \cup \set{x}$ if $\tt x$ is a global variable,
\item $\reachtoat{\outat{\frag'}} = \reachtoat{\outat{\frag}} \cup \set{x}$ if $\tt x$ is a global variable,
\end{itemize}
and add $\tt v'$ to $\tt S_{post}$,
\item for each fragment $\tt v$ where $\tt v \overset{+}\hookrightarrow_S \frag_y$, create $\tt v'$ which is same as $\tt v$ except that
\begin{itemize}
\item $\pvarsat{\inat{\frag'}} = \pvarsat{\inat{\frag}} \setminus \set{x}$,
\item $\pvarsat{\outat{\frag'}} = \pvarsat{\outat{\frag}} \setminus \set{x}$,
\item $\reachfromat{\inat{\frag'}} = \reachfromat{\inat{\frag}} \setminus \set{x}$ if $\tt x$ is a global variable,
\item $\reachtoat{\inat{\frag'}} = \reachtoat{\inat{\frag}} \cup \set{x}$ if $\tt x$ is a global variable,
\item $\reachfromat{\outat{\frag'}} = \reachfromat{\outat{\frag}} \setminus \set{x}$ if $\tt x$ is a global variable,
\item $\reachtoat{\outat{\frag'}} = \reachtoat{\outat{\frag}} \cup \set{x}$ if $\tt x$ is a global variable,
\end{itemize}
and add $\tt v'$ to $\tt S_{post}$,
\item for each fragment $\tt v_3$ where $\frag_y \overset{*}\hookrightarrow_S v$, create $\tt v'$ which is same as $\tt v$ except that
\begin{itemize}
\item $\pvarsat{\inat{\frag'}} = \pvarsat{\inat{\frag}} \setminus \set{x}$,
\item $\pvarsat{\outat{\frag'}} = \pvarsat{\outat{\frag}} \setminus \set{x}$,
\item $\reachfromat{\inat{\frag'}} = \reachfromat{\inat{\frag}} \cup \set{x}$ if $\tt x$ is a global variable,
\item $\reachtoat{\inat{\frag'}} = \reachtoat{\inat{\frag}} \setminus \set{x}$ if $\tt x$ is a global variable,

 \item $\reachfromat{\outat{\frag'}} = \reachfromat{\outat{\frag}} \cup \set{x}$ if $\tt x$ is a global variable,
 \item $\reachtoat{\outat{\frag'}} = \reachtoat{\outat{\frag}} \setminus \set{x}$ if $\tt x$ is a global variable,

\end{itemize}
and add $\tt v'$ to $\tt S_{post}$,
\item for each fragment $\tt v$ where $\frag_y \overset{*+}\leftrightarrow_S v$, create $\tt v'$ which is same as $\tt v$ except that
\begin{itemize}
\item $\pvarsat{\inat{\frag'}} = \pvarsat{\inat{\frag}} \setminus \set{x}$,
\item $\pvarsat{\outat{\frag'}} = \pvarsat{\outat{\frag}} \setminus \set{x}$,
\item $\reachfromat{\inat{\frag'}} = \reachfromat{\inat{\frag}} \setminus \set{x}$ if $\tt x$ is a global variable,
\item $\reachtoat{\inat{\frag'}} = \reachtoat{\inat{\frag}} \setminus \set{x}$ if $\tt x$ is a global variable,
\item $\reachfromat{\outat{\frag'}} = \reachfromat{\outat{\frag}} \setminus \set{x}$ if $\tt x$ is a global variable,
\item $\reachtoat{\outat{\frag'}} = \reachtoat{\outat{\frag}} \setminus \set{x}$ if $\tt x$ is a global variable,
\end{itemize}
and add $\tt v'$ to $\tt S_{post}$,
\item for each fragment $\tt v$ where $\frag_y \overset{*\circ}\leftrightarrow_S v$, create $\tt v'$ which is same as $\tt v$ except that
\begin{itemize}
\item $\pvarsat{\inat{\frag'}} = \pvarsat{\inat{\frag}} \setminus \set{x}$,
\item $\pvarsat{\outat{\frag'}} = \pvarsat{\outat{\frag}} \setminus \set{x}$,
\item $\reachfromat{\inat{\frag'}} = \reachfromat{\inat{\frag}} \setminus \set{x}$ if $\tt x$ is a global variable,
\item $\reachtoat{\inat{\frag'}} = \reachtoat{\inat{\frag}} \setminus \set{x}$ if $\tt x$ is a global variable,
\item $\reachfromat{\outat{\frag'}} = \reachfromat{\outat{\frag}} \cup \set{x}$ if $\tt x$ is a global variable,
\item $\reachtoat{\outat{\frag'}} = \reachtoat{\outat{\frag}} \setminus \set{x}$ if $\tt x$ is a global variable,

\end{itemize}
and add $\tt v'$ to $\tt S_{post}$,
\item create $\tt v'$ which is same as $\frag_y$ except that
\begin{itemize}
\item $\pvarsat{\inat{\frag'}} = \pvarsat{\inat{\frag}} \cup \set{x}$,
\item $\pvarsat{\outat{\frag'}} = \pvarsat{\outat{\frag}} \setminus \set{x}$,
\item $\reachtoat{\inat{\frag'}} = \reachtoat{\inat{\frag}} \cup \set{x}$ if $\tt x$ is a global variable,
 \item $\reachfromat{\inat{\frag'}} = \reachfromat{\inat{\frag}} \cup \set{x}$ if $\tt x$ is a global variable,
\item $\reachfromat{\outat{\frag'}} = \reachfromat{\outat{\frag}} \cup \set{x}$ if $\tt x$ is a global variable,
\item $\reachtoat{\outat{\frag'}} = \reachtoat{\outat{\frag}} \setminus \set{x}$ if $\tt x$ is a global variable,
\end{itemize}
and add $\tt v'$ to $\tt S_{post}$,
%\item for each fragment $\tt v'$ in $\tt S_{post}$, assign $\tt v'.data.(i).(y.d)$ to $\tt v'.data.(i).(x.d)$ for $\tt i \in \{i, o\}$.
\end{enumerate}

	
 \end{itemize}
