\section{Semantics}
\label{semantics:section}
In this section, we formalize the semantics of programs and
observers, as introduced in the preceding sections.
%
We do this in several steps.
%
First, we define the state of the heap and the transition
relation induced by a single thread.
%
From this we derive the semantics of the program and use it to
define its history set.
%
We then define the semantics of observers.
Finally, we define the product of the augmented program and the observer.
This product will ensure that observer
On this basis, we can formally state and prove the correctness of
our approach, as Theorem~\ref{thm:soundness}.
%% Finally, we introduce  the soundness theorem 
%% which states that linearizability of the original program
%% is implied by the fact that the monitor never enters
%% its ``error'' state.
%

For a function $\fun:A\mapsto B$ from a set $A$ to
a set $B$, we use 
$\fun[\aelem_1\assigned\belem_1,\ldots,\aelem_\nn\assigned\belem_\nn]$ 
to denote the function
$\fun'$ such that $\fun'(\aelem_\ii)=\belem_\ii$ and 
$\fun'(\aelem)=\fun(\aelem)$ if 
$\aelem\not\in\set{\aelem_1,\ldots,\aelem_\nn}$.
%

Below, we assume a program $\prog$  with a set 
$\glvarset$ of global variables,
a data structure $\dstruct$ specified as an observer $\observer$.
%
%% We assume that each cell contains a set $\fieldset$ of fields
%% where each field is either a lock or a data field.
We assume that each thread $\thread$ executes one method
denoted $\methodof\thread$.
%


\paragraph{Heaps.}
A {\it heap (state)} is a tuple $\heap=\heaptuple$, where
(i)
$\cellset$ is a finite set of cells, including the two special cells
$\nullconst$ and $\dangconst$ (dangling);
%
we define $\mcellset=\cellset-\set{\nullconst,\dangconst}$,
%
(ii)
For each pointer field $\pfield_i$ there is a total function
$\pfield_i:\mcellset\to\cellset$ is a total
function that defines to which cell that pointer field points,
%% the next-pointer relation, i.e., $\priflesuccof{\cell}=\cell'$ means that cell $\cell$ points to $\cell'$,
(iii)
$\glval:\glvarset\to\cellset$ maps the global (pointer)
variables to their values, and
(iv)
$\cellval:\cellset\times\fieldset\to\fset\cup\intgrs$
maps data and lock fields of each cell to their values.
%
We assume that the heap is initialized by
an initialization thread, and let $\initheap$ denote the initial
heap.

\paragraph{Threads.}
A {\it local state} $\lstate$ of a thread $\thread$
wrt.\ a heap $\heap$ defines 
the values of its local variables, including the program counter
{\tt pc} and the input parameter for the method executed by $\thread$.
In addition, there is the special initial state $\idlestate$,
and terminated state $\terminatedstate$.
%
A {\em projection} $\proj$ is a pair
$\tuple{\lstate,\heap}$, where $\lstate$
is a local state wrt.\  the heap $\heap$.
%
A thread $\thread$ induces a transition relation $\movesto\thread{}$
on projections.
%
Initially, $\thread$ is in the state $\idlestate$.
%
It becomes active by executing a transition
$\tuple{\idlestate,\heap}
\movesto\thread{{\call\thread\mname\indata}}
\tuple{\initlstateof\thread,\heap}$,
labeled by a call action with $\thread$ as the event identifier and
$\mname=\methodof\thread$.
It moves to an initial local state $\initlstateof\thread$
where $\indata$ is the value of its input parameter,
the value of {\tt pc} is the label of the first statement of the method, and
the other local variables are undefined.
%
Thereafter, $\thread$ executes statements one after one.
%
We write $\proj\movesto\thread\lbl\proj'$ 
to denote that the statement labeled by $\lbl$ can be executed
from $\proj$, yielding $\proj'$.
Note that the next move of $\thread$ is uniquely determined
by $\proj$, since $\thread$ cannot access
the local states of other threads.
%
Finally, $\thread$ terminates
when executing its {\tt return} command
giving rise to a transition
$\tuple{\lstate,\heap}\movesto\thread{\return\thread\mname\outdata}
\tuple{\terminatedstate,\heap}$, labeled 
by a return action with $\thread$ as event identifier,
$\mname=\methodof\thread$, and $\outdata$
as the returned value.
%

\paragraph{Programs.}
A {\it configuration} of a program
$\prog$ is a tuple
$\tuple{\athreads,\lstatemapping,\heap}$ where
$\athreads$ is a set of threads,
$\heap$ is a heap, and
$\lstatemapping$ is a {\it local state mapping}
over $\athreads$ wrt.\ $\heap$
that maps each thread $\thread\in\athreads$ to its
local state $\lstatemappingof\thread$ wrt.\ $\heap$.
%
%% We use $\confsetof\prog$ to denote the set of configurations
%% of $\prog$.
%
The initial configuration
%% $\initconfof\prog\in\confsetof\prog$ 
$\initconfof\prog$ 
is the pair
$\tuple{\initlstatemapping,\initheap}$,
where
%% $\initheap$ is the initial heap, and
$\initlstatemappingof\thread=\idlestate$ for each
$\thread\in\athreads$,
i.e.,
$\prog$ starts its execution from a configuration with an initial
heap, and with each thread in its initial local state.
%
A program $\prog$ induces a transition relation
$\movesto\prog{}$ where each step corresponds
to one move of a single thread.
%
This is captured by the rules 
{\tt P1}, {\tt P2}, and {\tt P3} of
Fig.~\ref{semantics:fig} 
(here $\emptyword$ denotes the empty word.)
%
Note that the only visible transitions are those
corresponding to call and return actions.
%
A {\it history} of $\prog$ is a sequence of form
$\action_1\action_2\cdots\action_\nn$, such that there is a
sequence
$
\conf_0\movesto\prog{\action_1}
\conf_1\movesto\prog{\action_2}
\cdots
\conf_{\nn-1}\movesto\prog{\action_\nn}\conf_\nn
$
of transitions from the initial configuration
$\conf_0=\initconfof\prog$.
%
We define $\historyof\prog$ to be the set of histories of $\prog$.
%
We say that $\prog$ is linearizable wrt.\ 
$\dstruct$ if the set 
$\historyof\prog$ is linearizable wrt.\ $\dstruct$.

%% \paragraph{Controllers.}
%% The controller of a thread $\thread$ induces a transition
%% relation $\movesto{\ctrlof\thread}{}$, which depends on the current
%% view $\view$ of $\thread$. 
%% %
%% For an expression $\eexpr$ denoting an operation, broadcast message, or
%% condition, let $\view(\eexpr)$ denote the result of evaluating $\eexpr$ in
%% the state $\view$.
%% For a predicate $\cnd$, we let $\view\models\cnd$ denote that
%% $\view(\cnd)$ is true.
%% Transitions are defined by the set $\cruleset{\methodof\thread}$
%% of rules ($\cruleset\thread$ for short) associated with the method of $\thread$.
%% Rule {\tt C1} describes transitions
%% induced by a triggered rule, and
%% states that if $\cnd$ is true under $\view$,
%% then the controller emits the operation obtained by instantiating
%% $\oevent$ and broadcasts the message obtained by instantiating
%% $\sevent$, according to $\view$.
%% %
%% Rule {\tt C2} describes transitions induced by receiving a message
%% $\event'$, and states that if $\revent$ equals the value of $\event'$ in $\view$, and $\cnd$ evaluates to true in $\view$, then
%% the operation $\view(\oevent)$ is emitted.
%% %
%% %% For a predicate $\cnd$ on the set of local variables of the thread,
%% %% we write $\view\models\cnd$ to denote that
%% %% $\cnd$ evaluates to true under $\view$.
%% %% %
%% %% %
%% %% For a send event $\event$, 
%% %% we use $\view(\sevent)$ to denote the ground send event we get
%% %% by replacing each variable in $\event$ by its value
%% %% according to $\view$.
%% %% %
%% %% For an observer event $\oevent$, we define $\view(\oevent)$ 
%% %% analogously.
%% %
%% %
%% %% \input rules-semantics
%% %
%% %
%% In both cases, the controller also emits to the observer the identity
%% $\thread$ of the thread.
%% %
%% This identity will be used for bookkeeping by the monitor (see below.)
%% %

%% \paragraph{Augmented Threads.}
%% An {\it augmented view} $\tuple{\cstate,\lstate,\heap}$ extends a view
%% $\tuple{\lstate,\heap}$ with a state $\cstate$ of the controller.
%% We define a transition relation $\movesto{\augof\thread}{}$ over
%% augmented views
%% that describes the behavior of a thread $\thread$ when augmented
%% with its controller.
%% %
%% For a label $\lbl$ of a statement we write
%% $\triggered\lbl$ to denote that the corresponding statement is triggered.
%% %
%% Rule {\tt TC1} states that a non-triggered statement by the thread
%% does not affect the state of the controller, and does not emit or broadcast
%% anything. 
%% %% %
%% %% In such a case the augmented thread performs a transition
%% %% in which only the local state of $\thread$ and the heap will change.
%% %% %
%% %% In particular, the controller state will not change and
%% %% no messages will be emitted/broadcast to the observer 
%% %% or to the other threads.
%% %
%% Rule {\tt TC2} states a return action by the thread. Rule {\tt TC3} states that when
%% the thread $\thread$ performs a triggered statement,
%% %
%% then the controller will emit an operation to the observer and broadcast
%% messages to the other threads.
%% %
%% The reception of broadcast messages will be described in Rule {\tt R}.
%% %% %
%% %% This reflects the fact that such messages are not emitted
%% %% by the thread or its controller.
%% %% %
%% %% Such rule are part of the semantics for the augmented 
%% %% program as describe below.

%% \paragraph{Augmented Programs.}
%% %
%% An {\em augmented program} $\augprog$ is obtained from a program $\prog$
%% by augmenting each thread with its controller.
%% %
%% A {\it configuration} $\conf$ of $\augprog$ is a tuple
%% $\tuple{\athreads,\cstatemapping,\lstatemapping,\heap}$, which extends a
%% configuration of $\prog$ by
%% %% where $\athreads$ is a set of threads, $\heap$ is a heap,
%% a {\it controller state mapping}, $\cstatemapping$, which maps
%% each thread $\thread\in\athreads$ to the state of its controller.
%% %% of the controller of $\thread$, and
%% %% $\lstatemapping$ is a {\it local state mapping}
%% %% over $\athreads$ wrt. $\heap$
%% %% that defines, for each thread $\thread\in\athreads$,  the local state 
%% %% $\lstatemappingof\thread$ of $\thread$ over $\heap$.
%% %
%% We define the size $\sizeof\conf=\sizeof\athreads$ and
%% define $\athreadsof\conf=\athreads$.
%% %
%% %
%% We use $\confsetof\augprog$ to denote the set of configurations
%% of $\augprog$.
%% %
%% Transitions of $\augprog$ are performed by the threads in $\athreads$.
%% %
%% The rule {\tt Q1} describes return action or a non-triggered statement in which only its local state and the heap
%% may change.
%% %
%% Rule {\tt Q2} describes the case where a thread
%% $\senderthread$ executes a triggered statement and hence also
%% broadcasts a message to the other threads.
%% %
%% Before describing this rule, we describe the effect
%% of message reception.
%% %
%% Consider a set $\athreads$ of threads,
%% a heap $\heap$,
%% a controller state mapping $\cstatemapping$ over $\athreads$,
%% and a local state mapping $\lstatemapping$
%% over $\athreads$ wrt.\ $\heap$.
%% %
%% We will define a relation that extracts the set of threads
%% in $\athreads$ that can receive $\event$ from $\senderthread$.
%% %
%% For $\ord\in\set{\before,\after}$,
%% define
%% $\enabled\athreads\cstatemapping\lstatemapping\heap\event\ord\senderthread$
%% to be the set of threads
%% $\thread\in\athreads$  such that
%% (i)
%% $\thread\neq\senderthread$, i.e., a receiver thread should be different from the sender thread,
%% and
%% (ii)
%% $\tuple{\lstatemapping(\thread),\heap}:
%% \cstatemapping(\thread)\movesto{\ctrlof\thread}{\tuple{\thread,\oevent}\mid\rcv{\tuple{\event,\ord}}}\cstate'$
%% for some $\cstate'$ and $\oevent$,
%% i.e., the controller of $\thread$ has an enabled
%% receiving rule that can receive $\event$, with the ordering flag
%% given by $\ord$.
%% %
%% The rule {\tt R} describes the effect of receiving a message
%% that has been broadcast by a sender thread $\senderthread$.
%% %
%% Each thread that can receive the message will do so, and
%% the thread potentially will all change their controller state
%% and emit an operation.
%% %
%% Notice that the receiver threads are collected in a set and therefore
%% the rule allows any ordering of the receiving threads.
%% %
%% We are now ready to explain the rule {\tt Q2}.
%% %
%% The sender thread $\senderthread$ broadcast a message $\event_2$
%% that will be received by other threads according to rule {\tt R}.
%% %
%% The sender thread and the receiver threads, all emit operations
%% ($\event_1$ and $\eventseq$ respectively).
%% %
%% Depending on the order specified in the specification of the controller,
%% a receiving thread may linearize {\it before} or {\it after}  $\senderthread$.
%% %%
%% Notice that the receiver threads only change the local states of their
%% controllers and the state of the observer.
%% %
%% An {\it initial configuration} of $\augprog\in\confsetof\augprog$ is of the form
%% $\tuple{\athreads,\initcstatemapping,\initlstatemapping,\initheap}$
%% where, for each $\thread\in\athreads$,
%%  $\initlstatemappingof\thread$ is the initial state of
%% the  controller of $\thread$, and
%% $\initlstatemappingof\thread=\idlestate$.

\paragraph{Observers.}

Let $\observer=\otuple$.
%
A configuration $\conf$ of $\observer$
is a pair $\tuple{\ostate,\val}$
where $\ostate\in\ostateset$ is an observer state and
$\val$ assigns a value to each register in $\observer$.
%
%
We use $\confsetof\observer$ to denote the set of configurations
of $\observer$.
%
The transition relation $\movesto\observer{}$
of the observer is described by the rules
{\tt O1} and {\tt O2} of Fig.~\ref{semantics:fig}.
%
Rule {\tt O1} states that
if there is an enabled  transition that is labeled by a given action
  then such a transition may be performed. 
%
Rule {\tt O2} states that if there is no
such a transition, the observer remains in its current state.
%
Notice that the rules imply that the register
values never change during a run of the observer. 
%
An initial configuration of $\observer\in\confsetof\observer$
is of the form $\tuple{\oinitstate,\val}$.


\paragraph{Monitors.}
%
The monitor $\monitor$ augments the observer. It keeps track of 
%% The monitor 
the observer state, and the sequence of operations and
call and return actions generated by the threads of the augmented program.
It reports
an error if one of the following happens:
(i)
a thread terminates without having linearized,
(ii) 
the parameters of call and return actions are different from those of
the corresponding emitted operation,
%
(iii)
a thread linearizes although it has previously changed the state
of the observer, or
%
(iv)
$\augprog$ generates a trace which violates the specification of the data
structure $\dstruct$.
%
The monitor keeps track of these conditions as follows.
%
A {\it configuration} of $\monitor$ is either
(i) a tuple
$\tuple{\athreads,\conf,\linstatus,\rvalstatus}$
where $\athreads$ is a set of threads,
$\conf$ is an observer configuration,
$\linstatus:\athreads\mapsto\set{0,1,2}$, and
$\rvalstatus:\athreads\mapsto\intgrs\cup\fset$;
or
(ii) the special state $\errorstate$.
%
For a thread $\thread\in\athreads$, 
the values $0,1,2$ of 
$\linstatusof\thread$ have the following interpretations:
$0$ means that $\thread$ has not linearized yet,
$1$ means that $\thread$ has linearized but not changed the state of the observer,
and
$2$ means that $\thread$ has both linearized and changed 
the state of the observer.
%
Furthermore, 
$\rvalstatusof\thread$ stores the value returned by $\thread$ the latest
time it performed a linearization.
\td{We also need to store the call values!!}
%
In case $\conf$ is of the first form,
we define $\athreadsof\conf=\athreads$.
%
We use $\confsetof\monitor$ to denote the set of configurations
of $\monitor$.
%
The rules
{\tt M1} through {\tt M8}
describe the transition relation
$\movesto\monitor{}$ induced by $\monitor$. 
%
Rule {\tt M1} describes the scenario when the monitor 
detects an operation $\mname(\indata,\outdata)$ performed by 
a thread $\thread$ such that $\observer$ does not change its state.
%
In such a case, the flag $\linstatus$ for $\thread$
is updated to $1$ (the thread has linearized but not changed the state
of the observer), and the latest value $\outdata$
returned by $\thread$ is stored $\rvalstatus$.
%
Rule {\tt M2} is similar except that the observer changes state
and hence the flag $\linstatus$ is updated to $2$ for $\thread$.
%
Notice that in both rules, a premise is that $\thread$ has not
changed the observer state previously (the flag $\linstatus$ for
$\thread$ is different from $2$.)
%
Rules 
{\tt M3},
{\tt M4},
{\tt M5}, and
{\tt M6}
describe the conditions (i), (ii), (iii), and (iv)
respectively described
above that lead to the error state.
%
The rule {\tt M7} describes the fact that the error state is a sink,
i.e., once $\monitor$ enters that state then it will never leave it.
%
Finally, rule {\tt M8} describes the reflexive transitive closure of
the relation $\movesto\monitor{}$.
%
An initial configuration of $\monitor\in\confsetof\monitor$
is of the form $\tuple{\athreads,\initconf,\initlinstatus,\initrvalstatus}$
where $\initconf\in\initconfsetof\observer$ is an initial 
configuration of $\observer$,
$\initlinstatusof\thread=0$ and 
$\initrvalstatusof\thread$ is undefined for every thread $\thread\in\athreads$.

\paragraph{Product.}
We use $\system=\augprog\compose\monitor$ to denote the system obtained by
running $\augprog$ and $\monitor$ together.
%
A (initial) configuration of $\system$ is of the form
$\tuple{\conf_1,\conf_2}$ where $\conf_1$ and $\conf_2$
are (initial) configurations of $\augprog$ and $\monitor$ respectively
such that $\athreadsof{\conf_1}=\athreadsof{\conf_2}$.
%
%
We use $\confsetof{\system}$ 
to denote the set of configurations
of $\system$.
%
The induced transition relation $\movesto{\system}{}$
is described by
rule {\tt QM} of Fig.~\ref{semantics:fig}.
%
Intuitively, in the composed systems, the augmented program and the monitor
synchronize through the actions they produce.
%
For a configuration $\conf\in\confsetof\system$, we define
$\postof\conf=\setcomp{\conf'}{\conf\movesto\system{}\conf'}$. For a set C of system configurations, we define $\postof{\confs} = \bigcup_{\conf \in \confs} \postof\conf$. 
We say that that $\errorstate$ is reachable in 
$\system$ if
$\conf{\movesto{\system}{}}^*\tuple{\conf',\errorstate}$
for an initial configuration $\conf$ of 
$\system$ and configuration $\conf'$ of $\augprog$.

\paragraph{Verifying Linearizability.}
The correctness of this approach is established in the following theorem.

\begin{theorem}
\label{thm:soundness}
%\todo{Define what reachable means.}
If $\errorstate$ is not reachable in  $\system$ 
then $\prog$ is linearizable wrt. $\dstruct$.
\end{theorem}
\label{Proof}
\textit{Proof}: To prove $\prog$ is linearizable wrt. $\dstruct$. We must establish that for any history $h$ of $\prog$, there exists a history $h'$ such that $h'$ is linearization of $h$ and $h'$ is valid wrt. $\dstruct$. So, consider a history $h$ of $\prog$. From Condition (i) in the paragraph about monitors in section 5, $h$ is a complete history. Then from Conditions (ii) and (iii), each action in $h$ has a LP and whose call and return parameters are consistent with the corresponding emitted operation by the controller. Therefore, there exists a complete sequential history $h'$ which is a permutation of $h$ whose actions are ordered according to the order of their LPs in $h$. Therefore $h'$ is linearization of $h$. From Condition (iv), $h'$ is $valid$ wrt. $\dstruct$. Therefore, $h$ is $linearizable$ wrt. $\dstruct$. $\Box$




