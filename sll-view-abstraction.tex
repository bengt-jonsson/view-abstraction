\begin{center}
\begin{figure}

\begin{tikzpicture}[]%show background rectangle]
\node[rounded corners,draw,name=cell1,minimum width=15pt, minimum height=15pt]{};
%\node[minimum width=15pt, minimum height=15pt,anchor=north west,font=\tiny,inner sep=0pt,scale=0.8] at ($(cell1.north west)+(0.5pt,0pt)$){{--$\infty$}};
%\node[minimum width=15pt, minimum height=15pt,anchor=north west,font=\tiny,inner sep=0pt]at ($(cell1.north west)+(0pt,-10pt)$){\cross};
%\node[minimum width=15pt, minimum height=15pt,anchor=north west,font=\tiny,inner sep=0pt]at ($(cell1.north west)+(0pt,-20pt)$){\cross};
\node[name=succ1,circle,fill,minimum size=2pt,inner sep=0pt,outer sep=0pt] at ($(cell1.east)+(-3pt,0pt)$) {};
%\draw ($(cell1.north west)+(0pt,-10pt)$) -- ($(cell1.north west)+(10pt,-10pt)$);
%\draw ($(cell1.north west)+(0pt,-20pt)$) -- ($(cell1.north west)+(10pt,-20pt)$);
%\draw ($(cell1.north west)+(10pt,0pt)$) -- ($(cell1.south west)+(10pt,0pt)$);
\node[anchor=south,font=\tiny,align=center] at ($(cell1.north)+(0pt,-2pt)$) {{\tt H}};
\node[rounded corners,draw,name=cell2,minimum width=15pt, minimum height=15pt,anchor=west]
at ($(cell1.east)+(10pt,0pt)$){};
%\node[minimum width=15pt, minimum height=15pt,anchor=north west,font=\tiny,inner sep=0pt] at ($(cell2.north west)+(0pt,0pt)$){{$8$}};
%\node[minimum width=15pt, minimum height=15pt,anchor=north west,font=\tiny,inner sep=0pt]at ($(cell2.north west)+(0pt,-10pt)$){\cross};
%\node[minimum width=15pt, minimum height=15pt,anchor=north west,font=\tiny,inner sep=0pt]at ($(cell2.north west)+(0pt,-20pt)$){\cross};
\node[name=succ2,circle,fill,minimum size=2pt,inner sep=0pt,outer sep=0pt] at ($(cell2.east)+(-3pt,0pt)$) {};
%\draw ($(cell2.north west)+(0pt,-10pt)$) -- ($(cell2.north west)+(10pt,-10pt)$);
%\draw ($(cell2.north west)+(0pt,-20pt)$) -- ($(cell2.north west)+(10pt,-20pt)$);
%\draw ($(cell2.north west)+(10pt,0pt)$) -- ($(cell2.south west)+(10pt,0pt)$);

\draw[->] (succ1) -- (cell2);
\node[rounded corners,draw,name=cell3,minimum width=15pt, minimum height=15pt,anchor=west]
at ($(cell2.east)+(10pt,0pt)$){};
%\node[minimum width=15pt, minimum height=15pt,anchor=north west,font=\tiny,inner sep=0pt] at ($(cell3.north west)+(0pt,0pt)$){{$9$}};
%\node[minimum width=15pt, minimum height=15pt,anchor=north west,font=\tiny,inner sep=0pt]at ($(cell3.north west)+(0pt,-10pt)$){\cross};
%\node[minimum width=15pt, minimum height=15pt,anchor=north west,font=\tiny,inner sep=0pt]at ($(cell3.north west)+(0pt,-20pt)$){\cross};
\node[name=succ3,circle,fill,minimum size=2pt,inner sep=0pt,outer sep=0pt] at ($(cell3.east)+(-3pt,0pt)$) {};
%\draw ($(cell3.north west)+(0pt,-10pt)$) -- ($(cell3.north west)+(10pt,-10pt)$);
%\draw ($(cell3.north west)+(0pt,-20pt)$) -- ($(cell3.north west)+(10pt,-20pt)$);
%\draw ($(cell3.north west)+(10pt,0pt)$) -- ($(cell3.south west)+(10pt,0pt)$);

\node[rounded corners,draw,name=cell5,minimum width=15pt, minimum height=15pt,anchor=west]
at ($(cell3.east)+(10pt,0pt)$){};
%\node[minimum width=15pt, minimum height=15pt,anchor=north west,font=\tiny,inner sep=0pt] at ($(cell5.north west)+(0pt,0pt)$){{$\infty$}};
%\node[minimum width=15pt, minimum height=15pt,anchor=north west,font=\tiny,inner sep=0pt] at ($(cell5.north west)+(0pt,-10pt)$){\cross};
%\node[minimum width=15pt, minimum height=15pt,anchor=north west,font=\tiny,inner sep=0pt] at ($(cell5.north west)+(0pt,-20pt)$){\cross};
%\draw ($(cell5.north east)+(-1pt,-2pt)$) -- ($(cell5.south east)+(-6pt,0pt)$);
%\draw ($(cell5.north east)+(-6pt,0pt)$) -- ($(cell5.south east)+(-1pt,2pt)$);
%\draw ($(cell5.north west)+(0pt,-10pt)$) -- ($(cell5.north west)+(10pt,-10pt)$);
%\draw ($(cell5.north west)+(0pt,-20pt)$) -- ($(cell5.north west)+(10pt,-20pt)$);
%\draw ($(cell5.north west)+(10pt,0pt)$) -- ($(cell5.south west)+(10pt,0pt)$);
\node[anchor=south,font=\tiny,align=center] at ($(cell5.north)+(0pt,-2pt)$) {{${\tt T}$}};
\draw[->] (succ2) -- (cell3);
\draw[->] (succ3) -- (cell5);


%%%%%%%%%%%%%%%%%%%%%%%%%%%%%%%%%%%%%


\node[rounded corners,draw,name=cell22,minimum width=15pt, minimum height=15pt,anchor=west]
at ($(cell1.west)+(10pt,-40pt)$){};
%\node[minimum width=15pt, minimum height=15pt,anchor=north west,font=\tiny,inner sep=0pt] at ($(cell22.north west)+(0pt,0pt)$){{$2$}};
%\node[minimum width=15pt, minimum height=15pt,anchor=north west,font=\tiny,inner sep=0pt]at ($(cell22.north west)+(0pt,-10pt)$){\tick};
%\node[minimum width=15pt, minimum height=15pt,anchor=north west,font=\tiny,inner sep=0pt]at ($(cell22.north west)+(0pt,-20pt)$){\cross};
\node[name=succ22,circle,fill,minimum size=2pt,inner sep=0pt,outer sep=0pt] at ($(cell22.east)+(-3pt,0pt)$) {};
%\draw ($(cell22.north west)+(0pt,-10pt)$) -- ($(cell22.north west)+(10pt,-10pt)$);
%\draw ($(cell22.north west)+(0pt,-20pt)$) -- ($(cell22.north west)+(10pt,-20pt)$);
%\draw ($(cell22.north west)+(10pt,0pt)$) -- ($(cell22.south west)+(10pt,0pt)$);
\node[anchor=north,font=\tiny,align=center] at ($(cell22.south)+(0pt,2pt)$) {${\tt c_1}$};


\node[rounded corners,draw,name=cell23,minimum width=15pt, minimum height=15pt,anchor=west]
at ($(cell22.east)+(10pt,0pt)$){};
%\node[minimum width=15pt, minimum height=15pt,anchor=north west,font=\tiny,inner sep=0pt] at ($(cell23.north west)+(0pt,0pt)$){{$4$}};
%\node[minimum width=15pt, minimum height=15pt,anchor=north west,font=\tiny,inner sep=0pt]at ($(cell23.north west)+(0pt,-10pt)$){\tick};
%\node[minimum width=15pt, minimum height=15pt,anchor=north west,font=\tiny,inner sep=0pt]at ($(cell23.north west)+(0pt,-20pt)$){\cross};
\node[name=succ23,circle,fill,minimum size=2pt,inner sep=0pt,outer sep=0pt] at ($(cell23.east)+(-3pt,0pt)$) {};
%\draw ($(cell23.north west)+(0pt,-10pt)$) -- ($(cell23.north west)+(10pt,-10pt)$);
%\draw ($(cell23.north west)+(0pt,-20pt)$) -- ($(cell23.north west)+(10pt,-20pt)$);
%\draw ($(cell23.north west)+(10pt,0pt)$) -- ($(cell23.south west)+(10pt,0pt)$);
\node[anchor=south,font=\tiny,align=center] at ($(cell23.north)+(0pt,-2pt)$) {};
\draw[->] (succ22) -- (cell23);
\node[rounded corners,draw,name=cell24,minimum width=15pt, minimum height=15pt,anchor=west]
at ($(cell23.east)+(10pt,0pt)$){};
%\node[minimum width=15pt, minimum height=15pt,anchor=north west,font=\tiny,inner sep=0pt] at ($(cell24.north west)+(0pt,0pt)$){{$5$}};
%\node[minimum width=15pt, minimum height=15pt,anchor=north west,font=\tiny,inner sep=0pt] at ($(cell24.north west)+(0pt,-10pt)$){\tick};
%\node[minimum width=15pt, minimum height=15pt,anchor=north west,font=\tiny,inner sep=0pt] at ($(cell24.north west)+(0pt,-20pt)$){\cross};
\node[name=succ24,circle,fill,minimum size=2pt,inner sep=0pt,outer sep=0pt] at ($(cell24.east)+(-3pt,0pt)$) {};
%\draw ($(cell24.north west)+(0pt,-10pt)$) -- ($(cell24.north west)+(10pt,-10pt)$);
%\draw ($(cell24.north west)+(0pt,-20pt)$) -- ($(cell24.north west)+(10pt,-20pt)$);
%\draw ($(cell24.north west)+(10pt,0pt)$) -- ($(cell24.south west)+(10pt,0pt)$);
\draw[->] (succ23) -- (cell24);
\node[rounded corners,draw,name=cell25,minimum width=15pt, minimum height=15pt,anchor=west]
at ($(cell24.east)+(10pt,0pt)$){};
%\node[minimum width=15pt, minimum height=15pt,anchor=north west,font=\tiny,inner sep=0pt] at ($(cell25.north west)+(0pt,0pt)$){{$7$}};
%\node[minimum width=15pt, minimum height=15pt,anchor=north west,font=\tiny,inner sep=0pt] at ($(cell25.north west)+(0pt,-10pt)$){\tick};
%\node[minimum width=15pt, minimum height=15pt,anchor=north west,font=\tiny,inner sep=0pt] at ($(cell25.north west)+(0pt,-20pt)$){\cross};
\node[name=succ25,circle,fill,minimum size=2pt,inner sep=0pt,outer sep=0pt] at ($(cell25.east)+(-3pt,0pt)$) {};
%\draw ($(cell25.north west)+(0pt,-10pt)$) -- ($(cell25.north west)+(10pt,-10pt)$);
%\draw ($(cell25.north west)+(0pt,-20pt)$) -- ($(cell25.north west)+(10pt,-20pt)$);
%\draw ($(cell25.north west)+(10pt,0pt)$) -- ($(cell25.south west)+(10pt,0pt)$);
\node[anchor=south,font=\tiny,align=center] at ($(cell25.north)+(0pt,-2pt)$) {};
\draw[->] (succ24) -- (cell25);
\draw[->] (succ25) .. controls (cell25.north east) .. ($(cell2.south east)+(-2.5pt,0pt)$);

%%%%%%%%%%%%%%%%%%%%%%%%%%%%%%%%%%%%%

\node[rounded corners,draw,name=cell33,minimum width=15pt, minimum height=15pt,anchor=west]
at ($(cell23.west)+(0pt,-40pt)$){};
%\node[minimum width=15pt, minimum height=15pt,anchor=north west,font=\tiny,inner sep=0pt] at ($(cell33.north west)+(0pt,0pt)$){{$3$}};
%\node[minimum width=15pt, minimum height=15pt,anchor=north west,font=\tiny,inner sep=0pt]at ($(cell33.north west)+(0pt,-10pt)$){\tick};
%\node[minimum width=15pt, minimum height=15pt,anchor=north west,font=\tiny,inner sep=0pt]at ($(cell33.north west)+(0pt,-20pt)$){\cross};
\node[name=succ33,circle,fill,minimum size=2pt,inner sep=0pt,outer sep=0pt] at ($(cell33.east)+(-3pt,0pt)$) {};
%\draw ($(cell33.north west)+(0pt,-10pt)$) -- ($(cell33.north west)+(10pt,-10pt)$);
%\draw ($(cell33.north west)+(0pt,-20pt)$) -- ($(cell33.north west)+(10pt,-20pt)$);
%\draw ($(cell33.north west)+(10pt,0pt)$) -- ($(cell33.south west)+(10pt,0pt)$);
\node[anchor=north,font=\tiny,align=center] at ($(cell33.south)+(0pt,2pt)$) {${\tt c_2}$};
\draw[->,out=0,in=270] (succ33) to (cell23);
%%%%%%%%% 6
\node[anchor=north,font=\tiny,align=center] at ($(cell33.south)+(0pt,2pt)$) {${}$};
\node[name=cell333,minimum width=15pt, minimum height=15pt,anchor=west]
at ($(cell33.south)+(-10pt,-10pt)$){(a)};
\end{tikzpicture}\;\;\;\;\;\;\;\;\;\;\;\;\;\;\;\;\;\;\;\;\;\begin{tikzpicture}[]%show background rectangle]
\node[rounded corners,draw,name=cell1,minimum width=15pt, minimum height=15pt]{};
%\node[minimum width=15pt, minimum height=15pt,anchor=north west,font=\tiny,inner sep=0pt,scale=0.8] at ($(cell1.north west)+(0.5pt,0pt)$){{--$\infty$}};
%\node[minimum width=15pt, minimum height=15pt,anchor=north west,font=\tiny,inner sep=0pt]at ($(cell1.north west)+(0pt,-10pt)$){\cross};
%\node[minimum width=15pt, minimum height=15pt,anchor=north west,font=\tiny,inner sep=0pt]at ($(cell1.north west)+(0pt,-20pt)$){\cross};
\node[name=succ1,circle,fill,minimum size=2pt,inner sep=0pt,outer sep=0pt] at ($(cell1.east)+(-3pt,0pt)$) {};
%\draw ($(cell1.north west)+(0pt,-10pt)$) -- ($(cell1.north west)+(10pt,-10pt)$);
%\draw ($(cell1.north west)+(0pt,-20pt)$) -- ($(cell1.north west)+(10pt,-20pt)$);
%\draw ($(cell1.north west)+(10pt,0pt)$) -- ($(cell1.south west)+(10pt,0pt)$);
\node[anchor=south,font=\tiny,align=center] at ($(cell1.north)+(0pt,-2pt)$) {{\tt H}};
\node[rounded corners,draw,name=cell2,minimum width=15pt, minimum height=15pt,anchor=west]
at ($(cell1.east)+(10pt,0pt)$){};
%\node[minimum width=15pt, minimum height=15pt,anchor=north west,font=\tiny,inner sep=0pt] at ($(cell2.north west)+(0pt,0pt)$){{$8$}};
%\node[minimum width=15pt, minimum height=15pt,anchor=north west,font=\tiny,inner sep=0pt]at ($(cell2.north west)+(0pt,-10pt)$){\cross};
%\node[minimum width=15pt, minimum height=15pt,anchor=north west,font=\tiny,inner sep=0pt]at ($(cell2.north west)+(0pt,-20pt)$){\cross};
\node[name=succ2,circle,fill,minimum size=2pt,inner sep=0pt,outer sep=0pt] at ($(cell2.east)+(-3pt,0pt)$) {};
%\draw ($(cell2.north west)+(0pt,-10pt)$) -- ($(cell2.north west)+(10pt,-10pt)$);
%\draw ($(cell2.north west)+(0pt,-20pt)$) -- ($(cell2.north west)+(10pt,-20pt)$);
%\draw ($(cell2.north west)+(10pt,0pt)$) -- ($(cell2.south west)+(10pt,0pt)$);

\draw[->] (succ1) -- (cell2);
\node[rounded corners,draw,name=cell3,minimum width=15pt, minimum height=15pt,anchor=west]
at ($(cell2.east)+(10pt,0pt)$){};
%\node[minimum width=15pt, minimum height=15pt,anchor=north west,font=\tiny,inner sep=0pt] at ($(cell3.north west)+(0pt,0pt)$){{$8$}};
%\node[minimum width=15pt, minimum height=15pt,anchor=north west,font=\tiny,inner sep=0pt]at ($(cell3.north west)+(0pt,-10pt)$){\cross};
%\node[minimum width=15pt, minimum height=15pt,anchor=north west,font=\tiny,inner sep=0pt]at ($(cell3.north west)+(0pt,-20pt)$){\cross};
\node[name=succ3,circle,fill,minimum size=2pt,inner sep=0pt,outer sep=0pt] at ($(cell3.east)+(-3pt,0pt)$) {};
%\draw ($(cell3.north west)+(0pt,-10pt)$) -- ($(cell3.north west)+(10pt,-10pt)$);
%\draw ($(cell3.north west)+(0pt,-20pt)$) -- ($(cell3.north west)+(10pt,-20pt)$);
%\draw ($(cell3.north west)+(10pt,0pt)$) -- ($(cell3.south west)+(10pt,0pt)$);

\node[rounded corners,draw,name=cell5,minimum width=15pt, minimum height=15pt,anchor=west]
at ($(cell3.east)+(10pt,0pt)$){};
%\node[minimum width=15pt, minimum height=15pt,anchor=north west,font=\tiny,inner sep=0pt] at ($(cell5.north west)+(0pt,0pt)$){{$9$}};
%\node[minimum width=15pt, minimum height=15pt,anchor=north west,font=\tiny,inner sep=0pt] at ($(cell5.north west)+(0pt,-10pt)$){\cross};
%\node[minimum width=15pt, minimum height=15pt,anchor=north west,font=\tiny,inner sep=0pt] at ($(cell5.north west)+(0pt,-20pt)$){\cross};
\node[name=succ5,circle,fill,minimum size=2pt,inner sep=0pt,outer sep=0pt] at ($(cell5.east)+(-3pt,0pt)$) {};

%\draw ($(cell5.north west)+(0pt,-10pt)$) -- ($(cell5.north west)+(10pt,-10pt)$);
%\draw ($(cell5.north west)+(0pt,-20pt)$) -- ($(cell5.north west)+(10pt,-20pt)$);
%\draw ($(cell5.north west)+(10pt,0pt)$) -- ($(cell5.south west)+(10pt,0pt)$);

%\draw[->] (succ2) -- (cell3);
\draw[->] (succ3) -- (cell5);








\node[rounded corners,draw,name=cell31,minimum width=15pt, minimum height=15pt,anchor=west]
at ($(cell5.east)+(10pt,0pt)$){};
%\node[minimum width=15pt, minimum height=15pt,anchor=north west,font=\tiny,inner sep=0pt] at ($(cell31.north west)+(0pt,0pt)$){{$9$}};
%\node[minimum width=15pt, minimum height=15pt,anchor=north west,font=\tiny,inner sep=0pt]at ($(cell31.north west)+(0pt,-10pt)$){\cross};
%\node[minimum width=15pt, minimum height=15pt,anchor=north west,font=\tiny,inner sep=0pt]at ($(cell31.north west)+(0pt,-20pt)$){\cross};
\node[name=succ31,circle,fill,minimum size=2pt,inner sep=0pt,outer sep=0pt] at ($(cell31.east)+(-3pt,0pt)$) {};
%\draw ($(cell31.north west)+(0pt,-10pt)$) -- ($(cell31.north west)+(10pt,-10pt)$);
%\draw ($(cell31.north west)+(0pt,-20pt)$) -- ($(cell31.north west)+(10pt,-20pt)$);
%\draw ($(cell31.north west)+(10pt,0pt)$) -- ($(cell31.south west)+(10pt,0pt)$);

\node[rounded corners,draw,name=cell51,minimum width=15pt, minimum height=15pt,anchor=west]
at ($(cell31.east)+(10pt,0pt)$){};
%\node[minimum width=15pt, minimum height=15pt,anchor=north west,font=\tiny,inner sep=0pt] at ($(cell51.north west)+(0pt,0pt)$){{$\infty$}};
%\node[minimum width=15pt, minimum height=15pt,anchor=north west,font=\tiny,inner sep=0pt] at ($(cell51.north west)+(0pt,-10pt)$){\cross};
%\node[minimum width=15pt, minimum height=15pt,anchor=north west,font=\tiny,inner sep=0pt] at ($(cell5.north west)+(0pt,-20pt)$){\cross};
%\draw ($(cell51.north east)+(-1pt,-2pt)$) -- ($(cell51.south east)+(-6pt,0pt)$);
%\draw ($(cell51.north east)+(-6pt,0pt)$) -- ($(cell51.south east)+(-1pt,2pt)$);
%\draw ($(cell51.north west)+(0pt,-10pt)$) -- ($(cell51.north west)+(10pt,-10pt)$);
%\draw ($(cell51.north west)+(0pt,-20pt)$) -- ($(cell51.north west)+(10pt,-20pt)$);
%\draw ($(cell51.north west)+(10pt,0pt)$) -- ($(cell51.south west)+(10pt,0pt)$);
\node[anchor=south,font=\tiny,align=center] at ($(cell51.north)+(0pt,-2pt)$) {{${\tt T}$}};

\draw[->] (succ31) -- (cell51);



%%%%%%%%%%%%%%%%%%%%%%%%%%%%%%%%%%%%%


\node[rounded corners,draw,name=cell22,minimum width=15pt, minimum height=15pt,anchor=west]
at ($(cell1.west)+(10pt,-40pt)$){};
%\node[minimum width=15pt, minimum height=15pt,anchor=north west,font=\tiny,inner sep=0pt] at ($(cell22.north west)+(0pt,0pt)$){{$2$}};
%\node[minimum width=15pt, minimum height=15pt,anchor=north west,font=\tiny,inner sep=0pt]at ($(cell22.north west)+(0pt,-10pt)$){\tick};
%\node[minimum width=15pt, minimum height=15pt,anchor=north west,font=\tiny,inner sep=0pt]at ($(cell22.north west)+(0pt,-20pt)$){\cross};
\node[name=succ22,circle,fill,minimum size=2pt,inner sep=0pt,outer sep=0pt] at ($(cell22.east)+(-3pt,0pt)$) {};
%\draw ($(cell22.north west)+(0pt,-10pt)$) -- ($(cell22.north west)+(10pt,-10pt)$);
%\draw ($(cell22.north west)+(0pt,-20pt)$) -- ($(cell22.north west)+(10pt,-20pt)$);
%\draw ($(cell22.north west)+(10pt,0pt)$) -- ($(cell22.south west)+(10pt,0pt)$);
\node[anchor=north,font=\tiny,align=center] at ($(cell22.south)+(0pt,2pt)$) {${\tt c_1}$};


\node[rounded corners,draw,name=cell23,minimum width=15pt, minimum height=15pt,anchor=west]
at ($(cell22.east)+(10pt,0pt)$){};
%\node[minimum width=15pt, minimum height=15pt,anchor=north west,font=\tiny,inner sep=0pt] at ($(cell23.north west)+(0pt,0pt)$){{$4$}};
%\node[minimum width=15pt, minimum height=15pt,anchor=north west,font=\tiny,inner sep=0pt]at ($(cell23.north west)+(0pt,-10pt)$){\tick};
%\node[minimum width=15pt, minimum height=15pt,anchor=north west,font=\tiny,inner sep=0pt]at ($(cell23.north west)+(0pt,-20pt)$){\cross};
\node[name=succ23,circle,fill,minimum size=2pt,inner sep=0pt,outer sep=0pt] at ($(cell23.east)+(-3pt,0pt)$) {};
%\draw ($(cell23.north west)+(0pt,-10pt)$) -- ($(cell23.north west)+(10pt,-10pt)$);
%\draw ($(cell23.north west)+(0pt,-20pt)$) -- ($(cell23.north west)+(10pt,-20pt)$);
%\draw ($(cell23.north west)+(10pt,0pt)$) -- ($(cell23.south west)+(10pt,0pt)$);
%\node[anchor=south,font=\tiny,align=center] at ($(cell23.north)+(0pt,-2pt)$) {};
\draw[->] (succ22) -- (cell23);
\node[rounded corners,draw,name=cell24,minimum width=15pt, minimum height=15pt,anchor=west]
at ($(cell23.east)+(10pt,0pt)$){};
%\node[minimum width=15pt, minimum height=15pt,anchor=north west,font=\tiny,inner sep=0pt] at ($(cell24.north west)+(0pt,0pt)$){{$4$}};
%\node[minimum width=15pt, minimum height=15pt,anchor=north west,font=\tiny,inner sep=0pt] at ($(cell24.north west)+(0pt,-10pt)$){\tick};
%\node[minimum width=15pt, minimum height=15pt,anchor=north west,font=\tiny,inner sep=0pt] at ($(cell24.north west)+(0pt,-20pt)$){\cross};
\node[name=succ24,circle,fill,minimum size=2pt,inner sep=0pt,outer sep=0pt] at ($(cell24.east)+(-3pt,0pt)$) {};
%\draw ($(cell24.north west)+(0pt,-10pt)$) -- ($(cell24.north west)+(10pt,-10pt)$);
%\draw ($(cell24.north west)+(0pt,-20pt)$) -- ($(cell24.north west)+(10pt,-20pt)$);
%\draw ($(cell24.north west)+(10pt,0pt)$) -- ($(cell24.south west)+(10pt,0pt)$);

\node[rounded corners,draw,name=cell25,minimum width=15pt, minimum height=15pt,anchor=west]
at ($(cell24.east)+(10pt,0pt)$){};
%\node[minimum width=15pt, minimum height=15pt,anchor=north west,font=\tiny,inner sep=0pt] at ($(cell25.north west)+(0pt,0pt)$){{$5$}};
%\node[minimum width=15pt, minimum height=15pt,anchor=north west,font=\tiny,inner sep=0pt] at ($(cell25.north west)+(0pt,-10pt)$){\tick};
%\node[minimum width=15pt, minimum height=15pt,anchor=north west,font=\tiny,inner sep=0pt] at ($(cell25.north west)+(0pt,-20pt)$){\cross};
\node[name=succ25,circle,fill,minimum size=2pt,inner sep=0pt,outer sep=0pt] at ($(cell25.east)+(-3pt,0pt)$) {};
%\draw ($(cell25.north west)+(0pt,-10pt)$) -- ($(cell25.north west)+(10pt,-10pt)$);
%\draw ($(cell25.north west)+(0pt,-20pt)$) -- ($(cell25.north west)+(10pt,-20pt)$);
%\draw ($(cell25.north west)+(10pt,0pt)$) -- ($(cell25.south west)+(10pt,0pt)$);
%\node[anchor=south,font=\tiny,align=center] at ($(cell25.north)+(0pt,-2pt)$) {};
\draw[->] (succ24) -- (cell25);




\node[rounded corners,draw,name=cell241,minimum width=15pt, minimum height=15pt,anchor=west]
at ($(cell25.east)+(10pt,0pt)$){};
%\node[minimum width=15pt, minimum height=15pt,anchor=north west,font=\tiny,inner sep=0pt] at ($(cell241.north west)+(0pt,0pt)$){{$5$}};
%\node[minimum width=15pt, minimum height=15pt,anchor=north west,font=\tiny,inner sep=0pt] at ($(cell241.north west)+(0pt,-10pt)$){\tick};
%\node[minimum width=15pt, minimum height=15pt,anchor=north west,font=\tiny,inner sep=0pt] at ($(cell241.north west)+(0pt,-20pt)$){\cross};
\node[name=succ241,circle,fill,minimum size=2pt,inner sep=0pt,outer sep=0pt] at ($(cell241.east)+(-3pt,0pt)$) {};
%\draw ($(cell241.north west)+(0pt,-10pt)$) -- ($(cell241.north west)+(10pt,-10pt)$);
%\draw ($(cell241.north west)+(0pt,-20pt)$) -- ($(cell241.north west)+(10pt,-20pt)$);
%\draw ($(cell241.north west)+(10pt,0pt)$) -- ($(cell241.south west)+(10pt,0pt)$);

\node[rounded corners,draw,name=cell251,minimum width=15pt, minimum height=15pt,anchor=west]
at ($(cell241.east)+(10pt,0pt)$){};
%\node[minimum width=15pt, minimum height=15pt,anchor=north west,font=\tiny,inner sep=0pt] at ($(cell251.north west)+(0pt,0pt)$){{$7$}};
%\node[minimum width=15pt, minimum height=15pt,anchor=north west,font=\tiny,inner sep=0pt] at ($(cell251.north west)+(0pt,-10pt)$){\tick};
%\node[minimum width=15pt, minimum height=15pt,anchor=north west,font=\tiny,inner sep=0pt] at ($(cell251.north west)+(0pt,-20pt)$){\cross};
\node[name=succ25,circle,fill,minimum size=2pt,inner sep=0pt,outer sep=0pt] at ($(cell25.east)+(-3pt,0pt)$) {};
%\draw ($(cell251.north west)+(0pt,-10pt)$) -- ($(cell251.north west)+(10pt,-10pt)$);
%\draw ($(cell251.north west)+(0pt,-20pt)$) -- ($(cell251.north west)+(10pt,-20pt)$);
%\draw ($(cell251.north west)+(10pt,0pt)$) -- ($(cell251.south west)+(10pt,0pt)$);
%\node[anchor=south,font=\tiny,align=center] at ($(cell251.north)+(0pt,-2pt)$) {};
\draw[->] (succ241) -- (cell251);


%%%%%%%%%%%%%%%%%%%%%%%%%%%%%%%%%%%%%

\node[rounded corners,draw,name=cell331,minimum width=15pt, minimum height=15pt,anchor=west]
at ($(cell23.west)+(0pt,-40pt)$){};
%\node[minimum width=15pt, minimum height=15pt,anchor=north west,font=\tiny,inner sep=0pt] at ($(cell331.north west)+(0pt,0pt)$){{$3$}};
%\node[minimum width=15pt, minimum height=15pt,anchor=north west,font=\tiny,inner sep=0pt]at ($(cell331.north west)+(0pt,-10pt)$){\tick};
%\node[minimum width=15pt, minimum height=15pt,anchor=north west,font=\tiny,inner sep=0pt]at ($(cell331.north west)+(0pt,-20pt)$){\cross};
\node[name=succ331,circle,fill,minimum size=2pt,inner sep=0pt,outer sep=0pt] at ($(cell331.east)+(-3pt,0pt)$) {};
%\draw ($(cell331.north west)+(0pt,-10pt)$) -- ($(cell331.north west)+(10pt,-10pt)$);
%\draw ($(cell331.north west)+(0pt,-20pt)$) -- ($(cell331.north west)+(10pt,-20pt)$);
%\draw ($(cell331.north west)+(10pt,0pt)$) -- ($(cell331.south west)+(10pt,0pt)$);
\node[anchor=north,font=\tiny,align=center] at ($(cell331.south)+(0pt,2pt)$) {${\tt c_2}$};



\node[rounded corners,draw,name=cell341,minimum width=15pt, minimum height=15pt,anchor=west]
at ($(cell331.east)+(10pt,0pt)$){};
%\node[minimum width=15pt, minimum height=15pt,anchor=north west,font=\tiny,inner sep=0pt] at ($(cell341.north west)+(0pt,0pt)$){{$4$}};
%\node[minimum width=15pt, minimum height=15pt,anchor=north west,font=\tiny,inner sep=0pt]at ($(cell341.north west)+(0pt,-10pt)$){\tick};
%\node[minimum width=15pt, minimum height=15pt,anchor=north west,font=\tiny,inner sep=0pt]at ($(cell341.north west)+(0pt,-20pt)$){\cross};
\node[name=succ33,circle,fill,minimum size=2pt,inner sep=0pt,outer sep=0pt] at ($(cell341.east)+(-3pt,0pt)$) {};
%\draw ($(cell341.north west)+(0pt,-10pt)$) -- ($(cell341.north west)+(10pt,-10pt)$);
%\draw ($(cell341.north west)+(0pt,-20pt)$) -- ($(cell341.north west)+(10pt,-20pt)$);
%\draw ($(cell341.north west)+(10pt,0pt)$) -- ($(cell341.south west)+(10pt,0pt)$);
\draw[->] (succ331) -- (cell341);



\node[rounded corners,draw,name=cell3310,minimum width=15pt, minimum height=15pt,anchor=west]
at ($(cell241.west)+(0pt,-40pt)$){};
%\node[minimum width=15pt, minimum height=15pt,anchor=north west,font=\tiny,inner sep=0pt] at ($(cell3310.north west)+(0pt,0pt)$){{$7$}};
%\node[minimum width=15pt, minimum height=15pt,anchor=north west,font=\tiny,inner sep=0pt]at ($(cell3310.north west)+(0pt,-10pt)$){\tick};
%\node[minimum width=15pt, minimum height=15pt,anchor=north west,font=\tiny,inner sep=0pt]at ($(cell3310.north west)+(0pt,-20pt)$){\cross};
\node[name=succ3310,circle,fill,minimum size=2pt,inner sep=0pt,outer sep=0pt] at ($(cell3310.east)+(-3pt,0pt)$) {};
%\draw ($(cell3310.north west)+(0pt,-10pt)$) -- ($(cell3310.north west)+(10pt,-10pt)$);
%\draw ($(cell3310.north west)+(0pt,-20pt)$) -- ($(cell3310.north west)+(10pt,-20pt)$);
%\draw ($(cell3310.north west)+(10pt,0pt)$) -- ($(cell3310.south west)+(10pt,0pt)$);



\node[rounded corners,draw,name=cell3410,minimum width=15pt, minimum height=15pt,anchor=west]
at ($(cell3310.east)+(10pt,0pt)$){};
%\node[minimum width=15pt, minimum height=15pt,anchor=north west,font=\tiny,inner sep=0pt] at ($(cell3410.north west)+(0pt,0pt)$){{$8$}};
%\node[minimum width=15pt, minimum height=15pt,anchor=north west,font=\tiny,inner sep=0pt]at ($(cell3410.north west)+(0pt,-10pt)$){\tick};
%\node[minimum width=15pt, minimum height=15pt,anchor=north west,font=\tiny,inner sep=0pt]at ($(cell3410.north west)+(0pt,-20pt)$){\cross};
\node[name=succ3410,circle,fill,minimum size=2pt,inner sep=0pt,outer sep=0pt] at ($(cell3410.east)+(-3pt,0pt)$) {};
%\draw ($(cell3410.north west)+(0pt,-10pt)$) -- ($(cell3410.north west)+(10pt,-10pt)$);
%\draw ($(cell3410.north west)+(0pt,-20pt)$) -- ($(cell3410.north west)+(10pt,-20pt)$);
%\draw ($(cell3410.north west)+(10pt,0pt)$) -- ($(cell3410.south west)+(10pt,0pt)$);

\draw[->] (succ3310) -- (cell3410);
%%%%%%%%% 6
\node[anchor=north,font=\tiny,align=center] at ($(cell331.south)+(0pt,2pt)$) {${}$};
\node[name=cell333,minimum width=15pt, minimum height=15pt,anchor=west]
at ($(cell33.south)+(30pt,-10pt)$){(b)};
\end{tikzpicture}



\begin{tikzpicture}[]%show background rectangle]
\node[name=cell0]{};
\node[rounded corners,draw,name=cell1,minimum width=15pt, minimum height=15pt] at ($(cell0.west)+(90pt,-50pt)$) {};
%\node[minimum width=15pt, minimum height=15pt,anchor=north west,font=\tiny,inner sep=0pt,scale=0.8] at ($(cell1.north west)+(0pt,0pt)$){{$<$$z$}};
%\node[minimum width=15pt, minimum height=15pt,anchor=north west,font=\tiny,inner sep=0pt]at ($(cell1.north west)+(0pt,-10pt)$){\cross};
%\node[minimum width=15pt, minimum height=15pt,anchor=north west,font=\tiny,inner sep=0pt]at ($(cell1.north west)+(0pt,-20pt)$){\cross};
\node[name=succ1,circle,fill,minimum size=2pt,inner sep=0pt,outer sep=0pt] at ($(cell1.east)+(-3pt,0pt)$) {};
%\draw ($(cell1.north west)+(0pt,-10pt)$) -- ($(cell1.north west)+(10pt,-10pt)$);
%\draw ($(cell1.north west)+(0pt,-20pt)$) -- ($(cell1.north west)+(10pt,-20pt)$);
%\draw ($(cell1.north west)+(10pt,0pt)$) -- ($(cell1.south west)+(10pt,0pt)$);
\node[anchor=south,font=\tiny,align=center] at ($(cell1.north)+(0pt,-2pt)$) {{\tt H}};


\node[rounded corners,draw,name=cell2,minimum width=15pt, minimum height=15pt,anchor=west]
at ($(cell1.east)+(20pt,0pt)$){};
%\node[minimum width=15pt, minimum height=15pt,anchor=north west,font=\tiny,inner sep=0pt] at ($(cell2.north west)+(0pt,0pt)$){{$=$$z$}};
%\node[minimum width=15pt, minimum height=15pt,anchor=north west,font=\tiny,inner sep=0pt]at ($(cell2.north west)+(0pt,-10pt)$){\cross};
%\node[minimum width=15pt, minimum height=15pt,anchor=north west,font=\tiny,inner sep=0pt]at ($(cell2.north west)+(0pt,-20pt)$){\cross};
\node[name=succ2,circle,fill,minimum size=2pt,inner sep=0pt,outer sep=0pt] at ($(cell2.east)+(-3pt,0pt)$) {};
%\draw ($(cell2.north west)+(0pt,-10pt)$) -- ($(cell2.north west)+(10pt,-10pt)$);
%\draw ($(cell2.north west)+(0pt,-20pt)$) -- ($(cell2.north west)+(10pt,-20pt)$);
%\draw ($(cell2.north west)+(10pt,0pt)$) -- ($(cell2.south west)+(10pt,0pt)$);

\draw[->] (succ1) -- (cell2);

\node[rounded corners,draw,name=cell3,minimum width=15pt, minimum height=15pt,anchor=west]
at ($(cell2.east)+(20pt,0pt)$){};
%\node[minimum width=15pt, minimum height=15pt,anchor=north west,font=\tiny,inner sep=0pt] at ($(cell3.north west)+(0pt,0pt)$){{$=$$z$}};
%\node[minimum width=15pt, minimum height=15pt,anchor=north west,font=\tiny,inner sep=0pt]at ($(cell3.north west)+(0pt,-10pt)$){\cross};
%\node[minimum width=15pt, minimum height=15pt,anchor=north west,font=\tiny,inner sep=0pt]at ($(cell3.north west)+(0pt,-20pt)$){\cross};
\node[name=succ3,circle,fill,minimum size=2pt,inner sep=0pt,outer sep=0pt] at ($(cell3.east)+(-3pt,0pt)$) {};
%\draw ($(cell3.north west)+(0pt,-10pt)$) -- ($(cell3.north west)+(10pt,-10pt)$);
%\draw ($(cell3.north west)+(0pt,-20pt)$) -- ($(cell3.north west)+(10pt,-20pt)$);
%\draw ($(cell3.north west)+(10pt,0pt)$) -- ($(cell3.south west)+(10pt,0pt)$);

%\draw[->] (succ2) -- (cell3);


\node[rounded corners,draw,name=cell4,minimum width=15pt, minimum height=15pt,anchor=west]
at ($(cell3.east)+(20pt,0pt)$){};
%\node[minimum width=15pt, minimum height=15pt,anchor=north west,font=\tiny,inner sep=0pt] at ($(cell4.north west)+(0pt,0pt)$){{$>$$z$}};
%\node[minimum width=15pt, minimum height=15pt,anchor=north west,font=\tiny,inner sep=0pt] at ($(cell4.north west)+(0pt,-10pt)$){\cross};
%\node[minimum width=15pt, minimum height=15pt,anchor=north west,font=\tiny,inner sep=0pt] at ($(cell4.north west)+(0pt,-20pt)$){\cross};
\node[name=succ4,circle,fill,minimum size=2pt,inner sep=0pt,outer sep=0pt] at ($(cell4.east)+(-3pt,0pt)$) {};
%\draw ($(cell4.north west)+(0pt,-10pt)$) -- ($(cell4.north west)+(10pt,-10pt)$);
%\draw ($(cell4.north west)+(0pt,-20pt)$) -- ($(cell4.north west)+(10pt,-20pt)$);
%\draw ($(cell4.north west)+(10pt,0pt)$) -- ($(cell4.south west)+(10pt,0pt)$);

\draw[->] (succ3) -- (cell4);





\node[rounded corners,draw,name=cell31,minimum width=15pt, minimum height=15pt,anchor=west]
at ($(cell4.east)+(20pt,0pt)$){};
%\node[minimum width=15pt, minimum height=15pt,anchor=north west,font=\tiny,inner sep=0pt] at ($(cell31.north west)+(0pt,0pt)$){{$>$$z$}};
%\node[minimum width=15pt, minimum height=15pt,anchor=north west,font=\tiny,inner sep=0pt]at ($(cell31.north west)+(0pt,-10pt)$){\cross};
%\node[minimum width=15pt, minimum height=15pt,anchor=north west,font=\tiny,inner sep=0pt]at ($(cell31.north west)+(0pt,-20pt)$){\cross};
\node[name=succ31,circle,fill,minimum size=2pt,inner sep=0pt,outer sep=0pt] at ($(cell31.east)+(-3pt,0pt)$) {};
%\draw ($(cell31.north west)+(0pt,-10pt)$) -- ($(cell31.north west)+(10pt,-10pt)$);
%\draw ($(cell31.north west)+(0pt,-20pt)$) -- ($(cell31.north west)+(10pt,-20pt)$);
%\draw ($(cell31.north west)+(10pt,0pt)$) -- ($(cell31.south west)+(10pt,0pt)$);

%\draw[->] (succ2) -- (cell3);


\node[rounded corners,draw,name=cell41,minimum width=15pt, minimum height=15pt,anchor=west]
at ($(cell31.east)+(20pt,0pt)$){};
%\node[minimum width=15pt, minimum height=15pt,anchor=north west,font=\tiny,inner sep=0pt] at ($(cell41.north west)+(0pt,0pt)$){{$>$$z$}};
%\node[minimum width=15pt, minimum height=15pt,anchor=north west,font=\tiny,inner sep=0pt] at ($(cell41.north west)+(0pt,-10pt)$){\cross};
%\node[minimum width=15pt, minimum height=15pt,anchor=north west,font=\tiny,inner sep=0pt] at ($(cell41.north west)+(0pt,-20pt)$){\cross};
\node[name=succ41,circle,fill,minimum size=2pt,inner sep=0pt,outer sep=0pt] at ($(cell41.east)+(-3pt,0pt)$) {};
%\draw ($(cell41.north west)+(0pt,-10pt)$) -- ($(cell41.north west)+(10pt,-10pt)$);
%\draw ($(cell41.north west)+(0pt,-20pt)$) -- ($(cell41.north west)+(10pt,-20pt)$);
%\draw ($(cell41.north west)+(10pt,0pt)$) -- ($(cell41.south west)+(10pt,0pt)$);
\node[anchor=south,font=\tiny,align=center] at ($(cell41.north)+(0pt,-2pt)$) {\tt T};
\draw[->] (succ31) -- (cell41);




\node[rounded corners,draw,name=cell311,minimum width=15pt, minimum height=15pt,anchor=west]
at ($(cell1.south)+(-7pt,-50pt)$){};
%\node[minimum width=15pt, minimum height=15pt,anchor=north west,font=\tiny,inner sep=0pt] at ($(cell311.north west)+(0pt,0pt)$){{$<$$z$}};
%\node[minimum width=15pt, minimum height=15pt,anchor=north west,font=\tiny,inner sep=0pt]at ($(cell311.north west)+(0pt,-10pt)$){\cross};
%\node[minimum width=15pt, minimum height=15pt,anchor=north west,font=\tiny,inner sep=0pt]at ($(cell311.north west)+(0pt,-20pt)$){\cross};
\node[name=succ311,circle,fill,minimum size=2pt,inner sep=0pt,outer sep=0pt] at ($(cell311.east)+(-3pt,0pt)$) {};
%\draw ($(cell311.north west)+(0pt,-10pt)$) -- ($(cell311.north west)+(10pt,-10pt)$);
%\draw ($(cell311.north west)+(0pt,-20pt)$) -- ($(cell311.north west)+(10pt,-20pt)$);
%\draw ($(cell311.north west)+(10pt,0pt)$) -- ($(cell311.south west)+(10pt,0pt)$);
\node[anchor=south,font=\tiny,align=center] at ($(cell311.north)+(0pt,-2pt)$) {${\tt c_1}$};
%\draw[->] (succ2) -- (cell3);






\node[rounded corners,draw,name=cell411,minimum width=15pt, minimum height=15pt,anchor=west]
at ($(cell311.east)+(20pt,0pt)$){};
%\node[minimum width=15pt, minimum height=15pt,anchor=north west,font=\tiny,inner sep=0pt] at ($(cell411.north west)+(0pt,0pt)$){{$<$$z$}};
%\node[minimum width=15pt, minimum height=15pt,anchor=north west,font=\tiny,inner sep=0pt] at ($(cell411.north west)+(0pt,-10pt)$){\cross};
%\node[minimum width=15pt, minimum height=15pt,anchor=north west,font=\tiny,inner sep=0pt] at ($(cell411.north west)+(0pt,-20pt)$){\cross};
\node[name=succ411,circle,fill,minimum size=2pt,inner sep=0pt,outer sep=0pt] at ($(cell411.east)+(-3pt,0pt)$) {};
%\draw ($(cell411.north west)+(0pt,-10pt)$) -- ($(cell411.north west)+(10pt,-10pt)$);
%\draw ($(cell411.north west)+(0pt,-20pt)$) -- ($(cell411.north west)+(10pt,-20pt)$);
%\draw ($(cell411.north west)+(10pt,0pt)$) -- ($(cell411.south west)+(10pt,0pt)$);
%\node[anchor=south,font=\tiny,align=center] at ($(cell411.north)+(0pt,-2pt)$) {};
\draw[->] (succ311) -- (cell411);






\node[rounded corners,draw,name=cell3110,minimum width=15pt, minimum height=15pt,anchor=west]
at ($(cell311.west)+(-70pt,0pt)$){};
%\node[minimum width=15pt, minimum height=15pt,anchor=north west,font=\tiny,inner sep=0pt] at ($(cell3110.north west)+(0pt,0pt)$){{$<$$z$}};
%\node[minimum width=15pt, minimum height=15pt,anchor=north west,font=\tiny,inner sep=0pt]at ($(cell3110.north west)+(0pt,-10pt)$){\cross};
%\node[minimum width=15pt, minimum height=15pt,anchor=north west,font=\tiny,inner sep=0pt]at ($(cell3110.north west)+(0pt,-20pt)$){\cross};
\node[name=succ3110,circle,fill,minimum size=2pt,inner sep=0pt,outer sep=0pt] at ($(cell3110.east)+(-3pt,0pt)$) {};
%\draw ($(cell3110.north west)+(0pt,-10pt)$) -- ($(cell3110.north west)+(10pt,-10pt)$);
%\draw ($(cell3110.north west)+(0pt,-20pt)$) -- ($(cell3110.north west)+(10pt,-20pt)$);
%\draw ($(cell3110.north west)+(10pt,0pt)$) -- ($(cell3110.south west)+(10pt,0pt)$);
\node[anchor=south,font=\tiny,align=center] at ($(cell3110.north)+(0pt,-2pt)$) {${\tt c_2}$};
%\draw[->] (succ2) -- (cell3);






\node[rounded corners,draw,name=cell4110,minimum width=15pt, minimum height=15pt,anchor=west]
at ($(cell3110.east)+(20pt,0pt)$){};
%\node[minimum width=15pt, minimum height=15pt,anchor=north west,font=\tiny,inner sep=0pt] at ($(cell4110.north west)+(0pt,0pt)$){{$<$$z$}};
%\node[minimum width=15pt, minimum height=15pt,anchor=north west,font=\tiny,inner sep=0pt] at ($(cell4110.north west)+(0pt,-10pt)$){\cross};
%\node[minimum width=15pt, minimum height=15pt,anchor=north west,font=\tiny,inner sep=0pt] at ($(cell4110.north west)+(0pt,-20pt)$){\cross};
\node[name=succ411,circle,fill,minimum size=2pt,inner sep=0pt,outer sep=0pt] at ($(cell4110.east)+(-3pt,0pt)$) {};
%\draw ($(cell4110.north west)+(0pt,-10pt)$) -- ($(cell4110.north west)+(10pt,-10pt)$);
%\draw ($(cell4110.north west)+(0pt,-20pt)$) -- ($(cell4110.north west)+(10pt,-20pt)$);
%\draw ($(cell4110.north west)+(10pt,0pt)$) -- ($(cell4110.south west)+(10pt,0pt)$);
%\node[anchor=south,font=\tiny,align=center] at ($(cell4110.north)+(0pt,-2pt)$) {};
\draw[->] (succ3110) -- (cell4110);






\node[rounded corners,draw,name=cell3111,minimum width=15pt, minimum height=15pt,anchor=west]
at ($(cell411.east)+(20pt,0pt)$){};
%\node[minimum width=15pt, minimum height=15pt,anchor=north west,font=\tiny,inner sep=0pt] at ($(cell3111.north west)+(0pt,0pt)$){{$<$$z$}};
%\node[minimum width=15pt, minimum height=15pt,anchor=north west,font=\tiny,inner sep=0pt]at ($(cell3111.north west)+(0pt,-10pt)$){\cross};
%\node[minimum width=15pt, minimum height=15pt,anchor=north west,font=\tiny,inner sep=0pt]at ($(cell3111.north west)+(0pt,-20pt)$){\cross};
\node[name=succ3111,circle,fill,minimum size=2pt,inner sep=0pt,outer sep=0pt] at ($(cell3111.east)+(-3pt,0pt)$) {};
%\draw ($(cell3111.north west)+(0pt,-10pt)$) -- ($(cell3111.north west)+(10pt,-10pt)$);
%\draw ($(cell3111.north west)+(0pt,-20pt)$) -- ($(cell3111.north west)+(10pt,-20pt)$);
%\draw ($(cell3111.north west)+(10pt,0pt)$) -- ($(cell3111.south west)+(10pt,0pt)$);

%\draw[->] (succ2) -- (cell3);


\node[rounded corners,draw,name=cell4111,minimum width=15pt, minimum height=15pt,anchor=west]
at ($(cell3111.east)+(20pt,0pt)$){};
%\node[minimum width=15pt, minimum height=15pt,anchor=north west,font=\tiny,inner sep=0pt] at ($(cell4111.north west)+(0pt,0pt)$){{$<$$z$}};
%\node[minimum width=15pt, minimum height=15pt,anchor=north west,font=\tiny,inner sep=0pt] at ($(cell4111.north west)+(0pt,-10pt)$){\cross};
%\node[minimum width=15pt, minimum height=15pt,anchor=north west,font=\tiny,inner sep=0pt] at ($(cell4111.north west)+(0pt,-20pt)$){\cross};
\node[name=succ4111,circle,fill,minimum size=2pt,inner sep=0pt,outer sep=0pt] at ($(cell4111.east)+(-3pt,0pt)$) {};
%\draw ($(cell4111.north west)+(0pt,-10pt)$) -- ($(cell4111.north west)+(10pt,-10pt)$);
%\draw ($(cell4111.north west)+(0pt,-20pt)$) -- ($(cell4111.north west)+(10pt,-20pt)$);
%\draw ($(cell4111.north west)+(10pt,0pt)$) -- ($(cell4111.south west)+(10pt,0pt)$);
%\node[anchor=south,font=\tiny,align=center] at ($(cell4111.north)+(0pt,-2pt)$) {};
\draw[->] (succ3111) -- (cell4111);



\node[rounded corners,draw,name=cell31111,minimum width=15pt, minimum height=15pt,anchor=west]
at ($(cell4111.east)+(20pt,0pt)$){};
%\node[minimum width=15pt, minimum height=15pt,anchor=north west,font=\tiny,inner sep=0pt] at ($(cell31111.north west)+(0pt,0pt)$){{$<$$z$}};
%\node[minimum width=15pt, minimum height=15pt,anchor=north west,font=\tiny,inner sep=0pt]at ($(cell31111.north west)+(0pt,-10pt)$){\cross};
%\node[minimum width=15pt, minimum height=15pt,anchor=north west,font=\tiny,inner sep=0pt]at ($(cell31111.north west)+(0pt,-20pt)$){\cross};
\node[name=succ31111,circle,fill,minimum size=2pt,inner sep=0pt,outer sep=0pt] at ($(cell31111.east)+(-3pt,0pt)$) {};
%\draw ($(cell31111.north west)+(0pt,-10pt)$) -- ($(cell31111.north west)+(10pt,-10pt)$);
%\draw ($(cell31111.north west)+(0pt,-20pt)$) -- ($(cell31111.north west)+(10pt,-20pt)$);
%\draw ($(cell31111.north west)+(10pt,0pt)$) -- ($(cell31111.south west)+(10pt,0pt)$);

%\draw[->] (succ2) -- (cell3);


\node[rounded corners,draw,name=cell41111,minimum width=15pt, minimum height=15pt,anchor=west]
at ($(cell31111.east)+(20pt,0pt)$){};
%\node[minimum width=15pt, minimum height=15pt,anchor=north west,font=\tiny,inner sep=0pt] at ($(cell41111.north west)+(0pt,0pt)$){{$=$$z$}};
%\node[minimum width=15pt, minimum height=15pt,anchor=north west,font=\tiny,inner sep=0pt] at ($(cell41111.north west)+(0pt,-10pt)$){\cross};
%\node[minimum width=15pt, minimum height=15pt,anchor=north west,font=\tiny,inner sep=0pt] at ($(cell41111.north west)+(0pt,-20pt)$){\cross};
\node[name=succ4111,circle,fill,minimum size=2pt,inner sep=0pt,outer sep=0pt] at ($(cell41111.east)+(-3pt,0pt)$) {};
%\draw ($(cell41111.north west)+(0pt,-10pt)$) -- ($(cell41111.north west)+(10pt,-10pt)$);
%\draw ($(cell41111.north west)+(0pt,-20pt)$) -- ($(cell41111.north west)+(10pt,-20pt)$);
%\draw ($(cell41111.north west)+(10pt,0pt)$) -- ($(cell41111.south west)+(10pt,0pt)$);
%\node[anchor=south,font=\tiny,align=center] at ($(cell41111.north)+(0pt,-2pt)$) {};
\draw[->] (succ31111) -- (cell41111);

%\node[rounded corners,draw,name=cell41,minimum width=15pt, minimum height=15pt,anchor=west]
%at ($(cell4.south)+(-30pt,-30pt)$){};
%\node[minimum width=15pt, minimum height=15pt,anchor=north west,font=\tiny,inner sep=0pt] at ($(cell41.north west)+(0pt,0pt)$){{$>$$z$}};
%\node[minimum width=15pt, minimum height=15pt,anchor=north west,font=\tiny,inner sep=0pt] at ($(cell41.north west)+(0pt,-10pt)$){\cross};
%\node[minimum width=15pt, minimum height=15pt,anchor=north west,font=\tiny,inner sep=0pt] at ($(cell41.north west)+(0pt,-20pt)$){\cross};
%\node[name=succ41,circle,fill,minimum size=3pt,inner sep=0pt,outer sep=0pt] at ($(cell41.east)+(-3pt,0pt)$) {};
%\draw ($(cell41.north west)+(0pt,-10pt)$) -- ($(cell41.north west)+(10pt,-10pt)$);
%\draw ($(cell41.north west)+(0pt,-20pt)$) -- ($(cell41.north west)+(10pt,-20pt)$);
%\draw ($(cell41.north west)+(10pt,0pt)$) -- ($(cell41.south west)+(10pt,0pt)$);
%\node[anchor=south,font=\tiny,align=center] at ($(cell41.north)+(0pt,-2pt)$) {};
%\draw[->] (succ41) -- (cell4);








%%%%%%%%%%%%%%%%%%%%%%%%%%%%%%%%%%%%%






%\node[minimum width=15pt, minimum height=15pt,anchor=north west,font=\tiny,inner sep=0pt] at ($(cell2.north west)+(15pt,10pt)$){{$<$}};






\node[minimum width=15pt, minimum height=15pt,anchor=north west,font=\tiny,inner sep=0pt] at ($(cell1.north)+(-8pt,15pt)$){{$\textcolor{black}{\tt \mapsto T}$}};

\node[minimum width=15pt, minimum height=15pt,anchor=north west,font=\tiny,inner sep=0pt] at ($(cell4.north)+(-12pt,15pt)$){{$\textcolor{black}{\tt H,z \mapsto T}$}};


\node[minimum width=15pt, minimum height=15pt,anchor=north west,font=\tiny,inner sep=0pt] at ($(cell1.south)+(-5pt,0pt)$){{$\tt i$}};
\node[minimum width=15pt, minimum height=15pt,anchor=north west,font=\tiny,inner sep=0pt] at ($(cell2.south)+(-5pt,0pt)$){{$\tt o$}};

\node[minimum width=15pt, minimum height=15pt,anchor=north west,font=\tiny,inner sep=0pt] at ($(cell31.south)+(-5pt,0pt)$){{$\tt i$}};
\node[minimum width=15pt, minimum height=15pt,anchor=north west,font=\tiny,inner sep=0pt] at ($(cell41.south)+(-5pt,0pt)$){{$\tt o$}};

\node[minimum width=15pt, minimum height=15pt,anchor=north west,font=\tiny,inner sep=0pt] at ($(cell311.south)+(-5pt,0pt)$){{$\tt i$}};
\node[minimum width=15pt, minimum height=15pt,anchor=north west,font=\tiny,inner sep=0pt] at ($(cell411.south)+(-5pt,0pt)$){{$\tt o$}};

\node[minimum width=15pt, minimum height=15pt,anchor=north west,font=\tiny,inner sep=0pt] at ($(cell3110.south)+(-5pt,0pt)$){{$\tt i$}};
\node[minimum width=15pt, minimum height=15pt,anchor=north west,font=\tiny,inner sep=0pt] at ($(cell4110.south)+(-5pt,0pt)$){{$\tt o$}};

\node[minimum width=15pt, minimum height=15pt,anchor=north west,font=\tiny,inner sep=0pt] at ($(cell3111.south)+(-5pt,0pt)$){{$\tt i$}};
\node[minimum width=15pt, minimum height=15pt,anchor=north west,font=\tiny,inner sep=0pt] at ($(cell4111.south)+(-5pt,0pt)$){{$\tt o$}};


\node[minimum width=15pt, minimum height=15pt,anchor=north west,font=\tiny,inner sep=0pt] at ($(cell31111.south)+(-5pt,0pt)$){{$\tt o$}};
\node[minimum width=15pt, minimum height=15pt,anchor=north west,font=\tiny,inner sep=0pt] at ($(cell41111.south)+(-5pt,0pt)$){{$\tt o$}};

\node[minimum width=15pt, minimum height=15pt,anchor=north west,font=\tiny,inner sep=0pt] at ($(cell1.west)+(20pt,10pt)$){{$<$}};

%\node[minimum width=15pt, minimum height=15pt,anchor=north west,font=\tiny,inner sep=0pt] at ($(cell41.west)+(25pt,20pt)$){{$<$}};

\node[minimum width=15pt, minimum height=15pt,anchor=north west,font=\tiny,inner sep=0pt] at ($(cell3.west)+(20pt,10pt)$){{$<$}};


\node[minimum width=15pt, minimum height=15pt,anchor=north west,font=\tiny,inner sep=0pt] at ($(cell2.north)+(-11pt,15pt)$){{$\textcolor{black}{\tt H \mapsto T}$}};

\node[minimum width=15pt, minimum height=15pt,anchor=north west,font=\tiny,inner sep=0pt] at ($(cell3.north)+(-11pt,15pt)$){{$\textcolor{black}{\tt H \mapsto T}$}};


\node[minimum width=15pt, minimum height=15pt,anchor=north west,font=\tiny,inner sep=0pt] at ($(cell31.north)+(-12pt,15pt)$){{$\textcolor{black}{\tt H,z \mapsto T}$}};

\node[minimum width=15pt, minimum height=15pt,anchor=north west,font=\tiny,inner sep=0pt] at ($(cell41.north)+(-8pt,15pt)$){{$\textcolor{black}{\tt H  \mapsto}$}};


%\node[minimum width=15pt, minimum height=15pt,anchor=north west,font=\tiny,inner sep=0pt] at ($(cell311.north)+(-8pt,15pt)$){{$\tt l$}};

%\node[minimum width=15pt, minimum height=15pt,anchor=north west,font=\tiny,inner sep=0pt] at ($(cell411.north)+(-8pt,15pt)$){{$\tt l$}};



%\node[minimum width=15pt, minimum height=15pt,anchor=north west,font=\tiny,inner sep=0pt] at ($(cell3110.north)+(-8pt,15pt)$){{$\tt l2$}};

\node[minimum width=15pt, minimum height=15pt,anchor=north west,font=\tiny,inner sep=0pt] at ($(cell3110.north)+(-8pt,13pt)$){{$\textcolor{black}{\mapsto \tt  [z,T]}$}};

%\node[minimum width=15pt, minimum height=15pt,anchor=north west,font=\tiny,inner sep=0pt] at ($(cell4110.north)+(-8pt,15pt)$){{$\tt l$}};
\node[minimum width=15pt, minimum height=15pt,anchor=north west,font=\tiny,inner sep=0pt] at ($(cell4110.north)+(-8pt,13pt)$){{$\textcolor{black}{\mapsto \tt  [z,T]}$}};
%\node[minimum width=15pt, minimum height=15pt,anchor=north west,font=\tiny,inner sep=0pt] at ($(cell3111.north)+(-8pt,15pt)$){{$\tt l$}};

\node[minimum width=15pt, minimum height=15pt,anchor=north west,font=\tiny,inner sep=0pt] at ($(cell3111.north)+(-8pt,13pt)$){{$\textcolor{black}{\mapsto \tt [z,T]}$}};
\node[minimum width=15pt, minimum height=15pt,anchor=north west,font=\tiny,inner sep=0pt] at ($(cell31111.north)+(-8pt,13pt)$){{$\textcolor{black}{\mapsto \tt  [z,T]}$}};


\node[minimum width=15pt, minimum height=15pt,anchor=north west,font=\tiny,inner sep=0pt] at ($(cell4111.north)+(-8pt,13pt)$){{$\textcolor{black}{\mapsto \tt [z,T]}$}};
\node[minimum width=15pt, minimum height=15pt,anchor=north west,font=\tiny,inner sep=0pt] at ($(cell311.north)+(-8pt,13pt)$){{$\textcolor{black}{\mapsto \tt [z,T]}$}};
\node[minimum width=15pt, minimum height=15pt,anchor=north west,font=\tiny,inner sep=0pt] at ($(cell411.north)+(-8pt,13pt)$){{$\textcolor{black}{\mapsto \tt [z,T]}$}};
%\node[minimum width=15pt, minimum height=15pt,anchor=north west,font=\tiny,inner sep=0pt] at ($(cell4111.north)+(-8pt,15pt)$){{$\tt l$}};




%\node[minimum width=15pt, minimum height=15pt,anchor=north west,font=\tiny,inner sep=0pt] at ($(cell31111.north)+(-8pt,15pt)$){{$\tt l$}};

\node[minimum width=15pt, minimum height=15pt,anchor=north west,font=\tiny,inner sep=0pt] at ($(cell41111.north)+(-8pt,13pt)$){{$\textcolor{black}{\tt H \mapsto T}$}};

%\node[minimum width=15pt, minimum height=15pt,anchor=north west,font=\tiny,inner sep=0pt] at ($(cell41.north)+(-8pt,15pt)$){{$\tt local$}};

\node[minimum width=15pt, minimum height=15pt,anchor=north west,font=\tiny,inner sep=0pt] at ($(cell3.south)+(-5pt,0pt)$){{$\tt i$}};
%\node[minimum width=15pt, minimum height=15pt,anchor=north west,font=\tiny,inner sep=0pt] at ($(cell41.south)+(-5pt,0pt)$){{$\tt i$}};
\node[minimum width=15pt, minimum height=15pt,anchor=north west,font=\tiny,inner sep=0pt] at ($(cell4.south)+(-5pt,0pt)$){{$\tt o$}};
%\node[minimum width=15pt, minimum height=15pt,anchor=north west,font=\tiny,inner sep=0pt] at ($(cell2.south)+(-5pt,-7pt)$){{$\textcolor{red}{\tt cut}$}};

%\node[minimum width=15pt, minimum height=15pt,anchor=north west,font=\tiny,inner sep=0pt] at ($(cell41111.south)+(-5pt,-7pt)$){{$\textcolor{red}{\tt cut}$}};

%\node[minimum width=15pt, minimum height=15pt,anchor=north west,font=\tiny,inner sep=0pt] at ($(cell411.south)+(-5pt,-7pt)$){{$\textcolor{red}{\tt cut}$}};

%\node[minimum width=15pt, minimum height=15pt,anchor=north west,font=\tiny,inner sep=0pt] at ($(cell4110.south)+(-5pt,-7pt)$){{$\textcolor{red}{\tt cut}$}};

\node[name=cell333,minimum width=15pt, minimum height=15pt,anchor=west]
at ($(cell31111.south)+(-80pt,-30pt)$){(c)};

\end{tikzpicture}





\caption{Example of Fragment Abstraction}
\label{lazylist}
\end{figure}
\end{center}
\section{Fragment Abstraction: Singly Linked Lists}
\label{sec:annotations}
By Theorem~\ref{thm:soundness}, linearizability can be verified by
establishing a 
reachability property for the product $\system=\augprog\compose\monitor$.
This verification must handle the challenges of
an unbounded number of threads, an unbounded heap, and unbounded data domain.
We have therefore developed a symbolic representation, which
is a careful combination of
data abstraction, thread abstraction, and shape abstraction. 

Our {\em data abstraction} abstracts from concrete data values in
$\intgrs$ by representing only {\em relations}
(in the set $\set{\prec,\equiv,\succ}$) between values of {\tt Data} fields of 
heap cells, local {\tt Data} variables, and registers of the observer.
%% We similarly abstract from concrete values of timestamps by representing
%% only relations between values of timestamps in heap cells and in
%% local variables.
In order to cope with an unbounded number of threads,
we extend the {\em thread-modular approach}~\cite{BLMRS:cav08}, in which
the symbolic representation characterizes
the projection of the configuration onto a single thread $t$.
%% The postcondition computation must consider how this projection can
%% be changed by execution steps of the program.
%% Such a step can be performed
%% either by the thread $t$ (a local step) or by 
%% some other thread $t'$ (an interference step).
%% Local steps can be handled by adapting standard techniques
%% for computing postconditions.
The corresponding postcondition computation must then take into account that
the projection onto a particular thread can be changed either by step of that
thread, or by a step of another thread.
%% interference steps involve to first compute a new abstract configuration
%% over two thread identifiers
%% which represents the combined state of both threads $t$ and $t'$,
%% and only therafter
%% performing the step of $t'$; thereafter the resulting abstract post-state
%% is projected back onto $t$.
In order to cope with an unbounded heap, we have developed a novel symbolic
representation, called {\em fragment abstraction}.
The idea of this abstraction is
to represent a heap by a set of (heap) {\em fragments}, where each fragment is
just a pair of nodes that are connected by a pointer.
In this section, we describe how this is done
for singly linked lists, i.e., where each heap cell contains exactly one
pointer field, named {\tt next}. In the following sections XXX and YYY, we describe
how fragment abstraction is applied to skiplists and to arrays of singly linked
lists.

\todo[inline]{Maybe we include timestamps from the very start?}

%% In the actual verification, we compute a symbolic representation
%% of an invariant that is satisfied by all reachable program configurations (including observer states).
%% This invariant is obtained by an abstract-interpretation-based
%% fixpoint procedure, which starts
%% from a representation of the set of initial configurations, and
%% thereafter repeatedly performs
%% postcondition computations that extend the
%% symbolic representation by the effect of any execution step of the program,
%% until convergence.
%% This analysis needs to deal with the challenges of an unbounded domain of
%% data values, an unbounded number of concurrently executing threads, and an
%% unbounded heap. 

\subsection{Symbolic Representation by Fragment Abstraction for Singly-Linked Lists}
In this subsection, we describe our symbolic representation for programs that
operate on singly-linked lists.
So, consider a program with global variables $\glvarset$ and thread-local variables $\locvarset$. We assume that all global variables are pointer variables.

Define a {\em local symbolic configuration} to consist of a mapping from local
variables (including the program counter) to appropriate domains:
\begin{itemize}
\item the program counter is mapped to a value,
\item each local variable with finite domain is mapped to that domain,
\item each variable that assumes values in $\intgrs$ is mapped to
  a mapping in  $[\ovarset \mapsto{\set{=,\neq}}]$, representing for
  each observer register $x  \in \ovarset$, whether or not the value
  of that equals the value of register $x$.
\item \bjcom{Quy: Do we need constraints for data and timer variables?}
\end{itemize}
\todo[inline]{for Quy: correct the definition of local symbolic conf., so
  that it is exactly what you use}
We use $\conf \satlocsymbconf{\thread} \locsymbconf$ to denote that in the
global configuration $\conf$, the
local configuration of thread $\thread$ satisfies the local symbolic
configuration $\locsymbconf$, defined in the natural way.

We now turn to describing our fragment abstraction for singly-linked lists.
First, define a {\em tag} as a tuple
$\atag = \tagtuple$, where
\begin{itemize}
\item
  $\pvars$ is a set of (global or local) pointer variables,
\item
  $\vals$ is a mapping from non-pointer fields of nodes, which
  \begin{enumerate}[(i)]
  \item
  maps each non-lock $\mathbb{F}$-field $\field$ to its domain, 
  \item
    maps each  lock field
    to $\set{\thread,\mathit{other},\mathit{free}}$,
%% representing whether the lock was last acquired by $\thread$, by another thread, or is free,
  and
\item
  maps each $\intgrs$-field $\field$ of $\cell$ to a mapping in
  $[\ovarset \mapsto{\set{=,\neq}}]$ from observer registers to
  $\set{=,\neq}$.
  \end{enumerate}
 \item
   $\reachfrom$ and $\reachto$ are sets of global pointer variables and observer registers, and
  \item $\private$ is a boolean value.
\end{itemize}

\begin{definition}[SLL-fragment]
\label{def:sll-fragment}
An {\em SLL-fragment} (or just fragment)
is a triple of of form
$\sllviewtuple$, of form
$\tuple{\inputtag,\nullconst}$, or of form
$\tuple{\inputtag,\dangconst}$,
where $\inputtag$ and $\outputtag$ are tags,
    and where $\datarel$ is a subset of $\set{<, =, >}$.
\end{definition}

Let us now define the meaning of SLL-fragments. We begin with tags:
We say that a heap cell $\cell$ be {\em accessible} from a thread $\thread$ if
it is reachable (directly or via sequence of
next-pointers) from a global variable or local variable of $\thread$.
Let $\cell$ be a cell accessible to thread $\thread$. For a tag
$\atag = \tagtuple$, we write $\cell \satfrag{\thread} \atag$ if
\begin{itemize}
\item
  $\pvars$ is the set of global pointer variables and
  local pointer variables of $\thread$ that point to $\cell$,
\item
  $\vals$ is satisfied in the following sense:
  \begin{enumerate}[(i)]
    \item
      $\vals(\field) = \cellval(\cell,\field)$ for each each non-lock
      $\mathbb{F}$-field $\field$,
  \item
      for each lock field $\field$, we have
      $\vals(\field) = \mathit{me}$ if $\thread$ holds the lock,
      $\vals(\field) = \mathit{other}$ if some other thread holds the lock, and
      $\vals(\field) = \mathit{free}$ if the lock is free,
  and
\item
  for each $\intgrs$-field $\field$ and each
  observer register $x  \in \ovarset$ we have that
  $\vals(\field)(x)$ is $=$ iff $\cellval(\cell,\field)$ is equal to the value
  of $x$.
\bjcom{Should we also do that for local data and timestamp variables?}
  \end{enumerate}
\item
  $\reachfrom$ is the set of
  \begin{enumerate}[(i)]
  \item global pointer variables
    %% and local pointer variables of $\thread$
    from which $\cell$ is reachable via a (possibly empty)
    sequence of {\tt next} pointers, and
\item observer registers whose value is equal to the value of the $\intgrs$-field of some cell which is accessible to $\thread$ and
  from which $\cell$ is reachable.
  \end{enumerate}
\bjinsert{Maybe give the intuition behind the second item}
\item
  $\reachto$ is the set of
  \begin{enumerate}[(i)]
  \item global pointer variables
    %% and local pointer variables of $\thread$
    that point to a cell which is reachable via a (possibly empty)
  sequence of {\tt next} pointers from $\cell$, and
  \item observer registers whose value is equal to the value of the $\intgrs$-field of some cell which is reachable from $\cell$.
  \end{enumerate}
%% \item
%%   $\privateofth{\cell}{\thread}$ is a boolean value, which is $\true$ if
%%   the cell $\cell$ can be seen only by thread $\thread$.
\end{itemize}

%% \begin{itemize}
%% \item
%%   $\pvarsofth{\cell}{\thread} = \pvars$,
%% \item
%%   $\valsofth{\cell}{\thread} = \vals$,
%% \item
%%   $\reachfromofth{\cell}{\thread} = \reachfrom$,
%% \item
%%   $\reachtoofth{\cell}{\thread} = \reachto$,
%%   and
%% \item
%%   if $\private$ is true then $\cell$ is not accessible from any thread other
%%   than $\thread$.
%% \end{itemize}
%% We can now define the notion of fragment.

For a fragment $\sllviewtuple$,
let $\cell \lhd \sllviewtuple$ denote that
$\mathtt{next}(\cell) \not\in \set{\nullconst,\dangconst}$, that
$\cell \lhd \inputtag$, that $\mathtt{next}(\cell) \lhd \outputtag$, 
and that
$\cell.\vals \sim \mathtt{next}(\cell).\vals$ for some
$\sim \in \datarel$.
For a fragment $\tuple{\inputtag,\nullconst}$, let
$\cell \lhd \tuple{\inputtag,\nullconst}$ denote that
$\cell \lhd \inputtag$ and $\mathtt{next}(\cell) = \nullconst$.
Define $\cell \lhd \tuple{\inputtag,\dangconst}$ analogously.

%% A {\em symbolic configuration} is a pair
%% $\tuple{\locsymbconf,\fragset}$ consisting of a local symbolic 
%% configuration $\locsymbconf$, and a set $\fragset$ of fragments.
%% A configuration $\conf$ of a program satisfies a symbolic configuration
%% $\tuple{\locsymbconf,\fragset}$ if for each thread $\thread$,
%% \begin{itemize}
%% \item the local configuration of $\thread$ satisfies $\locsymbconf$, in the
%%   natural way, and
%% \item for each $\cell$ that is accessible from $\thread$, there is
%%   a fragment $\frag$ such that $\cell \lhd_{\thread} \frag$.
%% \end{itemize}
A {\em symbolic representation} $\symbrep$ is a mapping from
local symbolic  configurations to sets of fragments.
A configuration $\conf$ of a program satisfies a symbolic representation
if for each thread $\thread$, 
\begin{itemize}
\item the domain of $\symbrep$ contains a local symbolic configuration
  $\locsymbconf$ such that $\conf \satlocsymbconf{\thread} \locsymbconf$, and
\item whenever $\conf \satlocsymbconf{\thread} \locsymbconf$, then
for each $\cell$ that is accessible from $\thread$, there is
  a fragment $\frag$ in $\symbrep(\locsymbconf)$ such that
  $\cell \lhd_{\thread} \frag$.
\end{itemize}

\paragraph{Example:} Let us show an example of how a singly linked list (SLL) is split into a set of fragments. Fig.~\ref{lazylist}(a) shows an example of a concrete shape of a singly linked list. Each cell contains the values of $\tt val$, $\tt mark$, and $\tt lock$ from top to bottom, where \tick \; denotes $\tt true$, and \cross \; denotes $\tt false$ (or $\tt free$ for $\tt lock$) and the value of $\tt val$ is denoted by a pair of the observer register $\tt z$ and a subset of $\set{<, =, >}$. There are two threads $1$ and $2$ with two local variables $\tt c_1$ and $\tt c_2$. There are two global variables $\tt H$ and $\tt T$ pointing to the head and tail of the list. The observer register $\tt z$ has value 8. Fig.~\ref{lazylist}(b) shows the result of splitting the list in Fig.~\ref{lazylist}(a) into fragments where each heap fragment is a small list of two nodes. Fig.~\ref{lazylist}(c) shows the abstraction of fragments in Fig.~\ref{lazylist}(b) where for each cell $\cell$, the value of $val$ is abstracted to a subset in $\tt \{<z, =z, >z\}$, the reachability relation between $\cell$ and global variables and observer registers is abstracted to the predicate $\tt X \mapsto \tt Y$ where $\tt X$, $\tt Y$ are sets of global variables and observer registers, $\tt i$ or $\tt o$ states that $\cell$ is an input or output cell.

\subsection{Computing Postconditions}
In the  verification, we must compute a symbolic representation
that is satisfied by all reachable program configurations.
%% (including observer states).
This invariant is obtained by an abstract-interpretation-based
fixpoint procedure, which starts
from a representation of the set of initial configurations, and
thereafter repeatedly performs
postcondition computations that extend the
symbolic representation by the effect of any execution step of the program,
until convergence.
In this subsection, we describe the symbolic postcondition computation, which
is the key step in this procedure.
%% \bjcom{The following should be included more smoothly, if really needed}
%% Let $\tt gtag(v,i)$ be the tuple $\tt \tuple{\tt gvars(i), gdata(i),{\tt greachfrom}(\tt i), {\tt reachto}(\tt i), private(i )}$ where $\tt gvars(i)$ is the set of global variables in $\tt vars(i)$ and $\tt greachfrom(i)$ is the set of global variables in $\tt reachfrom(i)$.

%% Note that in each fragment $\tt v$, for two indices $\tt i_1$, $\tt i_2$ where $\tt next(i_1) = i_2$, we always have $\tt {\tt reachfrom}(\tt i_1) \subseteq \tt {\tt reachfrom}(\tt i_2)$ and $\tt {\tt reachto}(\tt i_1) = \tt {\tt reachto}(\tt i_2) \cup X$ where $\tt X$ is the set of global variables in ${\tt vars}(\tt i_1)$ and observer registers $\tt z$ which are equal to $\tt i_1.d$ for $\intgrs$-field $\tt d$. 
  
%% $\tt \tuple{\tt vars(i), data(i),{\tt reachfrom}(\tt i), {\tt reachto}(\tt i), private(i )}$. Let $\tt gtag(v,i)$ be the tuple $\tt \tuple{\tt gvars(i), gdata(i),{\tt greachfrom}(\tt i), {\tt reachto}(\tt i), private(i )}$ where $\tt gvars(i)$ is the set of global variables in $\tt vars(i)$ and $\tt greachfrom(i)$ is the set of global variables in $\tt reachfrom(i)$. 

%% We first introduce the abstraction at the heart of the method. The abstraction takes its input as a set of pre-defined fragments, and any heap is ``broken down into pre-defined fragments''. A fragment retains the information about a particular part of a heap. This choice of fragments defines the level of abstraction that transforms heap into fragments. The abstract function $\alpha_k$ maps a heap into a set of fragments. Whereas, the concretization function $\gamma_k$ inputs a set of fragments, and returns the set of heaps that can be reconstructed from these fragments. 

%% \paragraph{Fragment Definition:} A heap fragment of size $\tt k$ is represented as \begin{center}
%% 	${\tt v} = \viewconf$ where 
%% \end{center} 

%We first introduce the abstraction at the heart of the method. The abstraction takes its input as a set of pre-defined fragments parameterized by a constant $\tt k$, and any heap is “broken down into pieces” of size (at most) k, called fragments. A fragment retains the information about $\tt k$ connected cells from a heap and abstracts away the other cells. This choice defines the level of abstraction that transforms heap into fragments.There is a certain freedom in what information to retain and what to abstract away. The abstract function $\alpha_k$ maps a heap into a set of heap fragments of sizes $\tt k$. The concretization function $\gamma_k$ inputs a set of heap fragments, and returns the set of heaps that can be reconstructed from the these fragments. Let us represent symbolic representations of heap fragments. A heap fragment of size $\tt k$ is represented as ${\tt \Omega} = \viewconf$ where 

%\paragraph{Abstract Configuration}:
%A abstract {\it configuration} of a program $\prog$ in the fragment of a single thread $\thread$
% is a tuple
%$\tuple{\tt c,\lambda, \Omega , S}$ where
%\begin{itemize}
%\item $\tt c$ is an observer configuration, 
%\item $\lambda$ is the label of the program statement, 
%\item $\tt \Omega$ is a tuple $\tt \tuple{\Omega_p, \Omega_l, Val}$ where $\tt \Omega_p$ contains a set ordering constrains between pointer variables, $\tt \Omega_l$ contains a set ordering constrains between data local variables and $\tt Val$ maps each data local variable to its value.
%\item $\tt S$ is a set of fragments.
%\end{itemize}
%
%% We use $\confsetof\prog$ to denote the set of configurations

In this subsection, we describe how to perform the abstract postcondition computation on our symbolic representation for singly linked lists. The
abstract postcondition computation must ensure that the symbolic representation
of the reachable configurations of a program is closed under computation
steps \bjcom{Check terminology}.
More precisely, assume that a global configuration $\conf$ satisfies
a symbolic representation $\symbrep$. As is standard in the thread-modular
approach, we must consider two cases:
\begin{itemize}
\item
  {\it Local Steps:}
  A thread $\thread$, which
  satisfies a local symbolic configuration $\locsymbconf$
  in $\domof{\symbrep}$ performs a computation step,
  which may change the local configuration of $\thread$
  and the heap that is accessible to
    $\thread$. In this case, we must compute
  a local symbolic configuration $\locsymbconf'$
which is satisfied by $\thread$ after its
computation step; this can be done by standard techniques. We must also
must compute a set $\fragset'$ of views that characterize the heap accessible to
$\thread$ after the step, and
make sure that $\fragset' \subseteq \symbrep(\locsymbconf')$.
for each $\cell$ that is accessible from $\thread$, there is
  a fragment $\frag$ in $\symbrep(\locsymbconf)$ such that
  $\cell \lhd_{\thread} \frag$.
\item    {\it Interference Steps:}
A thread $\thread_1$, which
  satisfies a local symbolic configuration $\locsymbconf_1$
  in $\domof{\symbrep}$, while another thread $\thread_2$, which
  satisfies a local symbolic configuration $\locsymbconf_2$
  in $\domof{\symbrep}$   performs a computation step,
  which potentially affects the
  local configuration of $\thread_1$ and the heap accessible to $\thread_1$.
  For this case, we first combine the local symbolic configurations by an
%%   of $\thread_1$ and $\thread_2$ and then compute a representation of the
%%   the possible heaps whose projection onto $\thread_i$ satisfies
%%   $\symbrep(\locsymbconf_i)$ for $i=1,2$.
%% This is done by performing an
  operation, called {\em intersection}, which
  \begin{inparaenum}[(i)]
\item combines a local symbolic configuration of thread $\thread_1$, represented by
  $\locsymbconf_1$, and a
 local symbolic configuration of thread $\thread_2$, represented by
 $\locsymbconf_2$, and form a joint local symbolic configuration, and
%%  $\locsymbconf_{12}$, and
\item
  computes a representation of all joint heap
 projections such that their projection onto $\thread_i$ satisfies
 $\symbrep(\locsymbconf_i)$ for $i = 1,2$.
  \end{inparaenum}
We thereafter perform a local
 step of $\thread_2$ as in the previous case, and finally project back onto
 $\thread_1$ to obtain the final result.
\end{itemize}
Let us introduce notation for describing the different attributes of
fragments. For a fragment $\frag = \fragtupleof{\frag}$,
let the input tag be denoted $\tagtupleat{\inat{\frag}}$ and the output tag
be denoted $\tagtupleat{\outat{\frag}}$.

\paragraph{Local Steps}
We now describe how to compute a set $\fragset'$ of fragments resulting
from a local step of $\thread$. The computation has do be done differently for
each statement.
%% that may affect the heap of the program, including the
%% pointer variables that point to heap cells.
The main difficulty in the
postcondition computation is to update the reachability information provided
in fields $\reachfrom$ and $\reachto$ fields in each tag of a 
fragment. For instance, consider a statement $\gvarof{g} := \lvarof{v}$
which assigns the value of a local pointer variable $\lvarof{v}$ to the global
pointer variable $\gvarof{g}$.
The postcondition computation is performed essentially
by updating the fields of fragments. It is then difficult to know,
for an arbitrary fragment $\frag$ whether, say, $\reachfromat{\inat{\frag}}$
should include the variable $\gvarof{g}$ or not. If $\lvarof{v}$ would
have been a global variable, this information would have been available in
$\reachfromat{\inat{\frag}}$ before the computation step. However, since
$\reachfromat{\inat{\frag}}$ does not include local variables, we 
start the postcondition computation by computing a number of transitive-closure-like relations between fragments, and thereafter check whether the fragment
$\frag$ can be reached from a fragment $\frag'$ in which
$\lvarof{v}  \in \pvarsat{\inat{\frag'}}$.

\bjcom{Question to Quy: Do you keep accurate reachability
  information in the sense that you know {\em both}
  \begin{itemize}
  \item
    whether it is {\em possible} that $\frag' \overset{*}\hookrightarrow_S \frag$
    holds
  \item
    whether it is {\em possible} that $\frag' \overset{*}\hookrightarrow_S \frag$
    does not hold
\end{itemize}
  Recall that for a set of fragments, it is possible that {\em both} of these
  cases occur!!}

In general, updating
such fragments requires to know how cells that satisfy different fragments
can and cannot be reached from each other in a heap. Given a set $\fragset$ of
fragments, we will therefore define a number of reachability relations on
$\fragset$.

Let $\frag_1$ and $\frag_2$ be two fragments in a set $\fragset$ of fragments.
Let
\begin{itemize}
\item  $\frag_1 \hookrightarrow_S \frag_2$ denote that $\tagat{\outat{\frag_1}}$
  and $\tagat{\inat{\frag_2}}$ are consistent;
  \todo[inline]{Quy: PLEASE  you provide the precise definition of when
    $\tagat{\outat{\frag_1}}$ and $\tagat{\inat{\frag_2}}$ are consistent?}
  this means that if
  $\cell_1 \lhd \frag_1$ and  $\cell_2 \lhd \frag_2$ then it is possible that
  $\mathtt{next}(\cell_1) = \cell_2$.
\item  $\frag_1 \leftrightarrow_S \frag_2$ denote that $\tagat{\outat{\frag_1}} = \tagat{\outat{\frag_2}}$ and that 
  the global variables in  $\reachfromat{\inat{\frag_1}}$ are disjoint from
  those in $\reachfromat{\inat{\frag_2}}$.
  $\reachfromat{\inat{\frag_1}} \cap \reachfromat{\inat{\frag_2}} = \emptyset$
\todo[inline]{Question: shouldn't also include
  $\pvarsat{\inat{\frag_1}} \cap \pvarsat{\inat{\frag_2}} = \emptyset$?.
More generally, please provide the precise definition.}
  $\reachfromat{\inat{\frag_1}} \cap \reachfromat{\inat{\frag_2}} = \emptyset$
  \quycom{$\reachfromat{\inat{\frag_1}} \cap \reachfromat{\inat{\frag_2}} = \emptyset$ here is not correct, we only need the global variables of reachfrom and reachto in this case}
\quycom{In my definition, the consistent here is defined for two views in the same threads. With the view from other thread, I use the notion of intersection.}
  intuitively, this means that if
  $\cell_1 \lhd \frag_1$ and  $\cell_2 \lhd \frag_2$ then it is possible that
  $\cell_1$ and $\cell_2$ are different cells with
  $\mathtt{next}(\cell_1) = \mathtt{next}(\cell_2)$.
\end{itemize}
\bjcom{Here is the case with size 1: I did not yet have time to get here}
For two fragments $\frag_1$ and $\frag_2$ with sizes of 1, let $\tt i$ be the {\em input} index of $\frag_1$, let $\tt j_i$ be the {\em input} index of $\frag_2$.
 \begin{itemize}
 \item $\frag_1$ and $\frag_2$ is branch consistent, denoted as $\frag_1 \leftrightarrow_S \frag_2$ if $\frag_1.next(i) = \frag_2.next(i)$ and $\tt G_1 \cap G_2 = \emptyset$ where $\tt G_1$ is the set of global variables in $\frag_1.{\tt reachfrom}(\tt i)$,  $\tt G_2$ is the set of global variables in $\frag_2.{\tt reachfrom}(\tt j_i)$.
 \end{itemize}

We use the above relations to define several derived relations:
\bjcom{We could use a common pattern for these and save space in definitions.}
\begin{itemize}
\item Define $\overset{+}\hookrightarrow_S$ as the transitive closure of
  $\hookrightarrow_S$, and define
   $\overset{*}\hookrightarrow_S$ as the reflexive transitive closure of
$\hookrightarrow_S$.
\item Let $\frag_1  \overset{**}\leftrightarrow_S  \frag_2$ denote that there are
  $\frag_1',\frag_2' \in \fragset$ with $\frag_1' \leftrightarrow_S \frag_2'$ such that
  $\frag_1  \overset{*}\hookrightarrow_S \frag_1'$ and $\frag_2  \overset{*}\hookrightarrow_S \frag_2'$.
\item Let $\frag_1  \overset{*+}\leftrightarrow_S  \frag_2$ denote that there are
  $\frag_1', \frag_2' \in \fragset$ with $\frag_1' \leftrightarrow_S \frag_2'$ such that
  $\frag_1  \overset{*}\hookrightarrow_S \frag_1'$ and $\frag_2  \overset{+}\hookrightarrow_S \frag_2'$.
\item Let $\frag_1  \overset{*\circ}\leftrightarrow_S  \frag_2$ denote that is a
  $\frag_1' \in \fragset$ with $\frag_1' \leftrightarrow_S \frag_2$ such that
  $\frag_1  \overset{*}\hookrightarrow_S \frag_1'$.
\item Let $\frag_1  \overset{++}\leftrightarrow_S  \frag_2$ denote that there are
  $\frag_1',\frag_2' \in \fragset$ with $\frag_1' \leftrightarrow_S \frag_2'$ such that
  $\frag_1  \overset{+}\hookrightarrow_S \frag_1'$ and $\frag_2  \overset{+}\hookrightarrow_S \frag_2'$.
\item Let $\frag_1  \overset{+*}\leftrightarrow_S  \frag_2$ denote that there are
  $\frag_1', \frag_2' \in \fragset$ with $\frag_1' \leftrightarrow_S \frag_2'$ such that
  $\frag_1  \overset{+}\hookrightarrow_S \frag_1'$ and $\frag_2  \overset{*}\hookrightarrow_S \frag_2'$.
\item Let $\frag_1  \overset{+\circ}\leftrightarrow_S  \frag_2$ denote that is a
  $\frag_1' \in \fragset$ with $\frag_1' \leftrightarrow_S \frag_2$ such that
  $\frag_1  \overset{+}\hookrightarrow_S \frag_1'$.
\end{itemize}

\todo[inline]{Quy: I think you should make a picture with
  a shape with several views that illustrate the relations above}

\paragraph{Abstract Transformers for Local Steps} We first consider
the abstract transformer for the case that the thread performs a local
step.
%% Let $\tt kill(x,v)$ be the procedure to remove $\tt x$ if it exists from
%% $\reachfromat{\inat{\frag}}$,
%% $\reachfromat{\outat{\frag}}$,
%% $\reachtoat{\inat{\frag}}$,
%% $\reachtoat{\outat{\frag}}$,
%% $\pvarsat{\inat{\frag}}$, and
%% $\pvarsat{\outat{\frag}}$.
We consider each form of program statement separately.
For each program statement, we compute a set $\fragset_{post}$  of fragments that
must be included in the set of fragments after executing the statement.
We let $\fragset_{post}$ be initialized as the empty set.
\begin{description}
\item[${\tt x := y}$] where $\mathtt{x}$ and $\mathtt{y}$ are global or local
  (to thread $\thread$) pointer variables.
  The transformer is computed as follows: For each fragment $\frag_y \in S$ such that $\mathtt{y} \in \pvarsat{\inat{\frag_y}}$, 
create $\frag'$ which is the same as $\frag_y$, except that
\begin{itemize}
\item $\pvarsat{\inat{\frag'}} = \pvarsat{\inat{\frag_y}} \cup \set{x}$ and
  $\pvarsat{\outat{\frag'}} = \pvarsat{\outat{\frag}} \setminus \set{x}$
\end{itemize}
and furthermore, if $\gvarof{x}$ is a global variable, then
\begin{itemize}
\item $\reachtoat{\inat{\frag'}} = \reachtoat{\inat{\frag}} \cup \set{x}$ and
$\reachfromat{\inat{\frag'}} = \reachfromat{\inat{\frag}} \cup \set{x}$,
\item $\reachfromat{\outat{\frag'}} = \reachfromat{\outat{\frag}} \cup \set{x}$ and
$\reachtoat{\outat{\frag'}} = \reachtoat{\outat{\frag}} \setminus \set{x}$.
\end{itemize}
and add $\frag'$ to $\tt S_{post}$.
Thereafter, we must ensure that any cell $\cell$ which is accessible to $\thread$
satisfies some fragement in $\fragset_{post}$. The cell $\cell$ satisfies some
fragment in $\fragset$, and must be in some reachability relationship to
the cell pointed to by $\gvarof{y}$. We make a case analysis on the possible
relationships:
\begin{enumerate}
\item 
\item  for each fragment $\frag$ where $\frag \hookrightarrow_S \frag_y$, create $\frag'$ which is same as $\frag$ except that
\begin{itemize}
\item $\pvarsat{\inat{\frag'}} = \pvarsat{\inat{\frag}} \setminus \set{x}$,
\item $\pvarsat{\outat{\frag'}} = \pvarsat{\outat{\frag}} \cup \set{x}$,
\item $\reachfromat{\inat{\frag'}} = \reachfromat{\inat{\frag}} \setminus \set{x}$ if $\tt x$ is a global variable,
\item $\reachtoat{\inat{\frag'}} = \reachtoat{\inat{\frag}} \cup \set{x}$ if $\tt x$ is a global variable,
 \item $\reachfromat{\outat{\frag'}} = \reachfromat{\outat{\frag}} \cup \set{x}$ if $\tt x$ is a global variable,
\item $\reachtoat{\outat{\frag'}} = \reachtoat{\outat{\frag}} \cup \set{x}$ if $\tt x$ is a global variable,
\end{itemize}
and add $\frag'$ to $\tt S_{post}$,
\item for each fragment $\frag$ where $\frag \overset{+}\hookrightarrow_S \frag_y$, create $\frag'$ which is same as $\frag$ except that
\begin{itemize}
\item $\pvarsat{\inat{\frag'}} = \pvarsat{\inat{\frag}} \setminus \set{x}$,
\item $\pvarsat{\outat{\frag'}} = \pvarsat{\outat{\frag}} \setminus \set{x}$,
\item $\reachfromat{\inat{\frag'}} = \reachfromat{\inat{\frag}} \setminus \set{x}$ if $\tt x$ is a global variable,
\item $\reachtoat{\inat{\frag'}} = \reachtoat{\inat{\frag}} \cup \set{x}$ if $\tt x$ is a global variable,
\item $\reachfromat{\outat{\frag'}} = \reachfromat{\outat{\frag}} \setminus \set{x}$ if $\tt x$ is a global variable,
\item $\reachtoat{\outat{\frag'}} = \reachtoat{\outat{\frag}} \cup \set{x}$ if $\tt x$ is a global variable,
\end{itemize}
and add $\frag'$ to $\tt S_{post}$,
\item for each fragment $\frag$ where $\frag_y \overset{*}\hookrightarrow_S \frag$, create $\frag'$ which is same as $\frag$ except that
\begin{itemize}
\item $\pvarsat{\inat{\frag'}} = \pvarsat{\inat{\frag}} \setminus \set{x}$,
\item $\pvarsat{\outat{\frag'}} = \pvarsat{\outat{\frag}} \setminus \set{x}$,
\item $\reachfromat{\inat{\frag'}} = \reachfromat{\inat{\frag}} \cup \set{x}$ if $\tt x$ is a global variable,
\item $\reachtoat{\inat{\frag'}} = \reachtoat{\inat{\frag}} \setminus \set{x}$ if $\tt x$ is a global variable,

 \item $\reachfromat{\outat{\frag'}} = \reachfromat{\outat{\frag}} \cup \set{x}$ if $\tt x$ is a global variable,
 \item $\reachtoat{\outat{\frag'}} = \reachtoat{\outat{\frag}} \setminus \set{x}$ if $\tt x$ is a global variable,

\end{itemize}
and add $\frag'$ to $\tt S_{post}$,
\item for each fragment $\frag$ where $\frag_y \overset{*+}\leftrightarrow_S \frag$, create $\frag'$ which is same as $\frag$ except that
\begin{itemize}
\item $\pvarsat{\inat{\frag'}} = \pvarsat{\inat{\frag}} \setminus \set{x}$,
\item $\pvarsat{\outat{\frag'}} = \pvarsat{\outat{\frag}} \setminus \set{x}$,
\item $\reachfromat{\inat{\frag'}} = \reachfromat{\inat{\frag}} \setminus \set{x}$ if $\tt x$ is a global variable,
\item $\reachtoat{\inat{\frag'}} = \reachtoat{\inat{\frag}} \setminus \set{x}$ if $\tt x$ is a global variable,
\item $\reachfromat{\outat{\frag'}} = \reachfromat{\outat{\frag}} \setminus \set{x}$ if $\tt x$ is a global variable,
\item $\reachtoat{\outat{\frag'}} = \reachtoat{\outat{\frag}} \setminus \set{x}$ if $\tt x$ is a global variable,
\end{itemize}
and add $\frag'$ to $\tt S_{post}$,
\item for each fragment $\frag$ where $\frag_y \overset{*o}\leftrightarrow_S \frag$, create $\frag'$ which is same as $\frag$ except that
\begin{itemize}
\item $\pvarsat{\inat{\frag'}} = \pvarsat{\inat{\frag}} \setminus \set{x}$,
\item $\pvarsat{\outat{\frag'}} = \pvarsat{\outat{\frag}} \setminus \set{x}$,
\item $\reachfromat{\inat{\frag'}} = \reachfromat{\inat{\frag}} \setminus \set{x}$ if $\tt x$ is a global variable,
\item $\reachtoat{\inat{\frag'}} = \reachtoat{\inat{\frag}} \setminus \set{x}$ if $\tt x$ is a global variable,
\item $\reachfromat{\inat{\frag'}} = \reachfromat{\outat{\frag}} \cup \set{x}$ if $\tt x$ is a global variable,
\item $\reachtoat{\inat{\frag'}} = \reachtoat{\outat{\frag}} \setminus \set{x}$ if $\tt x$ is a global variable,
\end{itemize}
and add $\frag'$ to $\tt S_{post}$,
\end{enumerate}
\item[${\tt x := y.next}$]: The local abstract transformer is quite similar to the previous case with some slight differences. For each fragment $\frag_y \in S$ where $\gvarof{y} \in \pvarsat{\inat{\frag_y}}$, first
create $\frag'$ which is same as $\frag_y$ except that
\begin{itemize}
\item $\pvarsat{\inat{\frag'}} = \pvarsat{\inat{\frag}} \setminus \set{x}$,
\item $\pvarsat{\outat{\frag'}} = \pvarsat{\outat{\frag}} \cup \set{x}$,
\item $\reachfromat{\inat{\frag'}} = \reachfromat{\inat{\frag}} \setminus \set{x}$ if $\tt x$ is a global variable,
\item $\reachtoat{\inat{\frag'}} = \reachtoat{\inat{\frag}} \cup \set{x}$ if $\tt x$ is a global variable,
 \item $\reachfromat{\outat{\frag'}} = \reachfromat{\outat{\frag}} \cup \set{x}$ if $\tt x$ is a global variable,
 \item $\reachtoat{\outat{\frag'}} = \reachtoat{\outat{\frag}} \cup \set{x}$ if $\tt x$ is a global variable,
\end{itemize}
and add $\frag'$ to $\tt S_{post}$. Thereafter, we perform a case analysis on
the relationship between the view that $\cell$ satisfies and $\frag_y$:
\begin{enumerate}
\item  for each fragment $\frag$ where $\frag \overset{*}\hookrightarrow_S \frag_y$, create $\frag'$ which is same as $\frag$ except that
\begin{itemize}
\item $\pvarsat{\inat{\frag'}} = \pvarsat{\inat{\frag}} \setminus \set{x}$,
\item $\pvarsat{\outat{\frag'}} = \pvarsat{\outat{\frag}} \setminus \set{x}$,
\item $\reachfromat{\inat{\frag'}} = \reachfromat{\inat{\frag}} \setminus \set{x}$ if $\tt x$ is a global variable,

\item $\reachtoat{\inat{\frag'}} = \reachtoat{\inat{\frag}} \cup \set{x}$ if $\tt x$ is a global variable,
\item $\reachfromat{\outat{\frag'}} = \reachfromat{\outat{\frag}} \setminus \set{x}$ if $\tt x$ is a global variable,

 \item $\reachtoat{\outat{\frag'}} = \reachtoat{\outat{\frag}} \cup \set{x}$ if $\tt x$ is a global variable,
\end{itemize}
and add $\frag'$ to $\tt S_{post}$,
\item for each fragment $\frag$ where $\frag_y \hookrightarrow_S \frag$, create $\frag'$ which is same as $\frag$ then
\begin{itemize}
\item $\pvarsat{\inat{\frag'}} = \pvarsat{\inat{\frag}} \cup \set{x}$,
\item $\pvarsat{\outat{\frag'}} = \pvarsat{\outat{\frag}} \setminus \set{x}$,
\item $\reachfromat{\inat{\frag'}} = \reachfromat{\inat{\frag}} \cup \set{x}$ if $\tt x$ is a global variable,
 \item $\reachtoat{\outat{\frag'}} = \reachtoat{\outat{\frag}} \cup \set{x}$ if $\tt x$ is a global variable,
 \item $\reachfromat{\outat{\frag'}} = \reachfromat{\outat{\frag}} \cup \set{x}$ if $\tt x$ is a global variable,
 \item $\reachtoat{\outat{\frag'}} = \reachtoat{\outat{\frag}} \setminus \set{x}$ if $\tt x$ is a global variable,

\end{itemize}
and add $\frag'$ to $\tt S_{post}$,
\item for each fragment $\frag$ where $\frag_y \leftrightarrow_S \frag$, create $\frag'$ which is same as $\frag$ except that
\begin{itemize} 
\item $\pvarsat{\inat{\frag'}} = \pvarsat{\inat{\frag}} \setminus \set{x}$,
\item $\pvarsat{\outat{\frag'}} = \pvarsat{\outat{\frag}} \cup \set{x}$,
\item $\reachfromat{\inat{\frag'}} = \reachfromat{\inat{\frag}} \setminus \set{x}$ if $\tt x$ is a global variable,
\item $\reachtoat{\inat{\frag'}} = \reachtoat{\inat{\frag}} \cup \set{x}$ if $\tt x$ is a global variable,
 \item $\reachfromat{\outat{\frag'}} = \reachfromat{\outat{\frag}} \cup \set{x}$ if $\tt x$ is a global variable,
 \item $\reachtoat{\outat{\frag'}} = \reachtoat{\outat{\frag}} \cup \set{x}$ if $\tt x$ is a global variable,
\end{itemize}
and add $\frag'$ to $\tt S_{post}$,

\item for each fragment $\frag$ where $\frag_y \overset{+}\hookrightarrow_S \frag$, create $\frag'$ which is same as $\frag$ except that
\begin{itemize}
\item $\pvarsat{\inat{\frag'}} = \pvarsat{\inat{\frag}} \setminus \set{x}$,
\item $\pvarsat{\outat{\frag'}} = \pvarsat{\outat{\frag}} \setminus \set{x}$,
\item $\reachfromat{\inat{\frag'}} = \reachfromat{\inat{\frag}} \cup \set{x}$ if $\tt x$ is a global variable,
\item $\reachtoat{\inat{\frag'}} = \reachtoat{\inat{\frag}} \setminus \set{x}$ if $\tt x$ is a global variable,
 \item $\reachfromat{\outat{\frag'}} = \reachfromat{\outat{\frag}} \cup \set{x}$ if $\tt x$ is a global variable,
 \item $\reachtoat{\outat{\frag'}} = \reachtoat{\outat{\frag}} \setminus \set{x}$ if $\tt x$ is a global variable,

\end{itemize}
and add $\frag'$ to $\tt S_{post}$,
\item for each fragment $\frag$ where $\frag_y \overset{*o}\leftrightarrow_S \frag$, create $\frag'$ which is same as $\frag$ except that
\begin{itemize}
\item $\pvarsat{\inat{\frag'}} = \pvarsat{\inat{\frag}} \setminus \set{x}$,
\item $\pvarsat{\outat{\frag'}} = \pvarsat{\outat{\frag}} \setminus \set{x}$,
\item $\reachfromat{\inat{\frag'}} = \reachfromat{\inat{\frag}} \setminus \set{x}$ if $\tt x$ is a global variable,
\item $\reachtoat{\inat{\frag'}} = \reachtoat{\inat{\frag}} \setminus \set{x}$ if $\tt x$ is a global variable,
\item $\reachfromat{\outat{\frag'}} = \reachfromat{\outat{\frag}} \cup \set{x}$ if $\tt x$ is a global variable,
\item $\reachtoat{\outat{\frag'}} = \reachtoat{\outat{\frag}} \setminus \set{x}$ if $\tt x$ is a global variable,

\end{itemize}
and add $\frag'$ to $\tt S_{post}$,
\item for each fragment $\frag$ where $\frag_y \overset{*+}\leftrightarrow_S \frag$, create $\frag'$ which is same as $\frag$ except that 
\begin{itemize}
\item $\pvarsat{\inat{\frag'}} = \pvarsat{\inat{\frag}} \setminus \{x\{$,
\item $\pvarsat{\outat{\frag'}} = \pvarsat{\outat{\frag}} \setminus \set{x}$,
\item $\reachfromat{\inat{\frag'}} = \reachfromat{\inat{\frag}} \setminus \set{x}$ if $\tt x$ is a global variable,
\item $\reachtoat{\inat{\frag'}} = \reachtoat{\inat{\frag}} \setminus \set{x}$ if $\tt x$ is a global variable,
\item $\reachfromat{\outat{\frag'}} = \reachfromat{\outat{\frag}} \setminus \set{x}$ if $\tt x$ is a global variable,
\item $\reachtoat{\outat{\frag'}} = \reachtoat{\outat{\frag}} \setminus \set{x}$ if $\tt x$ is a global variable,
\end{itemize}
and add $\frag'$ to $\tt S_{post}$,
\item for each fragment $\frag$ where $\frag \overset{+\circ}\leftrightarrow_S \frag_y$, create $\frag'$ which is same as $\frag$ except that 
\begin{itemize}
\item $\pvarsat{\inat{\frag'}} = \pvarsat{\inat{\frag}} \setminus \set{x}$,
\item $\pvarsat{\outat{\frag'}} = \pvarsat{\outat{\frag}} \setminus \set{x}$,
\item $\reachfromat{\inat{\frag'}} = \reachfromat{\inat{\frag}} \setminus \set{x}$ if $\tt x$ is a global variable,
\item $\reachtoat{\inat{\frag'}} = \reachtoat{\inat{\frag}} \cup \set{x}$ if $\tt x$ is a global variable,
 \item $\reachfromat{\outat{\frag'}} = \reachfromat{\outat{\frag}} \setminus \set{x}$ if $\tt x$ is a global variable,
 \item $\reachtoat{\outat{\frag'}} = \reachtoat{\outat{\frag}} \cup \set{x}$ if $\tt x$ is a global variable,
\end{itemize}
and add $\frag'$ to $\tt S_{post}$,
\end{enumerate}	
\item[${\tt x.next:= y}$]
   The local abstract transformer is performed by several steps as follows: For each pair of fragments $\tt v_x$, $\frag_y$ in $\tt S$ where $\tt x \in v_x.{\tt vars}(\tt i)$, $\tt y \in \frag_y.{\tt vars}(\tt i)$, and $\tt v_x \overset{*}\hookrightarrow \frag_y$ or $\frag_y \overset{*}\hookrightarrow v_x$ or $\tt v_x \overset{**}\leftrightarrow \frag_y$
\begin{enumerate}
\item let $\tt v_{new}$ be the fragment of size 2 where $\tt tag(v_{new},i) = tag(v_x, i)$ and $\tt tag(v_{new},o) = tag(\frag_y,i)$ except that $\tt v_{new}.o.reachfrom = \frag_y.i.reachfrom \cup v_x.i.reachfrom$,
\item  for each fragment $\tt v \in S$ where $\tt v \overset{*}\hookrightarrow_S v_x$, we do as follows: For each subset $\tt regs$ of observer registers in $\tt v.reachto(\tt i) \cap \tt v_x.i.reachfrom$
\begin{itemize}
\item create $\tt v'$ which is same as $\tt v$, except that
\begin{itemize}
\item $\tt v'.reachto(\tt i) = (\tt v.reachto(\tt i) \cap v_x.i.reachfrom) \cup \frag_y.reachto(\tt i) \cup regs$.
\item $\tt v'.reachto(\tt o) = (\tt v.reachto(\tt o) \cap v_x.i.reachfrom) \cup \frag_y.reachto(\tt i) \cup regs$.
\end{itemize}
\item add $\tt v'$ to $\tt S_{post}$
\end{itemize}

\item for each fragment $\tt v$ where $\frag_y \overset{*}\hookrightarrow_S v$ or $\frag_y = v$,
  \bjcom{Question to Quy: Isn't $\frag_y = v$ included in $\frag_y \overset{*}\hookrightarrow_S v$?}
\begin{itemize}
	\item create  $\tt v'$ which is same as $\tt v$ except that
\begin{itemize}
 \item $\tt \reachfromat{\inat{\frag'}} = v_x.i.reachfrom \cup \tt \reachfromat{\inat{\frag}}$, 
\item $\tt \reachfromat{\outat{\frag'}} = v_x.i.reachfrom \cup \tt \reachfromat{\outat{\frag}}$, 
\end{itemize}
\item add $\tt v'$ to $\tt S_{post}$
\end{itemize}
\item for each fragment $\tt v$ where $\tt v_x \overset{**}\leftrightarrow_S v$ and either $\tt v \hookrightarrow_S \frag_y$ or $\frag_y \overset{*\circ}\leftrightarrow_S v$, 
\begin{itemize}
\item  create $\tt v'$ which is same as $\tt v$ except that $\tt \reachfromat{\outat{\frag'}} = v_x.i.reachfrom \cup \tt \reachfromat{\outat{\frag}}$, 
 \item add $\tt v'$ to $\tt S_{post}$ 
\end{itemize}


\item for each fragment $\tt v$ where $\tt v_x \overset{*}\hookrightarrow_S v$ and either $\tt v \hookrightarrow_S \frag_y$ or $\frag_y \overset{*\circ}\leftrightarrow_S v$,  
\begin{itemize}
\item create $\tt v'$ which is same as $\tt v$ then except that $\tt \reachfromat{\outat{\frag'}} = v_x.i.reachfrom \cup \tt \reachfromat{\outat{\frag}}$, 
\item for each subset $\tt regs$ of observer registers in $\reachfromat{\inat{\frag'}} \cap \tt v_x.i.reachfrom$
\begin{itemize}
\item create $\tt v''$ which is same as $\tt v'$, except that $\tt v''.i.reachfrom = (\reachfromat{\inat{\frag'}} \setminus v_x.i.reachfrom) \cup regs$.
\item add $\tt v''$ to $\tt S_{post}$
\end{itemize}
\end{itemize}

\item for each fragment $\tt v$ where $\tt v_x \overset{**}\leftrightarrow_S v$ and either $\tt v \overset{+}\hookrightarrow_S \frag_y$ or $\tt v \overset{*+}\leftrightarrow_S \frag_y$, 
\begin{itemize}
\item create $\tt v'$ which is same as $\tt v$
\item  add $\tt v'$ to $\tt S_{post}$
\end{itemize}

\item for each fragment $\tt v$ where $\tt v_x \overset{*}\hookrightarrow_S v$ and either $\tt v \overset{+}\hookrightarrow_S \frag_y$ or $\frag_y \overset{*+}\leftrightarrow v$, then for each subset $\tt regs$ of observer registers in $\tt \reachfromat{\inat{\frag}} \cap \tt v_x.i.reachfrom$, 
\begin{itemize}
\item create $\tt v'$ which is same as $\tt v$, except that $\reachfromat{\inat{\frag'}} = (\tt \reachfromat{\inat{\frag}} \setminus v_x.i.reachfrom) \cup regs$. 
 \item for each set $\tt regs'$ of observer registers in $\reachfromat{\outat{\frag'}} \cap \tt v_x.i.reachfrom$,  
\begin{itemize}
\item create $\tt v''$ which is same as $\tt v'$, except that $\tt v''.o.reachfrom = (\reachfromat{\outat{\frag'}} \setminus v_x.i.reachfrom) \cup regs'$. 
\item add $\tt v''$ to $\tt S_{post}$
\end{itemize}
\end{itemize}
\item add $\tt v_{new}$ to $\tt S_{post}$
\end{enumerate}

\end{description}

\todo[inline]{After here, I will insert text on Intersection}

\bjcom{This paragraph needs revision}
For a fragment $\tt v$ and index $\tt i$, let $\tt tag(v,i)$ be the tuple $\tt \tuple{\tt vars(i), data(i),{\tt reachfrom}(\tt i), {\tt reachto}(\tt i), private(i )}$. Let $\tt gtag(v,i)$ be the tuple $\tt \tuple{\tt gvars(i), gdata(i),{\tt greachfrom}(\tt i), {\tt reachto}(\tt i), private(i )}$ where $\tt gvars(i)$ is the set of global variables in $\tt vars(i)$ and $\tt greachfrom(i)$ is the set of global variables in $\tt reachfrom(i)$. Note that in each fragment $\tt v$, for two indices $\tt i_1$, $\tt i_2$ where $\tt next(i_1) = i_2$, we always have $\tt {\tt reachfrom}(\tt i_1) \subseteq \tt {\tt reachfrom}(\tt i_2)$ and $\tt {\tt reachto}(\tt i_1) = \tt {\tt reachto}(\tt i_2) \cup X$ where $\tt X$ is the set of global variables in ${\tt vars}(\tt i_1)$ and observer registers $\tt z$ which are equal to $\tt i_1.d$ for $\intgrs$-field $\tt d$. 


\section{Abstract transformers for skip-lists}
Let us show how to perform the abstract transformer for skip-list programs on the set of fragments $\tt S$ depending on the particular statement.  We consider fragments of size $2$ where $\tt I_{inp} = \{i\}$, $\tt I_{out} = \{{o}\}$, and $\tt next(i) = {o}$ and fragments of size $1$ where $\tt I_{inp} = \{i\}$, $\tt I_{out} = \emptyset$, and $\tt next(i) = null$ or $\tt next(i) = \bot$. For each fragment $\tt v$, let $\tt v.level \in \{1,2\}$ be the level of $\tt v$.

\paragraph{Local Abstract Transformers:} First, let us show the abstract transformer on the set of fragment $\tt S$ in the fragment of the concurrent thread. Let $\tt S_1$ be set of fragments of level 1 in $\tt S$, $\tt S_2$ be set of fragments of level 2 in $\tt S$. For each program statement, let $\tt S_{post}$ be the set of fragments after executing the statement. Let $\tt S_{post}$ be initialized as the empty set.  Let $\tt R$ be the set of pairs of fragments. Intuitively, in each element in $\tt R$, the second fragment is the transformation of the first fragment. Let $\tt R$ be initialized as the empty set.
\begin{itemize}
\item ${\tt x := y}$: The transformer is performed as follows:  For each fragment $\tt v_y \in S_1$ where $\tt y \in v_y.{\tt vars}(\tt i)$, 
%\begin{itemize}
\begin{enumerate}
\item  for each fragment $\tt v \in S_1$ where $\tt v \hookrightarrow_S v_y$, create $\tt v'$ which is same as $\tt v$ except that
\begin{itemize}
\item $\pvarsat{\inat{\frag'}} = \pvarsat{\inat{\frag}} \setminus \set{x}$,
\item $\pvarsat{\outat{\frag'}} = \pvarsat{\outat{\frag}} \cup \set{x}$,
\item if $\tt x$ is a global variable
\begin{itemize}
\item $\reachfromat{\inat{\frag'}} = \reachfromat{\inat{\frag}} \setminus \set{x}$,
\item $\reachtoat{\inat{\frag'}} = \reachtoat{\inat{\frag}} \cup \set{x}$,
 \item $\reachfromat{\outat{\frag'}} = \reachfromat{\outat{\frag}} \cup \set{x}$,
\item $\reachtoat{\outat{\frag'}} = \reachtoat{\outat{\frag}} \cup \set{x}$,
\end{itemize}
\end{itemize}
then add $\tt v'$ to $\tt S_{post}$, and $\tt (v,v')$ to $\tt R$
\item for each fragment $\tt v \in S_1$ where $\tt v \overset{+}\hookrightarrow_S v_y$, create $\tt v'$ which is same as $\tt v$ except that
\begin{itemize}
\item $\pvarsat{\inat{\frag'}} = \pvarsat{\inat{\frag}} \setminus \set{x}$,
\item $\pvarsat{\outat{\frag'}} = \pvarsat{\outat{\frag}} \setminus \set{x}$,
\item if $\tt x$ is a global variable
\begin{itemize}
\item $\reachfromat{\inat{\frag'}} = \reachfromat{\inat{\frag}} \setminus \set{x}$,
\item $\reachtoat{\inat{\frag'}} = \reachtoat{\inat{\frag}} \cup \set{x}$,
\item $\reachfromat{\outat{\frag'}} = \reachfromat{\outat{\frag}} \setminus \set{x}$,
\item $\reachtoat{\outat{\frag'}} = \reachtoat{\outat{\frag}} \cup \set{x}$,
\end{itemize}
\end{itemize}
then add $\tt v'$ to $\tt S_{post}$, and $\tt (v,v')$ to $\tt R$
\item for each fragment $\tt v \in S_1$ where $\tt v_y \overset{*}\hookrightarrow_S v$, create $\tt v'$ which is same as $\tt v$ except that
\begin{itemize}
\item $\pvarsat{\inat{\frag'}} = \pvarsat{\inat{\frag}} \setminus \set{x}$,
\item $\pvarsat{\outat{\frag'}} = \pvarsat{\outat{\frag}} \setminus \set{x}$,
\item if $\tt x$ is a global variable
\begin{itemize}
\item $\reachfromat{\inat{\frag'}} = \reachfromat{\inat{\frag}} \cup \set{x}$,
\item $\reachtoat{\inat{\frag'}} = \reachtoat{\inat{\frag}} \setminus \set{x}$,
 \item $\reachfromat{\outat{\frag'}} = \reachfromat{\outat{\frag}} \cup \set{x}$,
 \item $\reachtoat{\outat{\frag'}} = \reachtoat{\outat{\frag}} \setminus \set{x}$,
\end{itemize}
\end{itemize}
then add $\tt v'$ to $\tt S_{post}$, and $\tt (v,v')$ to $\tt S'$
\item for each fragment $\tt v$ where $\tt v_y \overset{*+}\leftrightarrow_S v$, create $\tt v'$ which is same as $\tt v$ except that
\begin{itemize}
\item $\pvarsat{\inat{\frag'}} = \pvarsat{\inat{\frag}} \setminus \set{x}$,
\item $\pvarsat{\outat{\frag'}} = \pvarsat{\outat{\frag}} \setminus \set{x}$,
\item if $\tt x$ is a global variable
\begin{itemize}
\item $\reachfromat{\inat{\frag'}} = \reachfromat{\inat{\frag}} \setminus \set{x}$,
\item $\reachtoat{\inat{\frag'}} = \reachtoat{\inat{\frag}} \setminus \set{x}$,
\item $\reachfromat{\outat{\frag'}} = \reachfromat{\outat{\frag}} \setminus \set{x}$,
\item $\reachtoat{\outat{\frag'}} = \reachtoat{\outat{\frag}} \setminus \set{x}$,
\end{itemize}
\end{itemize}
then add $\tt v'$ to $\tt S_{post}$, and $\tt (v,v')$ to $\tt S'$
\item for each fragment $\tt v \in S_1$ where $\tt v_y \overset{*\circ}\leftrightarrow_S v$, create $\tt v'$ which is same as $\tt v$ except that
\begin{itemize}
\item $\pvarsat{\inat{\frag'}} = \pvarsat{\inat{\frag}} \setminus \set{x}$,
\item $\pvarsat{\outat{\frag'}} = \pvarsat{\outat{\frag}} \setminus \set{x}$,
\item if $\tt x$ is a global variable
\begin{itemize}
\item $\reachfromat{\inat{\frag'}} = \reachfromat{\inat{\frag}} \setminus \set{x}$,
\item $\reachtoat{\inat{\frag'}} = \reachtoat{\inat{\frag}} \setminus \set{x}$,
\item $\reachfromat{\outat{\frag'}} = \reachfromat{\outat{\frag}} \cup \set{x}$,
\item $\reachtoat{\outat{\frag'}} = \reachtoat{\outat{\frag}} \setminus \set{x}$,
\end{itemize}
\end{itemize}
then add $\tt v'$ to $\tt S_{post}$, and $\tt (v,v')$ to $\tt S'$
\item create $\tt v'$ which is same as $\tt v_y$ except that
\begin{itemize}
\item $\pvarsat{\inat{\frag'}} = \pvarsat{\inat{\frag}} \cup \set{x}$,
\item $\pvarsat{\outat{\frag'}} = \pvarsat{\outat{\frag}} \setminus \set{x}$,
\item if $\tt x$ is a global variable
\begin{itemize}
\item $\reachtoat{\inat{\frag'}} = \reachtoat{\inat{\frag}} \cup \set{x}$,
 \item $\reachfromat{\inat{\frag'}} = \reachfromat{\inat{\frag}} \cup \set{x}$,
\item $\reachfromat{\outat{\frag'}} = \reachfromat{\outat{\frag}} \cup \set{x}$,
\item $\reachtoat{\outat{\frag'}} = \reachtoat{\outat{\frag}} \setminus \set{x}$,
\end{itemize}
\end{itemize}
then add $\tt v'$ to $\tt S_{post}$, and $\tt (v,v')$ to $\tt S'$
%\item for each fragment $\tt v'$ in $\tt S_{post}$, assign $\tt v'.data.(i).(y.d)$ to $\tt v'.data.(i).(x.d)$ for $\tt i \in \{i, o\}$.
\item for each fragment $\tt v \in S_2$ we do as follows. For each $\tt (v_1,v_1')$, $\tt (v_2,v_2') \in R$, for each pair of indices $\tt i_1$, $\tt i_2$ such that $\tt i_1 \in \{v_1.i, v_1.o\}$, $\tt i_2 \in \{v_2.i, v_2.o\}$, $\tt tag(v,i) = tag(v_1, i_1)$, and $\tt tag(v,o) = tag(v_2, i_2)$. Create $\tt v'$ which is same as $\tt v$ except that 
\begin{itemize}
\item $\pvarsat{\inat{\frag'}} = v_1'.i_1.vars$,
\item $\pvarsat{\outat{\frag'}} = v_2'.i_2.vars$,

\item $\reachfromat{\inat{\frag'}} = v_1'.i_1.reachfrom$,
\item $\reachfromat{\outat{\frag'}} = v_2'.i_2.reachfrom$,
\item $\reachtoat{\inat{\frag'}} = v_1'.i_1.reachto$,
\item $\reachtoat{\outat{\frag'}} = v_2'.i_2.reachto$,
\end{itemize} then add $\tt v'$ to $\tt S_{post}$. 
\end{enumerate}
%\item If there does not exist such $\tt v_y$ in $\tt S_1$
%\end{itemize}
%\begin{enumerate}
%%1%
%\item  for each fragment $\tt v$ where 
%\begin{itemize}
%	\item $\tt tag(v,o) = tag(v_y,i)$,
%	\item $\tt tag(v,i) = tag(v_1,i)$ where $\tt v_1 \overset{*}\hookrightarrow_S v_y$,
%\end{itemize}
%
%%\begin{itemize}
%%\item $\pvarsat{\inat{\frag'}} = \pvarsat{\inat{\frag}} \setminus \set{x}$,
%%\item $\pvarsat{\outat{\frag'}} = \pvarsat{\outat{\frag}} \cup \set{x}$,
%%\item if $\tt x$ is a global variable
%%\begin{itemize}
%%\item $\reachfromat{\inat{\frag'}} = \reachfromat{\inat{\frag}} \setminus \set{x}$,
%%\item $\reachtoat{\inat{\frag'}} = \reachtoat{\inat{\frag}} \cup \set{x}$,
%% \item $\reachfromat{\outat{\frag'}} = \reachfromat{\outat{\frag}} \cup \set{x}$,
%%\item $\reachtoat{\outat{\frag'}} = \reachtoat{\outat{\frag}} \cup \set{x}$,
%%\end{itemize}
%%\end{itemize}
%%and add $\tt v'$ to $\tt S_{post}$,
%%2%
%\item  for each fragment $\tt v$ where
%\begin{itemize}
%	\item $\tt tag(v,o) = tag(v_y,i)$,
%	\item $\tt tag(v,i) = tag(v_1,i)$ where $\tt v_y \overset{*+}\leftrightarrow_S v_1$ or $\tt v_1 \overset{*\circ}\leftrightarrow_S v_y$
%\end{itemize}
%%\begin{itemize}
%%\item $\pvarsat{\inat{\frag'}} = \pvarsat{\inat{\frag}} \setminus \set{x}$,
%%\item $\pvarsat{\outat{\frag'}} = \pvarsat{\outat{\frag}} \cup \set{x}$,
%%\item if $\tt x$ is a global variable
%%\begin{itemize}
%%\item $\reachfromat{\inat{\frag'}} = \reachfromat{\inat{\frag}} \setminus \set{x}$,
%%\item $\reachtoat{\inat{\frag'}} = \reachtoat{\inat{\frag}} \setminus \set{x}$,
%% \item $\reachfromat{\outat{\frag'}} = \reachfromat{\outat{\frag}} \cup \set{x}$,
%%\item $\reachtoat{\outat{\frag'}} = \reachtoat{\outat{\frag}} \cup \set{x}$,
%%\end{itemize}
%%\end{itemize}
%%and add $\tt v'$ to $\tt S_{post}$,
%
%%3%
%\item  for each fragment $\tt v$ where
%\begin{itemize}
%	\item $\tt tag(v,o) = tag(v_y,i)$,
%	\item $\tt tag(v,i) = tag(v_1,i)$ where $\tt v_y \overset{*}\hookrightarrow_S v_1$ 
%\end{itemize}
%
%%4%
%
%\item  for each fragment $\tt v$ where 
%\begin{itemize}
%	\item $\tt tag(v,o) = tag(v_1,i)$ where $\tt v_1 \overset{*}\hookrightarrow_S v_y$,
%	\item $\tt tag(v,i) = tag(v_2,i)$ where $\tt v_2 \overset{*}\hookrightarrow_S v_y$,
%\end{itemize}
%
%
%%5%
%\item  for each fragment $\tt v$ where
%\begin{itemize}
%	\item $\tt tag(v,o) = tag(v_1,i)$ where $\tt v_1 \overset{*}\hookrightarrow_S v_y$,
%	\item $\tt tag(v,i) = tag(v_2,i)$ where $\tt v_y \overset{*+}\leftrightarrow_S v_2$ or $\tt v_2 \overset{*\circ}\leftrightarrow_S v_y$
%\end{itemize}
%
%
%%6%
%\item  for each fragment $\tt v$ where
%\begin{itemize}
%	\item $\tt tag(v,o) = tag(v_1,i)$ where $\tt v_1 \overset{*}\hookrightarrow_S v_y$,
%	\item $\tt tag(v,i) = tag(v_2,i)$ where $\tt v_y \overset{*}\hookrightarrow_S v_2$ 
%\end{itemize}
%
%
%%7%
%
%\item  for each fragment $\tt v$ where 
%\begin{itemize}
%	\item $\tt tag(v,o) = tag(v_1,i)$ where $\tt v_y \overset{*}\hookrightarrow_S v_1$,
%	\item $\tt tag(v,i) = tag(v_2,i)$ where $\tt v_2 \overset{*}\hookrightarrow_S v_y$,
%\end{itemize}
%
%
%%8%
%\item  for each fragment $\tt v$ where
%\begin{itemize}
%	\item $\tt tag(v,o) = tag(v_1,i)$ where $\tt v_y \overset{*}\hookrightarrow_S v_1$,
%	\item $\tt tag(v,i) = tag(v_2,i)$ where $\tt v_y \overset{*+}\leftrightarrow_S v_2$ or $\tt v_2 \overset{*\circ}\leftrightarrow_S v_y$
%\end{itemize}
%
%
%%9%
%\item  for each fragment $\tt v$ where
%\begin{itemize}
%	\item $\tt tag(v,o) = tag(v_1,i)$ where $\tt v_y \overset{*}\hookrightarrow_S v_1$,
%	\item $\tt tag(v,i) = tag(v_2,i)$ where $\tt v_y \overset{*}\hookrightarrow_S v_2$ 
%\end{itemize}
%
%
%%10%
%
%\item  for each fragment $\tt v$ where 
%\begin{itemize}
%	\item $\tt tag(v,o) = tag(v_1,i)$ where $\tt v_y \overset{*+}\leftrightarrow_S v_1$ or $\tt v_1 \overset{*\circ} \leftrightarrow_S v_y$,
%	\item $\tt tag(v,i) = tag(v_2,i)$ where $\tt v_2 \overset{*}\hookrightarrow_S v_y$,
%\end{itemize}
%
%
%%11%
%\item  for each fragment $\tt v$ where
%\begin{itemize}
%	\item $\tt tag(v,o) = tag(v_1,i)$ where $\tt v_y \overset{*+}\leftrightarrow_S v_1$ or $\tt v_1 \overset{*\circ} \leftrightarrow_S v_y$,
%	\item $\tt tag(v,i) = tag(v_2,i)$ where $\tt v_y \overset{*+}\leftrightarrow_S v_2$ or $\tt v_2 \overset{*\circ}\leftrightarrow_S v_y$
%\end{itemize}
%
%
%%12%
%\item  for each fragment $\tt v$ where
%\begin{itemize}
%	\item $\tt tag(v,o) = tag(v_1,i)$ where $\tt v_y \overset{*+}\leftrightarrow_S v_1$ or $\tt v_1 \overset{*\circ} \leftrightarrow_S v_y$,
%	\item $\tt tag(v,i) = tag(v_2,i)$ where $\tt v_y \overset{*}\hookrightarrow_S v_2$ 
%\end{itemize}
%
%\end{enumerate}
	
	
%%%%%%%%%%%%%%%%%%%%%%%%%%%%%%%%%%%%%%%%%%%%%%%%%% X := Y.NEXT1 %%%%%%%%%%%%%%%%%%%%%%%%%%%%%%%%%%%%%%%%%%%%%%%%
\item ${\tt x := y.next1}$: The local abstract transformer is quite similar to the previous case with slightly differences. For each fragment $\tt v_y \in S$ where $\tt y \in {\tt vars}(\tt i)$,
%\begin{enumerate}
%
%%1%
%\item  for each fragment $\tt v$ where 
%\begin{itemize}
%	\item $\tt tag(v,o) = tag(v_y,o)$,
%	\item $\tt tag(v,i) = tag(v_1,i)$ where $\tt v_1 \overset{*}\hookrightarrow_S v_y$ or $\tt v_1 = v_y$,
%\end{itemize}
%
%%2%
%\item  for each fragment $\tt v$ where
%\begin{itemize}
%	\item $\tt tag(v,o) = tag(v_y,o)$,
%	\item $\tt tag(v,i) = tag(v_1,i)$ where $\tt v_y \overset{*+}\leftrightarrow_S v_1$ or $\tt v_1 \overset{*\circ}\leftrightarrow_S v_y$
%\end{itemize}
%
%%3%
%\item  for each fragment $\tt v$ where
%\begin{itemize}
%	\item $\tt tag(v,o) = tag(v_y,o)$,
%	\item $\tt tag(v,i) = tag(v_1,i)$ where $\tt v_y \overset{+}\hookrightarrow_S v_1$ 
%\end{itemize}
%
%%-------------------------------------------4------------------------------------------%
%
%\item  for each fragment $\tt v$ where 
%\begin{itemize}
%	\item $\tt tag(v,o) = tag(v_1,i)$ where $\tt v_1 \overset{*}\hookrightarrow_S v_y$ or $\tt v_1 = v_y$,
%	\item $\tt tag(v,i) = tag(v_2,i)$ where $\tt v_2 \overset{*}\hookrightarrow_S v_y$,
%\end{itemize}
%
%
%%-------------------------------------------5------------------------------------------%
%\item  for each fragment $\tt v$ where
%\begin{itemize}
%	\item $\tt tag(v,o) = tag(v_1,i)$ where $\tt v_1 \overset{*}\hookrightarrow_S v_y$ or $\tt v_1 = v_y$,
%	\item $\tt tag(v,i) = tag(v_2,i)$ where $\tt v_y \overset{*+}\leftrightarrow_S v_2$ or $\tt v_2 \overset{*\circ}\leftrightarrow_S v_y$
%\end{itemize}
%
%
%%-------------------------------------------6------------------------------------------%
%\item  for each fragment $\tt v$ where
%\begin{itemize}
%	\item $\tt tag(v,o) = tag(v_1,i)$ where $\tt v_1 \overset{*}\hookrightarrow_S v_y$ or $\tt v_1 = v_y$,
%	\item $\tt tag(v,i) = tag(v_2,i)$ where $\tt v_y \overset{*}\hookrightarrow_S v_2$ 
%\end{itemize}
%
%
%%-------------------------------------------7-------------------------------------------%
%
%\item  for each fragment $\tt v$ where 
%\begin{itemize}
%	\item $\tt tag(v,o) = tag(v_1,i)$ where $\tt v_y \overset{*+}\leftrightarrow_S v_1$ or $\tt v_1 \overset{*\circ}\leftrightarrow_S v_y$,
%	\item $\tt tag(v,i) = tag(v_2,i)$ where $\tt v_2 \overset{*}\hookrightarrow_S v_y$,
%\end{itemize}
%
%
%%------------------------------------------8-------------------------------------------%
%\item  for each fragment $\tt v$ where
%\begin{itemize}
%	\item $\tt tag(v,o) = tag(v_1,i)$ where $\tt v_y \overset{*+}\leftrightarrow_S v_1$ or $\tt v_1 \overset{*\circ}\leftrightarrow_S v_y$,
%	\item $\tt tag(v,i) = tag(v_2,i)$ where $\tt v_y \overset{*+}\leftrightarrow_S v_2$ or $\tt v_2 \overset{*\circ}\leftrightarrow_S v_y$
%\end{itemize}
%
%
%%-----------------------------------------9-------------------------------------------%
%\item  for each fragment $\tt v$ where
%\begin{itemize}
%	\item $\tt tag(v,o) = tag(v_1,i)$ where $\tt v_y \overset{*+}\leftrightarrow_S v_1$ or $\tt v_1 \overset{*\circ}\leftrightarrow_S v_y$,
%	\item $\tt tag(v,i) = tag(v_2,i)$ where $\tt v_y \overset{*}\hookrightarrow_S v_2$ 
%\end{itemize}
%
%
%%-----------------------------------------10-------------------------------------------%
%
%\item  for each fragment $\tt v$ where 
%\begin{itemize}
%	\item $\tt tag(v,o) = tag(v_1,i)$ where $\tt v_y \overset{*}\hookrightarrow_S v_1$,
%	\item $\tt tag(v,i) = tag(v_2,i)$ where $\tt v_2 \overset{*}\hookrightarrow_S v_y$,
%\end{itemize}
%
%
%%-----------------------------------------11-------------------------------------------%
%\item  for each fragment $\tt v$ where
%\begin{itemize}
%	\item $\tt tag(v,o) = tag(v_1,i)$  where $\tt v_y \overset{*}\hookrightarrow_S v_1$,
%	\item $\tt tag(v,i) = tag(v_2,i)$ where $\tt v_y \overset{*+}\leftrightarrow_S v_2$ or $\tt v_2 \overset{*\circ}\leftrightarrow_S v_y$
%\end{itemize}
%
%
%%-----------------------------------------12-------------------------------------------%
%\item  for each fragment $\tt v$ where
%\begin{itemize}
%	\item $\tt tag(v,o) = tag(v_1,i)$ where $\tt v_1y\overset{*}\hookrightarrow_S v_1$,
%	\item $\tt tag(v,i) = tag(v_2,i)$ where $\tt v_y \overset{*}\hookrightarrow_S v_2$ 
%\end{itemize}
%
%
%\end{enumerate}




\begin{enumerate}
\item  for each fragment $\tt v \in S_1$ where $\tt v \overset{*}\hookrightarrow_S v_y$, create $\tt v'$ which is same as $\tt v$ except that
\begin{itemize}
\item $\pvarsat{\inat{\frag'}} = \pvarsat{\inat{\frag}} \setminus \set{x}$,
\item $\pvarsat{\outat{\frag'}} = \pvarsat{\outat{\frag}} \setminus \set{x}$,
\item if $\tt x$ is a global variable
\begin{itemize}
\item $\reachfromat{\inat{\frag'}} = \reachfromat{\inat{\frag}} \setminus \set{x}$,
\item $\reachtoat{\inat{\frag'}} = \reachtoat{\inat{\frag}} \cup \set{x}$,
\item $\reachfromat{\outat{\frag'}} = \reachfromat{\outat{\frag}} \setminus \set{x}$,
\item $\reachtoat{\outat{\frag'}} = \reachtoat{\outat{\frag}} \cup \set{x}$,
\end{itemize}
\end{itemize}
then add $\tt v'$ to $\tt S_{post}$, and $\tt (v,v')$ to $\tt R$
\item for each fragment $\tt v \in S_1$ where $\tt v_y \hookrightarrow_S v$, create $\tt v'$ which is same as $\tt v$ then
\begin{itemize}
\item $\pvarsat{\inat{\frag'}} = \pvarsat{\inat{\frag}} \cup \set{x}$,
\item $\pvarsat{\outat{\frag'}} = \pvarsat{\outat{\frag}} \setminus \set{x}$,
\item if $\tt x$ is a global variable
\begin{itemize}
\item $\reachfromat{\inat{\frag'}} = \reachfromat{\inat{\frag}} \cup \set{x}$,
 \item $\reachtoat{\outat{\frag'}} = \reachtoat{\outat{\frag}} \cup \set{x}$,
 \item $\reachfromat{\outat{\frag'}} = \reachfromat{\outat{\frag}} \cup \set{x}$,
 \item $\reachtoat{\outat{\frag'}} = \reachtoat{\outat{\frag}} \setminus \set{x}$,
\end{itemize}
\end{itemize}
then add $\tt v'$ to $\tt S_{post}$, and $\tt (v,v')$ to $\tt R$
\item for each fragment $\tt v \in S_1$ where $\tt v_y \leftrightarrow_S v$, create $\tt v'$ which is same as $\tt v$ except that
\begin{itemize} 
\item $\pvarsat{\inat{\frag'}} = \pvarsat{\inat{\frag}} \setminus \set{x}$,
\item $\pvarsat{\outat{\frag'}} = \pvarsat{\outat{\frag}} \cup \set{x}$,
\item if $\tt x$ is a global variable
\begin{itemize}
\item $\reachfromat{\inat{\frag'}} = \reachfromat{\inat{\frag}} \setminus \set{x}$,
\item $\reachtoat{\inat{\frag'}} = \reachtoat{\inat{\frag}} \cup \set{x}$,
 \item $\reachfromat{\outat{\frag'}} = \reachfromat{\outat{\frag}} \cup \set{x}$,
 \item $\reachtoat{\outat{\frag'}} = \reachtoat{\outat{\frag}} \cup \set{x}$,
\end{itemize}
\end{itemize}
then add $\tt v'$ to $\tt S_{post}$, and $\tt (v,v')$ to $\tt R$

\item for each fragment $\tt v \in S_1$ where $\tt v_y \overset{+}\hookrightarrow_S v$, create $\tt v'$ which is same as $\tt v$ except that
\begin{itemize}
\item $\pvarsat{\inat{\frag'}} = \pvarsat{\inat{\frag}} \setminus \set{x}$,
\item $\pvarsat{\outat{\frag'}} = \pvarsat{\outat{\frag}} \setminus \set{x}$,
\item if $\tt x$ is a global variable
\begin{itemize}
\item $\reachfromat{\inat{\frag'}} = \reachfromat{\inat{\frag}} \cup \set{x}$,
\item $\reachtoat{\inat{\frag'}} = \reachtoat{\inat{\frag}} \setminus \set{x}$,
 \item $\reachfromat{\outat{\frag'}} = \reachfromat{\outat{\frag}} \cup \set{x}$,
 \item $\reachtoat{\outat{\frag'}} = \reachtoat{\outat{\frag}} \setminus \set{x}$,
\end{itemize}
\end{itemize}
then add $\tt v'$ to $\tt S_{post}$, and $\tt (v,v')$ to $\tt R$
\item for each fragment $\tt v \in S_1$ where $\tt v_y \overset{*\circ}\leftrightarrow_S v$, create $\tt v'$ which is same as $\tt v$ except that
\begin{itemize}
\item $\pvarsat{\inat{\frag'}} = \pvarsat{\inat{\frag}} \setminus \set{x}$,
\item $\pvarsat{\outat{\frag'}} = \pvarsat{\outat{\frag}} \setminus \set{x}$,
\item if $\tt x$ is a global variable
\begin{itemize}
\item $\reachfromat{\inat{\frag'}} = \reachfromat{\inat{\frag}} \setminus \set{x}$,
\item $\reachtoat{\inat{\frag'}} = \reachtoat{\inat{\frag}} \setminus \set{x}$,
\item $\reachfromat{\outat{\frag'}} = \reachfromat{\outat{\frag}} \cup \set{x}$,
\item $\reachtoat{\outat{\frag'}} = \reachtoat{\outat{\frag}} \setminus \set{x}$,
\end{itemize}
\end{itemize}
then add $\tt v'$ to $\tt S_{post}$, and $\tt (v,v')$ to $\tt R$
\item for each fragment $\tt v \in S_1$ where $\tt v_y \overset{*+}\leftrightarrow_S v$, create $\tt v'$ which is same as $\tt v$ except that 
\begin{itemize}
\item $\pvarsat{\inat{\frag'}} = \pvarsat{\inat{\frag}} \setminus \{x\{$,
\item $\pvarsat{\outat{\frag'}} = \pvarsat{\outat{\frag}} \setminus \set{x}$,
\item if $\tt x$ is a global variable
\begin{itemize}
\item $\reachfromat{\inat{\frag'}} = \reachfromat{\inat{\frag}} \setminus \set{x}$ ,
\item $\reachtoat{\inat{\frag'}} = \reachtoat{\inat{\frag}} \setminus \set{x}$,
\item $\reachfromat{\outat{\frag'}} = \reachfromat{\outat{\frag}} \setminus \set{x}$,
\item $\reachtoat{\outat{\frag'}} = \reachtoat{\outat{\frag}} \setminus \set{x}$,
\end{itemize}
\end{itemize}
then add $\tt v'$ to $\tt S_{post}$, and $\tt (v,v')$ to $\tt R$
\item for each fragment $\tt v \in S_1$ where $\tt v \overset{+\circ}\leftrightarrow_S v_y$, create $\tt v'$ which is same as $\tt v$ except that 
\begin{itemize}
\item $\pvarsat{\inat{\frag'}} = \pvarsat{\inat{\frag}} \setminus \set{x}$,
\item $\pvarsat{\outat{\frag'}} = \pvarsat{\outat{\frag}} \setminus \set{x}$,
\item if $\tt x$ is a global variable
\begin{itemize}
\item $\reachfromat{\inat{\frag'}} = \reachfromat{\inat{\frag}} \setminus \set{x}$,
\item $\reachtoat{\inat{\frag'}} = \reachtoat{\inat{\frag}} \cup \set{x}$,
 \item $\reachfromat{\outat{\frag'}} = \reachfromat{\outat{\frag}} \setminus \set{x}$,
 \item $\reachtoat{\outat{\frag'}} = \reachtoat{\outat{\frag}} \cup \set{x}$,
\end{itemize}
\end{itemize}
then add $\tt v'$ to $\tt S_{post}$, and $\tt (v,v')$ to $\tt R$

\item create $\tt v'$ which is same as $\tt v_y$ except that
\begin{itemize}
\item $\pvarsat{\inat{\frag'}} = \pvarsat{\inat{\frag}} \setminus \set{x}$,
\item $\pvarsat{\outat{\frag'}} = \pvarsat{\outat{\frag}} \cup \set{x}$,
\item if $\tt x$ is a global variable
\begin{itemize}
\item $\reachfromat{\inat{\frag'}} = \reachfromat{\inat{\frag}} \setminus \set{x}$,
\item $\reachtoat{\inat{\frag'}} = \reachtoat{\inat{\frag}} \cup \set{x}$,
 \item $\reachfromat{\outat{\frag'}} = \reachfromat{\outat{\frag}} \cup \set{x}$,
 \item $\reachtoat{\outat{\frag'}} = \reachtoat{\outat{\frag}} \cup \set{x}$,
\end{itemize}
\end{itemize}
then add $\tt v'$ to $\tt S_{post}$, and $\tt (v,v')$ to $\tt R$
%\item for each fragment $\tt v'$ in $\tt S_o$, assign $\tt v'.data.(i).(y.d)$ to $\tt v'.data.(i).(x.d)$ for $\tt i \in \{i, o\}$.
%

\item for each fragment $\tt v \in S_2$ we do as follows. For each $\tt (v_1,v_1')$, $\tt (v_2,v_2') \in R$, for each pair of indices $\tt i_1$, $\tt i_2$ such that $\tt i_1 \in \{v_1.i, v_1.o\}$, $\tt i_2 \in \{v_2.i, v_2.o\}$, $\tt tag(v,i) = tag(v_1, i_1)$, and $\tt tag(v,o) = tag(v_2, i_2)$. Create $\tt v'$ which is same as $\tt v$ except that 
\begin{itemize}
\item $\pvarsat{\inat{\frag'}} = v_1'.i_1.vars$,
\item $\pvarsat{\outat{\frag'}} = v_2'.i_2.vars$,

\item $\reachfromat{\inat{\frag'}} = v_1'.i_1.reachfrom$,
\item $\reachfromat{\outat{\frag'}} = v_2'.i_2.reachfrom$,
\item $\reachtoat{\inat{\frag'}} = v_1'.i_1.reachto$,
\item $\reachtoat{\outat{\frag'}} = v_2'.i_2.reachto$,
\end{itemize} then add $\tt v'$ to $\tt S_{post}$. 
\end{enumerate}	




%%%%%%%%%%%%%%%%%%%%%%%%%%%%%%%%%%%%%%%%%%%%%%%%%% X.NEXT1 := Y %%%%%%%%%%%%%%%%%%%%%%%%%%%%%%%%%%%%%%%%%%%%%%%%

\item  ${\tt x.next1:= y}$: The local abstract transformer is performed by several steps as follows: For each pair of fragments $\tt v_x$, $\tt v_y$ in $\tt S_1$ where $\tt x \in v_x.{\tt vars}(\tt i)$, $\tt y \in v_y.{\tt vars}(\tt i)$, and $\tt v_x \overset{*}\hookrightarrow v_y$ or $\tt v_y \overset{*}\hookrightarrow v_x$ or $\tt v_x \overset{**}\leftrightarrow v_y$, let $\tt R_1$, $\tt R_2$, $\tt R_3$, $\tt R_4$, $\tt R_5$, $\tt R_6$ be initialized as $\tt R$,  
\begin{enumerate}
\item let $\tt v_{new}$ be the fragment of size 2 and of level 1 where $\tt tag(v_{new},i) = tag(v_x, i)$ and $\tt tag(v_{new},o) = tag(v_y,i)$ except that $\tt v_{new}.o.reachfrom = v_y.i.reachfrom \cup v_x.i.reachfrom$,
\item  for each fragment $\tt v \in S_1$ where $\tt v \overset{*}\hookrightarrow_S v_x$, we do as follows: For each subset $\tt regs$ of observer registers in $\tt v.reachto(\tt i) \cap \tt v_x.i.reachfrom$


\begin{itemize}
\item create $\tt v'$ which is same as $\tt v$, except that
\begin{itemize}
\item $\reachtoat{\inat{\frag'}} = (\tt \reachtoat{\inat{\frag}} \cap v_x.i.reachfrom) \cup v_y.i.reachto \cup regs$.
\item $\reachtoat{\outat{\frag'}} = (\tt \reachtoat{\outat{\frag}} \cap v_x.i.reachfrom) \cup v_y.i.reachto \cup regs$.
\end{itemize}
\item add $\tt v'$ to $\tt S_{post}$ 
\item add $\tt (v,v')$ to $\tt R_1$
\end{itemize}

\item for each fragment $\tt v \in S_1$ where $\tt v_y \overset{*}\hookrightarrow_S v$ or $\tt v_y = v$, 
\begin{itemize}
	\item create  $\tt v'$ which is same as $\tt v$ except that
\begin{itemize}
 \item $\tt \reachfromat{\inat{\frag'}} = v_x.i.reachfrom \cup \tt \reachfromat{\inat{\frag}}$, 
\item $\tt \reachfromat{\outat{\frag'}} = v_x.i.reachfrom \cup \tt \reachfromat{\outat{\frag}}$, 
\end{itemize}
\item add $\tt v'$ to $\tt S_{post}$,
\item add $\tt (v,v')$ to $\tt R_2$
\end{itemize}
\item for each fragment $\tt v \in S_1$ where $\tt v_x \overset{**}\leftrightarrow_S v$ and either $\tt v \hookrightarrow_S v_y$ or $\tt v_y \overset{*o}\leftrightarrow_S v$, 
\begin{itemize}
\item  create $\tt v'$ which is same as $\tt v$ except that $\tt \reachfromat{\outat{\frag'}} = v_x.i.reachfrom \cup \tt \reachfromat{\outat{\frag}}$, 
 \item add $\tt v'$ to $\tt S_{post}$, 
 \item add $\tt (v,v')$ to $\tt R_3$

\end{itemize}


\item for each fragment $\tt v \in S_1$ where $\tt v_x \overset{*}\hookrightarrow_S v$ and either $\tt v \hookrightarrow_S v_y$ or $\tt v_y \overset{*o}\leftrightarrow_S v$,  
\begin{itemize}
\item create $\tt v'$ which is same as $\tt v$ then except that $\tt \reachfromat{\outat{\frag'}} = v_x.i.reachfrom \cup \tt \reachfromat{\outat{\frag}}$, 
\item for each subset $\tt regs$ of observer registers in $\reachfromat{\inat{\frag'}} \cap \tt v_x.i.reachfrom$
\begin{itemize}
\item create $\tt v''$ which is same as $\tt v'$, except that $\tt v''.i.reachfrom = (\reachfromat{\inat{\frag'}} \setminus v_x.i.reachfrom) \cup regs$.
\item add $\tt v''$ to $\tt S_{post}$,
\item add $\tt (v,v'')$ to $\tt R_4$
\end{itemize}
\end{itemize}

\item for each fragment $\tt v \in S_1$ where $\tt v_x \overset{**}\leftrightarrow_S v$ and either $\tt v \overset{+}\hookrightarrow_S v_y$ or $\tt v \overset{*+}\leftrightarrow_S v_y$, 
\begin{itemize}
\item create $\tt v'$ which is same as $\tt v$
\item  add $\tt v'$ to $\tt S_{post}$,
\item add $\tt (v,v')$ to $\tt R_5$

\end{itemize}


\item for each fragment $\tt v \in S_1$ where $\tt v_x \overset{*}\hookrightarrow_S v$ and either $\tt v \overset{+}\hookrightarrow_S v_y$ or $\tt v_y \overset{*+}\leftrightarrow v$, then for each subset $\tt regs$ of observer registers in $\tt \reachfromat{\inat{\frag}} \cap \tt v_x.i.reachfrom$, 
\begin{itemize}
\item create $\tt v'$ which is same as $\tt v$, except that $\reachfromat{\inat{\frag'}} = (\tt \reachfromat{\inat{\frag}} \setminus v_x.i.reachfrom) \cup regs$. 
 \item for each set $\tt regs'$ of observer registers in $\reachfromat{\outat{\frag'}} \cap \tt v_x.i.reachfrom$,  
\begin{itemize}
\item create $\tt v''$ which is same as $\tt v'$, except that $\tt v''.o.reachfrom = (\reachfromat{\outat{\frag'}} \setminus v_x.i.reachfrom) \cup regs'$. 
\item add $\tt v''$ to $\tt S_{post}$
\item add $\tt (v,v'')$ to $\tt R_6$
\end{itemize}
\end{itemize}
\item add $\tt v_{new}$ to $\tt S_{post}$ 
\end{enumerate}

\item for each fragment $\tt v \in S_2$ then we do as follows. for each $\tt (v_1,v_1') \in R_i$, $\tt (v_2,v_2') \in R_j$ where $\tt i \neq j$ and $\tt 1 \leq i,j \leq 6$. For each pair of indices $\tt i_1$, $\tt i_2$ such that $\tt i_1 \in \{v_1.i, v_1.o\}$, $\tt i_2 \in \{v_2.i, v_2.o\}$, $\tt tag(v,i) = tag(v_1, i_1)$, $\tt tag(v,o) = tag(v_2, i_2)$. Create $\tt v'$ which is same as $\tt v$ except that 
\begin{itemize}
\item $\pvarsat{\inat{\frag'}} = v_1'.i_1.vars$,
\item $\pvarsat{\outat{\frag'}} = v_2'.i_2.vars$,

\item $\reachfromat{\inat{\frag'}} = v_1'.i_1.reachfrom$,
\item $\reachfromat{\outat{\frag'}} = v_2'.i_2.reachfrom$,
\item $\reachtoat{\inat{\frag'}} = v_1'.i_1.reachto$,
\item $\reachtoat{\outat{\frag'}} = v_2'.i_2.reachto$,
\end{itemize} then add $\tt v'$ to $\tt S_{post}$. 


\newpage
 \end{itemize}
 
 \paragraph{Fragment Intersection:} Let us describe the intersection of two fragments $\tt v_1 \in S_1$ and $\tt v_2 \in S_2$ denoted as $\tt v_1 \sqcap v_2$. Firstly, we consider the case where both $\tt v_1$ and $\tt v_2$ have size 2.
 \begin{itemize}
 \item if $\tt v_1.greachfrom(i,\{1,2\}) \neq \emptyset$ and $\tt v_2.greachfrom(i,\{1,2\}) \neq \emptyset$ then 
  \begin{itemize}
  \item if $\tt gtag(v_1, i) = gtag(v_2, i)$ and $\tt gtag(v_1, o) = gtag(v_2, o)$ then $\tt v_1 \sqcap v_2 = \{v_{12}\}$ where $\tt v_{12}$ is same as $\tt v_1$ except that, for all $\tt l \in \{1,2 \{1,2\}\}$ 
  \begin{itemize}
  \item $\tt v_{12}.vars(i) = v_1.vars(i) \cup v_2.vars(i)$ 
  \item  $\tt v_{12}.vars(o) = v_1.vars(o) \cup v_2.vars(o)$
  \item $\tt v_{12}.reachfrom(i,l) = v_{1}.reachfrom(i,l) \cup v_{2}.reachfrom(i,l)$
   \item $\tt v_{12}.reachfrom(i,l) = v_{1}.reachfrom(o,l) \cup v_{2}.reachfrom(o,l)$
    \end{itemize}
 \end{itemize}

 \item if $\tt v_1.greachfrom(i,\{1,2\}) = \emptyset$, $\tt v_2.greachfrom(i,\{1,2\}) = \emptyset$, $\tt v_1.greachfrom(o,\{1,2\}) \neq \emptyset$ and $\tt v_2.greachfrom(o,\{1,2\}) \neq \emptyset$ then 
  \begin{itemize}
  \item if $\tt gtag(v_1, o) = gtag(v_2, o)$, $\tt v_1.private.(i) = false$ and $\tt v_2.private.(i) = false$ then $\tt v_1 \sqcap v_2 = \{v_1', v_2', v_{12}\}$ where %$\tt v_{12}$ is same as $\tt v_1$ except that 
  $\tt v_1'$ is same as $\tt v_1$ except that, for all $\tt l \in \{1,2 \{1,2\}\}$
    \begin{itemize}
  \item  $\tt v_1'.vars(o) = v_1.vars(o) \cup v_2.vars(o)$
  \item $\tt v_1'.reachfrom(o,l) = v_{1}.reachfrom(o,l) \cup v_{2}.reachfrom(o,l)$
    \end{itemize}
   and $\tt v_2'$ is same as $\tt v_2$ except that 
    \begin{itemize}
  \item  $\tt v_2'.vars(o) = v_1.vars(o) \cup v_2.vars(o)$
  \item $\tt v_2'.reachfrom(o,l) = v_{1}.reachfrom(o,l) \cup v_{2}.reachfrom(o,l)$
    \end{itemize}
     \item if $\tt gtag(v_1, o) = gtag(v_2, o)$ and ($\tt v_1.private.(i) = true$ or $\tt v_2.private.(i) = true$) then $\tt v_1 \sqcap v_2 = \{v_1', v_2'\}$ 
 \end{itemize}
 
 
  \item if $\tt v_1.greachfrom(i,\{1,2\}) = \emptyset$, $\tt v_2.greachfrom(i,\{1,2\}) = \emptyset$, $\tt v_1.greachfrom(o,\{1,2\}) = \emptyset$ and $\tt v_2.greachfrom(o,\{1,2\}) = \emptyset$ then 
 \begin{itemize}
 
 
 \item if $\tt gtag(v_1, o) = gtag(v_2, o)$, $\tt v_1.private.(i) = false$, $\tt v_1.private.(o) = false$, $\tt v_1.private.(o) = false$ and $\tt v_2.private.(o) = false$ then  $\tt v_1 \sqcap v_2 = \{v_1,v_2,v_1',v_2',v_{12}\}$ %where $\tt v_{12}$ is same as $\tt v_1$ except that 
    \item if $\tt gtag(v_1, o) = gtag(v_2, o)$, ($\tt v_1.private.(i) = true$ or $\tt v_2.private.(i) = true$) and $\tt v_1.private.(o) = false$ and $\tt v_2.private.(o) = false$ then  $\tt v_1 \sqcap v_2 = \{v_1,v_2,v_1',v_2'\}$  

  \item if $\tt gtag(v_1, o) = gtag(v_2, o)$ and ($\tt v_1.private.(o) = true$ or $\tt v_1.private.(o) = true$) then  $\tt v_1 \sqcap v_2 = \{v_1,v_2\}$ 

    \item if $\tt gtag(v_1, o) \neq gtag(v_2, o)$ then $\tt v_1 \sqcap v_2 = \{v_1,v_2\}$
 \end{itemize}
 
 \end{itemize}



%
