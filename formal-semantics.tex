\subsection{Formal Semantics}
\label{semantics:section}
For preciseness, we formalize the semantics of programs and
observers. Below, we assume a program $\prog$  with a set 
$\glvarset$ of global variables, and an ordering constraint
specified by an observer $\observer$.
%
%% We assume that each cell contains a set $\fieldset$ of fields
%% where each field is either a lock or a data field.
We assume that each thread $\thread$ executes one method
denoted $\methodof\thread$.
%% First, we define the state of the heap and the transition
%% relation induced by a single thread.
%% %
%% From this we derive the semantics of the program and use it to
%% define its history set.
%% %
%% We then define the semantics of observers.
%% Finally, we define the product of the augmented program and the observer.
%% This product will ensure that observer
%% On this basis, we can formally state and prove the correctness of
%% our approach, as Theorem~\ref{thm:soundness}.
%% Finally, we introduce  the soundness theorem 
%% which states that linearizability of the original program
%% is implied by the fact that the monitor never enters
%% its ``error'' state.

For a function $\fun:A\mapsto B$ from a set $A$ to
a set $B$, we use 
$\fun[\aelem_1\assigned\belem_1,\ldots,\aelem_\nn\assigned\belem_\nn]$ 
to denote the function
$\fun'$ such that $\fun'(\aelem_\ii)=\belem_\ii$ and 
$\fun'(\aelem)=\fun(\aelem)$ if 
$\aelem\not\in\set{\aelem_1,\ldots,\aelem_\nn}$.
%

%


\paragraph{Heaps.}
A {\it heap (state)} is a tuple $\heap=\heaptuple$, where
(i)
$\cellset$ is a finite set of cells, including the two special cells
$\nullconst$ and $\dangconst$ (dangling);
%
we define $\mcellset=\cellset\setminus\set{\nullconst,\dangconst}$,
%
(ii)
for each pointer field $\pfield_i$ there is a total function
$\pfield_i:\mcellset\to\cellset$
that defines where that pointer field points,
%% the next-pointer relation, i.e., $\priflesuccof{\cell}=\cell'$ means that cell $\cell$ points to $\cell'$,
(iii)
$\glval:\glvarset\to\cellset$ maps the global (pointer)
variables to their values, and
(iv)
$\cellval:\cellset\times\fieldset\to\fset\cup\ddomain$
maps data and lock fields of each cell to their values.
%
We let $\initheap$ denote the initial heap produced by the
${\tt init}$ method.

\paragraph{Threads.}
A {\it local state} $\lstate$ of a thread $\thread$
wrt.\ a heap $\heap$ defines 
the values of its local variables, including the program counter
${\tt pc}$ and the input parameter for the method executed by $\thread$.
In addition, there is the special initial state $\idlestate$,
and terminated state $\terminatedstate$.
%
%% A {\em projection} $\proj$ is a pair
%% $\tuple{\lstate,\heap}$, where $\lstate$
%% is a local state wrt.\  the heap $\heap$.
%
We formalize the behavior of a thread $\thread$ by
a labeled transition relation $\movesto\thread{}$
on pairs $\tuple{\lstate,\heap}$ consisting of a local state
$\lstate$ and a heap $\heap$, with three types of transitions:
%
\begin{enumerate}
\item
$\tuple{\idlestate,\heap}
\movesto\thread{{\call\thread\mname\indata}}
\tuple{\initlstateof\thread,\heap}$ activates
thread $\thread$ through a transition
labeled by a call action with $\thread$ as the action identifier and
$\mname=\methodof\thread$, taking $\thread$
to an initial local state $\initlstateof\thread$
where $\indata$ is the value of its input parameter,
the value of ${\tt pc}$ is the label of the first statement of the method, and
the other local variables are undefined.
\item
$\tuple{\lstate,\heap}\movesto\thread{}\tuple{\lstate',\heap'}$
  denotes execution of method statements, which are unlabeled; these are defined
  in the standard way for each statement form.
%
%% We write $\proj\movesto\thread\lbl\proj'$ 
%% to denote that the statement labeled by $\lbl$ can be executed
%% from $\proj$, yielding $\proj'$.
%% Note that the next move of $\thread$ is uniquely determined
%% by $\proj$, since $\thread$ cannot access
%% the local states of other threads.
%
\item 
$\tuple{\lstate,\heap}\movesto\thread{\return\thread\mname\outdata}
\tuple{\terminatedstate,\heap}$ terminates
thread $\thread$ through execution of its ${\tt return}$ command, labeled
by a return action with $\thread$ as action identifier,
$\mname=\methodof\thread$, and $\outdata$
as the returned value.
\end{enumerate}


\paragraph{Programs.}
A {\it configuration} of a program
$\prog$ is a tuple
$\tuple{\athreads,\lstatemapping,\heap}$ where
$\athreads$ is a set of threads,
$\heap$ is a heap, and
$\lstatemapping$
%% is a {\it local state mapping} over $\athreads$ wrt.\ $\heap$ that
maps each thread $\thread\in\athreads$ to its
local state $\lstatemappingof\thread$ wrt.\ $\heap$.
%
%% We use $\confsetof\prog$ to denote the set of configurations
%% of $\prog$.
%
The initial configuration
%% $\initconfof\prog\in\confsetof\prog$ 
$\initconfof\prog$ 
is the pair
$\tuple{\initlstatemapping,\initheap}$,
where
%% $\initheap$ is the initial heap, and
$\initlstatemappingof\thread=\idlestate$ for each
$\thread\in\athreads$,
%% i.e.,
%% $\prog$ starts its execution from a configuration with an initial
%% heap, and with each thread in its initial local state.
%
A program $\prog$ induces a transition relation
$\movesto\prog{}$ where each step corresponds to one move of a single thread.
I.e., 
there is a transition of form
$\tuple{\athreads,\lstatemapping,\heap}
\movesto\prog{\emptyword}
\tuple{\athreads,\lstatemapping[\thread\leftarrow\lstate'],\heap'}$
whenever the transition relation $\movesto\thread{}$ has a transition
$\tuple{\lstate,\heap}\movesto\thread\lbl\tuple{\lstate',\heap'}$,
where the label $\lbl$ is either a call or return action or the empty label.
%
Note that the only visible transitions are those
corresponding to call and return actions.

\paragraph{Cross-Product of Program and Observer}
We use $\system=\prog\compose\observer$ to denote the cross-product obtained by
running $\prog$ and $\observer$ together.
%
The initial configuration of $\system$ is
$\tuple{\initconfof\prog,\oinitstate}$.
Transitions of $\system$ are of the form
$\tuple{\pconf,\ostate},\movesto{\system}{},\tuple{{\pconf}',\ostate'}$,
obtained from a transition
$\pconf \movesto\prog{l} {\pconf}'$ of the program with some (possibly empty)
label $l$, and
a corresponding synchronizing transition
$\tuple{\ostate,\returntwo{\mname}{l},\ostate'}$
of the monitor.
The verification problem is now to check that $\system$ cannot reach
a configuration $\tuple{\pconf,\oaccstate}$ with an accepting
observer state.

%% A {\it history} of $\prog$ is the sequence of call and return actions in
%% some sequence of transitions of $\movesto\prog{}$ from the initial configuration.
%% We define $\historyof\prog$ to be the set of histories of $\prog$.


