\section{Fragment Abstraction for Arrays of Singly-Linked Lists}
\label{sec:list-arrays}
In this section, we describe the application of our fragment abstraction to concurrent programs that operate on arrays of singly-linked lists. This structure
was introduced already in the motivating example in
Section~\ref{sec:overview}.
In this setting, the algorithm uses an array of singly-linked lists, accessed via the thread-indexed array ${\tt pools[maxThreads]}$ of pointers to the first cell of each list. In these algorithm, each cell also contains a timestamp value.

\subsection{Fragment Abstraction}
We verify these algorithm by extending the fragment abstraction for SLLs in
two ways:
\begin{itemize}
\item   To handle timestamps, we extend the data abstraction. Each timestamp value is abstracted to a mapping from local timestamp variables and observer
  registers to the set $\set{<,=,>}$.
  An element $\tsabsmap$ in the abstract domain represents a concrete timetamp value $\tt ts$ if the value $\tt ts'$ of any local timestamp variable
  $\lvarof{tsvar}$ satisfies $\tt ts \sim ts'$ for some $\sim \in
  \tsabsmap(\lvarof{tsvar})$ and the value $\tt ts'$ of the timetamp field
  of any $\reg_i$-cell satisfies $\tt ts \sim ts'$ for some $\sim \in
  \tsabsmap(\reg_i)$.
\item
  We abstract the unbounded set of indices $\tt i$ in arrays of form
  $\tt pools[maxThreads]$ into the set $\set{\mathit{me},\mathit{other}}$, where
  for a thread $\thread$, we let $\mathit{me}$ represent the index
  $\tt i$ such that $\thread$ is currently manipulating the list pointed to
  by $\tt pools[i]$, and $\mathit{other}$ any other index.
\end{itemize}

\subsection{Symbolic Postcondition Computation}
The symbolic postcondition computation is quite similar to that for singly-linked lists with the following differences
\todo[inline]{to Quy: can you rephrase the rest of this paragraph? I could not understand it}
For the pop method, before computing the post condition of the statement which accesses to $\tt pools[ot]$. we have to swap values of $\tt ot$ and $\tt me$ because the current list after the post condition now is the list pointed by $\tt pools[ot]$.  

\paragraph{Interference Steps} 
The intersection between two views are computed same as in the case of singly-linked lists with the difference that two views of push methods should not be intersected. The reason for it is that we do not have more concurrent pushes in a same list.








