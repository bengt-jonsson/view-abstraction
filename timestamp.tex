\section{Arrays of Singly-Linked Lists with Timestamps}
\label{sec:ts}

In this section, we show how to apply fragment abstraction to concurrent
programs that  operate on a shared heap which represents
an array of singly linked lists.
We use this abstraction to provide the first automated verification of
linearizability for the Timedstamped stack and
Timestamped queues algorithms of~\cite{ts-stack}. 
as reported in Section~\ref{section:experiments}.


\input tscode

%\todo[inline]{Provide a description of the algorithm}
%\todo[inline]{Quy: The line numbers for Pop are scrambled. Can you fix?
%Also, the variable {\tt i} at line 6 is initialized strangely. Can you fix?}
Figure~\ref{ts-stack:fig} shows a simplified version of the Timestamped Stack (TS stack) of~\cite{ts-stack}, where we have omitted the check for emptiness in the ${\tt pop}$ method, and the optimization using ${\tt push}$-${\tt pop}$ elimination. These features are included in the full version of the algorithm, described in Appendix~\ref{app:codes}, that we have verified automatically.

The algorithm uses an array of singly-linked lists (SLLs), one for each thread, accessed via the thread-indexed array ${\tt pools[maxThreads]}$ of pointers to the first cell of each list. The ${\tt init}$ method initializes each of these pointers to $\nullconst$. Each list cell contains a data value, a timestamp value, a ${\tt next}$ pointer, and a boolean flag ${\tt mark}$ which indicates whether the node is logically removed from the stack. Each thread pushes elements only to ``its own'' list, but can pop elements from any list.

A ${\tt push}$ method for inserting a data element ${\tt d}$ works as follows: first, a new cell with element ${\tt d}$ and minimal timestamp ${\tt -1}$ is inserted at the beginning of the list indexed by the calling thread (line 1-3). After that, a new timestamp is created and assigned (via the variable ${\tt t}$) to the ${\tt ts}$ field of the inserted cell (line 4-5).
%\bjcom{Quesion to Quy: What happens when pop methods sees uninitialized timestamps?} \quycom{uninitialized time is -1, its smaller than any other time.}
Finally, the method unlinks (i.e., physically removes) all cells that are reachable (through a sequence of $\tt next$ pointers) from the inserted cell and whose ${\tt mark}$ field is ${\tt true}$; these cells are already logically removed. This is done by redirecting the $\tt next$ pointer of the inserted cell to the first cell with a $\false$ $\tt mark$ field, which is
reachable from the inserted cell.

A ${\tt pop}$ method first traverses all lists, finding in each list
the first cell whose ${\tt mark}$ field is ${\tt false}$(line 8), and letting the variable ${\tt youngest}$ point to the most recent such cell
(i.e., with the largest timestamp) (line 1-11).
A compare-and-swap (CAS) is used
to set the ${\tt mark}$ field of this youngest cell to $\true$,
thereby logically removing it.
This procedure will restart if the CAS fails. After the youngest cell has been removed, the method will unlink all cells, whose ${\tt mark}$ field is ${\tt true}$,
that appear before (line 17-19) or after (line 20-23) the removed cell.
Finally, the method returns the ${\tt data}$ value of the removed cell.
%\todo[inline]{Comments by Bengt: How can you return empty? Also, I think you
% need to explain the extra removal of marked cells better} \quycom{this code is a simplified version without emptiness checking. If we add emptiness part, the code will be longer and more complicated}

\subsubsection{Fragment Abstraction}
In our verification, we establish that the TS stack algorithm of
Figure~\ref{ts-stack:fig} is is correct in the sense that it is a
linearizable implementation of a stack data structure.
For stacks and queues, we specify linearizability by 
 observers that synchronize on call and return actions of
  methods, as shown by~\cite{BEEH:icalp15}; this is done without
 any user annotation.

 The verification is performed analogously as for skiplists, as described
 in Section~\ref{sec:fragment-abstraction}. Here we show how fragment
 abstraction is used for arrays of singly-linked lists.
Figure~\ref{fig:tsshape} shows an  example heap state of TS stack.
The heap consists of a set of singly linked lists (SLLs), each of which
is accessed from a pointer in the array ${\tt pools[maxThreads]}$
in a configuration when % Quy write from here
it is accessed concurrently by three threads $\thread_1$, $\thread_2$, and $\thread_3$. The heap consists of three SLLs accessed from the three pointers $\tt pools[1]$, $\tt pools[2]$, and $\tt pools[3]$ respectively. Each heap cell is
shown with the values of its fields, using the layout shown to the right in
Figure~\ref{fig:tsshape}.
In addition, each cell is labeled by the pointer variables that point to it.
We use ${\tt lvar(i)}$ to denote the local
variable ${\tt lvar}$ of thread $\thread_i$.
%\todo[inline]{Quy: Use a better scheme for indexing local variables. Previously,
%  we used ${\tt var(2]}$, for variable ${\tt var}$ of thread 2. Maybe use it also here?}
%\todo[inline]{You must start by explaining the notation better. E.g., what means
%  a pointer variable that labels a node?}

In the heap state of Figure~\ref{fig:tsshape},
thread $\thread_1$ is trying to push a new node with data value $4$, pointed by its local variable $\tt new$, having reached line 3.
Thread $\thread_3$ has just called the ${\tt push}$ method.
Thread $\thread_2$ has reached line 12 in the execution of the ${\tt pop}$ method,  and has just assigned ${\tt youngest}$ to the first node in the list
pointed to by $\tt pools[3]$ which is not logically removed (in this case it is the last node of that list).
%% The local state of thread $\thread_2$
%% consists of the values of its program counter, the boolean variable ${\tt success}$, the pointer variables ${\tt youngest}$, ${\tt myTop}$, and ${\tt n}$,
%% and the timestamp variable ${\tt maxTS}$.
%% Not shown in Figure~\ref{fig:tsshape} is the configuration of the observer.
%% The cross-product of program and observer also includes the configuration of the
%% observer. also In verification, we must also consider the observer.
%% In this case, the observer is the one in Figure~\ref{fig:lifostack:fig},
The observer has two registers
$\tt x_1$ and $\tt x_2$, which are assigned the values $4$ and $2$,
respectively.

\begin{figure}
\center
	\input tsshape
\caption{A possible heap state of TS stack with three threads.}
\label{fig:tsshape}
\end{figure} 

We verify the algorithm using a symbolic representation that is analogous
to the one used for skiplists. There are two main differences.
\begin{itemize}
\item
  Since the array $\tt pools$ is global, all threads can reach all lists in
  the heap (the only cells that cannot be reached by all threads are
  new cells that are not yet inserted).
\item
  We therefore represent the view of a thread by a thread-dependent abstraction
  of thread indices, which index the array $\tt pools$. In the view of
  a thread, the index of the list where it is currently active
  is abstracted to $\tt me$, and all other
  indices are abstracted to $\tt ot$. The currently active index is taken to
  be the index the list to which $\tt myTop$ points.
\bjcom{Quy: can you make this text precise}
\end{itemize}
In the definition of tags, the only global variables that can occur in
the fields $\reachfrom$ and $\reachto$ are therefore
$\tt pools[me]$ and $\tt pools[other]$. The data abstraction repesents
ordering information between timestamps, including to represent the ordering
of each timestamp with the timestamp of a cell that has the same data value
as an observer register.

\begin{figure}
\center
	\input tsabsshape
\caption{Fragment abstraction}
\label{fig:tsviewshape}
\end{figure} 

Figure~\ref{fig:tsviewshape} shows a set of fragments that is
satisfied wrp.\ to $\thread_2$ by the configuration in Figure~\ref{fig:tsshape}.
There are $7$ fragments, named $\frag_1$, \ldots , $\frag_7$.
%% Two of
%% these ($\frag_3$ and $\frag_7$) consist of a tag that points to $\dangconst$,
%% and the other consist of a pair of pointer-connected tags.
Consider the tag which occurs in fragment $\frag_7$.
%% ; it is the bottom-rightmost tag.
This tag is an abstraction of the
bottom-rightmost heap cell in Figure~\ref{fig:tsshape},
%% using the same layout for fields as Figure~\ref{fig:tsshape}.
The different non-pointer fields are represented as follows.
\begin{itemize}
\item The $\tt data$ field of the tag (to the left) abstracts the data value
  $2$ to the set of observer registers with that value: in this case
  $\reg_2$.
\item The $\tt ts$ field (at the top) abstracts the timer value $15$ to
  the possible relations with $\tt ts$-fields of heap cells with the same
  data value as each observer registers. Recall that observer registers
  $\reg_1$ and $\reg_2$ have values $4$ and $2$, respectively. There are
  three heap cells with $\tt data$ field value $4$, all with a $\tt ts$
  value less than $15$. There is one heap cell with
  $\tt data$ field value $2$, having $\tt ts$ value $15$.
  Consequently, the abstraction of the $\tt ts$ field maps $\reg_1$ to
  $\set{>}$ and $\reg_2$ to $\set{=}$: this is shown as the mapping
  $\lambda_4$ in Figure~\ref{fig:tsviewshape}.
\item The $\tt mark$ field assumes values from a small finite domain and
  is represented precisely as in concrete heap cells
\end{itemize}
%% Above the top, the tag contains the thread-local and global pointer variables
%% that point to the cell, in this case $\tt youngest$ and $\tt n$.
%% At the bottom of the tag, the first row contains the global variables 
%% pointing to cells from which the cell can be reached, in this case
%% $\tt pools[3]$, as well as observer registers whose value is equal to the
%% ${\tt data}$ field of a cell from which  the cell can be reached, in this
%% case $\reg_2$ (since the cell itself has the same data value as $\reg_2$).
%% The second row contains dual information: now for cells that can be reached
%% from the cell itself (this is again $\reg_2$).

%% \todo[inline]{Fix the issue with ${\tt pools[3]}$}

%% Each cell in the heap state of Figure~\ref{fig:tsshape} now satisfies
%% some tag in Figure~\ref{fig:tsshape}. Moreoever, each pair of pointer-connected
%% cells (where the pointed-to ``cell'' can also be $\dangconst$)
%% satisfies some fragment in Figure~\ref{fig:tsshape} in the obvious way.
%% Conversely, the set of fragments in
%% Figure~\ref{fig:tsshape} represents the set of heaps in which each pair of
%% pointer-connected cells satisfies one of its fragments. For instance, the
%% list pointed to by $\tt pools[3]$ is represented by the sequence of
%% fragments $\frag_4 \frag_5 \frag_6 \frag_7$.


%% \subsection{Fragment Abstraction}
%% We verify these algorithm by extending the fragment abstraction for SLLs in
%% two ways:
%% \begin{itemize}
%% \item   To handle timestamps, we extend the data abstraction. Each timestamp value is abstracted to a mapping from local timestamp variables and observer
%%   registers to the set $\set{<,=,>}$.
%%   An element $\tsabsmap$ in the abstract domain represents a concrete timetamp value $\tt ts$ if the value $\tt ts'$ of any local timestamp variable
%%   $\lvarof{tsvar}$ satisfies $\tt ts \sim ts'$ for some $\sim \in
%%   \tsabsmap(\lvarof{tsvar})$ and the value $\tt ts'$ of the timetamp field
%%   of any $\reg_i$-cell satisfies $\tt ts \sim ts'$ for some $\sim \in
%%   \tsabsmap(\reg_i)$.
%% \item
%%   We abstract the unbounded set of indices $\tt i$ in arrays of form
%%   $\tt pools[maxThreads]$ into the set $\set{\mathit{me},\mathit{other}}$, where
%%   for a thread $\thread$, we let $\mathit{me}$ represent the index
%%   $\tt i$ such that $\thread$ is currently manipulating the list pointed to
%%   by $\tt pools[i]$, and $\mathit{other}$ any other index.
%% \end{itemize}

\subsubsection{Symbolic Postcondition Computation}
The symbolic postcondition computation is similar to that for skiplist.
Main differences are as follows.
\begin{itemize}
\item
  Whenever a thread assigns to $\tt myTop$, the abstraction must consider
  to swap between the abstractions $\tt me$ and $\tt ot$.
  \item
    In interference steps, we must consider that the abstraction
    $\tt me$ for the interfering thread may have to be changed into $\tt ot$.
    Furthermoe, the abstractions $\tt me$ for two $\tt push$ methods
    cannot coincide, since each thread pushes only to its own list.
\end{itemize}

%% \section{Timestamp Abstraction}
%% \subsection{View Abstraction}
%% For timestamp data structures we have to deal with timestamp ordering and unbound number of lists. The solutions are described as follows:
%% \begin{itemize}
%% 	\item  About timestamp ordering, we add timestamp ordering information for each index of a view. The order of index $\tt i$ of view $\tt v$ is of the form $\tt v.i.ts \diamond x$ where $\diamond\in \set{<,=,>}$ and $\tt x$ is an observer register. Intuitively, $\tt v.i.ts \diamond x$ means that $\diamond$ is the order between the timestamp of $\tt v.i$ and timestamp of an index whose data is equal to $\tt x$. 
%% 	\item To deal with the problem of unbounded number of lists. We use two kind of views which are c-views $\tt v_c$ in a current list and o-views $\tt v_o$ in other lists. Note that, current list is the list where the current thread is accessing to.
%% \end{itemize}
%% \subsection{Post-computation}
%% The post-computation is quite similar to singly-linked lists with several differences as follows: Before computing the post condition of this statement, we change all o-views in the other list to c-views and previous c-views to o-views.  
%% \subsection{Intersection}
%% The intersection between two views are computed same as in the case of singly-linked lists with several differences as follows: 
%% \begin{itemize}
%% 	\item Two views of push methods should not be intersected. The reason for it is that we do not have more concurrent pushes in a same list.
%% 	\item We can intersect o-views and c-views, o-views and o-views as well as c-views and c-views
%% \end{itemize}








