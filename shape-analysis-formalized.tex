\section{Symbolic Shape and Data Abstraction: Singly Linked Lists}
\label{sec:annotations}
This section describes how we handle an
unbounded heap and data domain in
thread-abstracted configurations.
We assume that each heap cell contains exactly one $\intgrs$-field.
%More detailed descriptions of representations and postconditions are described in
%Appendix~\ref{sec:shape-details}.

The abstraction of thread-abstracted configurations we use is a variation of a
  well-known abstraction of singly-linked lists ~\cite{MYRS:Canonical}. We explicitly represent only a finite subset of the heap cells, called
the {\em relevant} cells. The remaining
cells are summarized into linked-list segments of arbitrary length.
Thus, we intuitively abstract the heap into a graph-like  structure, whose
nodes are relevant cells, and whose edges are the connecting segments.
More precisely, 
a cell $\cell$ (in a thread-abstracted configuration) is {\em relevant}
%% (wrt.\ $\tuple{\taconf,\private}$)
if it is either
\begin{inparaenum}[(i)]
\item
pointed to by a (global or local) pointer variable, or
\item
the value of
${\cell_1.{\tt next}}$ and  ${\cell_2.{\tt next}}$ for
two different cells $\cell_1$, $\cell_2 \in \cellset$,
one of which is reachable from some global variable, or
\item the value of ${\cell'.{\tt next}}$ for some cell $\cell' \in \cellset$
  such that $\private(\cell')$ but $\neg \private(\cell)$.
\end{inparaenum}
Each relevant cell is connected, by a linked-list segment, to a uniquely
defined next relevant cell.
%% (which can also be ${\tt null}$ or undefined).
For instance, the thread abstracted
heap in Figure~\ref{fig:llshapes}(b) has $5$ relevant
cells: $4$ cells that are pointed to by variables
(${\tt Head}$, ${\tt Tail}$, ${\tt p[1]}$, and ${\tt c[1]}$) and the cell where
the lower list segment meets the upper one (the second cell in the top row).
The corresponding symbolic representation, shown 
in Figure~\ref{fig:llshapes}(c), contains only the relevant cells, and
connects each relevant cell to its successor relevant cell. 
Note that the cell containing ${\tt 4}$ is not relevant,
%% although it is pointed to from two other cells,
since it is not globally reachable.
%% Consequently, the next relevant cell of the two relevant cells in the
%% bottom row is the globally reachable cell in the top row.
Consequently,
%% In other words,
we do not represent the precise branching structure among cells that
are not globally reachable. This is an optimization to limit the
size of our symbolic representation.

Our symbolic representation must now represent
(A) data variables and data fields of relevant cells, and
(B) data fields of list segments.
For (A) we do as follows.
\begin{enumerate}
\item
Each  data and lock variable, and each data and lock field of each relevant cell,
  is mapped to a domain which depends on the set of values it can assume:
  \begin{inparaenum}[(i)]
  \item
  variables and fields that assume values in $\fset$ are mapped to $\fset$,
\item
  lock variables and fields are mapped to
  $\set{\thread_1,\mathit{other},\mathit{free}}$,
  representing whether the lock was last acquired by $\thread_1$, by
  another thread, or is free,
\item
  variables and fields that assume values in $\intgrs$ are mapped to 
  $[\ovarset \mapsto{\set{=,\neq}}]$, representing for
  each observer register $x  \in \ovarset$, whether or not the value of the variable
  or field equals the value of $x$.
  \end{inparaenum}
\item each pair of variables or fields that assume values in $\intgrs$
  is mapped to a subset of ${\set{<,=,>}}$, representing the
  set of possible relations between them.=
\end{enumerate}
For (B), we define a domain of {\em path expressions}, which
capture the following information about a segment.
\begin{itemize}
\item
  The possible relations between adjacent $\intgrs$-fields in the heap segment,
  represented as a subset of ${\set{<,=,>}}$,
e.g., to represent that the segment is sorted.
\item
  The set of observer registers whose value appears in some $\intgrs$-field
  of the segment. For these
  registers, the path expression provides
  \begin{inparaenum}[(i)]
  \item
    the order in which the first occurrence of their values appear
    in the segment; this allows, e.g., to check
ordering properties between data values for stacks and queues, and
\item
  whether there is exactly one,
  or more than one occurrence of their value.
  \end{inparaenum}
\item
For each data and lock field, 
the set of possible values of that field in the heap segment, represented
as a subset of the domain defined in case 1 above.
This subset is provided both
for the cells whose $\intgrs$-field is not equal to any
observer register, and for the cells whose $\intgrs$-field is
equal to each observer register.
\end{itemize}
To illustrate, in
Figure~\ref{fig:llshapes}(c), each heap segment is labeled by
a representation of its path expression. The first component of
each path expression is $\set{<}$, i.e., the segment
is sorted. The path expression at the top right has as its second component
the sequence $x^1$, expressing that there is exactly one cell whose
$\intgrs$-field has the same value as the observer register $x$.
The row $x:[\set{\mbox{\tick}},\set{\mbox{\cross}}]$ expresses
that the cells whose $\intgrs$-value is equal to that of $x$, have
their ${\tt mark}$ field ${\tt true}$, and their ${\tt lock}$ field ${\tt free}$.
The row $\neq:[\set{\mbox{\cross}},\set{\mbox{\cross}}]$ expresses
that the other cells have
their ${\tt mark}$ field ${\tt false}$, and their ${\tt lock}$ field ${\tt free}$.
The two lower path expressions express that no cell has a value equal to
that of $x$, and summarize the possible values of fields.
  

The symbolic representation combines the above described shape and data
representations,  representing thread-abstracted configurations by
{\em symbolic configurations}.
%% In symbolic configurations, we use
%% a set of {\em indices} to represent the relevant cells of a
%% thread-abstract configuration.
%% Let a {\em data term} be either a data variable (of the thread or
%% the monitor), or field of a relevant cell, represented as a term
%% of form $i.\field$ for an index $i$ (representing a relevant cell)
%% and non-pointer field $\field$, of form
%% $\lstatemappingof{\thread_1}(\xvar)$ for a local non-pointer variable $x$,
%% $\linstatus(\thread_1)$, or $\rvalstatus(\thread_1)$.
%% Let $\ddomainof{\dterm}$ be the domain used to represent the values of
%% data term $\dterm$ according to case 1.\ above.
%% We use the term {\em path expression} to denote a summary of the data fields of
%% a heap segment according to case 3.\ above.
These are tuples of form
%% A {\em symbolic configuration} is a tuple
%$\tuple{\indices,\globmap,\nrglobloc,\pexpglobloc,\dmap,\ordconstr,\eta}$
$\ac = \tuple{\indices, \pointervarmap, {\tt nextrel}, \pexpglobloc, \datatermmap, \datarelmap,\private}$, where
\begin{itemize}
\item
  $\indices$ is a finite set of {\em indices}, denoting the relevant cells of the heap,
\item
  $\pointervarmap$ maps each (global or local) pointer variable to an index in $\indices$,
\item ${\tt nextrel}$ maps each index $\ii$
  to $\indices \cup \set{\nullconst,\bot}$,
\item
  $\pexpglobloc$ maps each index $\ii$ to
  a path expression;
  intuitively, if the index $\ii$ represents the relevant cell $\cell$, then
${\tt nextrel}(i)$ represents the next relevant cell, and
  $\pexpglobloc(\ii)$ summarizes  the
  heap segment between $\cell$ and the cell represented by ${\tt nextrel}(i)$,
\item
%%   $\datatermmap$ maps each data term $\dterm$ to an element in $\ddomainof{\dterm}$, and
  $\datatermmap$ maps data variables and fields of relevant cells to
  appropriate domains,
%%  data term $\dterm$ to an element in the appropriate in $\ddomainof{\dterm}$,
  and
  $\datarelmap$ maps pair of those that assume values in $\intgrs$ to a subset of $\set{<, =, >}$.
\item
  $\private$ is a predicate on indices.
\end{itemize}
We define
%% formalize the above intuition for symbolic configurations by defining
a satisfaction relation between thread-abstracted and
symbolic configurations. 
%% Details are in Appendix~\ref{sec:symbrepsem}.
A {\em symbolic representation} $\symbrep$ is a set of
symbolic configurations. 
A  set of thread-abstracted configurations then satisfies a symbolic
representation $\symbrep$ if each of its thread-abstracted
configuration satisfies some symbolic configuration in $\symbrep$.


%% \td{Move the following}
%% \paragraph{Example}
%% We illustrate our symbolic configurations by the example in
%% Figure~\ref{fig:llshapes}.
%% %\todo[inline]{Here, we should have an updated picture, if space permits.
%% %Below is the old text}
%% To the left is a picture of a possible
%% heap structure of the  Lazy Set algorithm of Figure~\ref{lazy:list:fig}.
%% Each cell contains the values of {\tt val}, {\tt mark}, and {\tt lock}
%% from top to bottom, where \tick denotes {\tt true}, and \cross denotes
%% {\tt false} (or {\it free} for {\tt lock}). There are three threads,
%% {\tt 1}, {\tt 2}, and {\tt 3} with
%% pointer variable {\tt p[i]} of thread {\tt i} labeling the cell that it points
%% to. The value of the observer register is assumed to be $9$. Here the
%% top $5$ cells are globally accessible, and of these all except the $4$th
%% are relevant.
%% To the right is a graphical representation of a heap expression that
%% abstracts the heap structure. The set $\indices$ is represented by the $4$
%% cells that are horizontally organized in the middle. Note that the branching
%% structure in the part of the heap that is not globally accessible is
%% abstracted by only representing how the local pointer variables
%% {\tt c[i]} and {\tt p[i]} for
%% $\mathtt{i} \in \set{\mathtt{1},\mathtt{2},\mathtt{3}}$ reach the nearest
%% relevant node. Each edge is labeled by a path expression. For instance the
%% incoming edge of the {\tt Tail} node states that the segment is sorted,
%% has exactly one {\tt val} field that equals the observer register, and no
%% marked fields or acquired locks. Data variables and data
%% fields of relevant cells should then be constrained by a data constraint,
%% which we omit here.
%% \qed

\expcloseparagraph{Symbolic Postcondition Computation}
It remains to define a symbolic post operation $\symbpost$ on symbolic
representations that reflects 
$\thpost$ on sets of thread-abstracted configurations.
%% I.e., we must define a symbolic 
%% the property that whenever 
%% $\confs \models \symbrep$, then
%% $\thpostof{\confs} \models \symbpostof{\symbrep}$.
%% Naturally, $\symbpost$ will mimic $\thpost$.
%% thread-abstracted configurations.
Given a set $\symbrep$ of symbolic
configurations representing possible views of single threads,
it first generates all ways of merging them to symbolic representations 
%% representation $\symbrep$, we construct all possible ways of consistently
%% merging 
%% $k$ symbolic configurations, representing the individual views of $k$
%% different threads, into a symbolic configuration representing the
of combined views of $k$ threads;
%% We call such a combined view a {\em symbolic $k$-configuration}.
thereafter computing their postconditions
%% symbolic $k$-configuration,
%% we thereafter compute its postcondition 
wrp.\ to the next statement in each thread; and finally
projecting the results onto each of the $k$ participating threads.
As follows from Lemmas~\ref{small:lemma} and~\ref{two:lemma}, it is sufficient
to consider only bounded values of $k$, and indeed $k=2$ is sufficient
for all examples that we considered.
%% it is sufficient to consider the case where $k$ equals $2$.
%% Details are in Appendix~\ref{sec:symbpost}.

%% e define a projection operation $\symbproj$ from 
%% symbolic $k$-configurations onto the set of their $k$ projections
%% onto each thread $\thread_i$ for $i = 1, \ldots, k$, such that
%% for any $k$-configuration $\sconf_k$ we have that
%% $\sconf_k \models \ac_k$ implies
%% $\thabstractionof{\sconf_k} \models \symbproj(\ac_k)$.
%% We extend $\symbproj$ to sets of symbolic $k$-configurations by
%% $\symbproj(\symbrep_k) = \mathop\cup_{\ac_k \in \symbrep_k}\symbproj(\ac_k)$.
%%   In the other direction, we define an operation $\symbintersect$
%%   on symbolic representations,
%% which produces a symbolic representation of all ways in which they can
%% be combined to symbolic $k$-configurations, such that
%% $\confs \models \symbrep$ implies
%% $\thconcretizationof{\confs} \models \symbintersect(\symbrep)$.
%% We can then derive a postcondition operation on symbolic representations
%% $\symbrep$, which
%% \begin{inparaenum}[(1)]
%% \item
%%   first derives $\symbintersect(\symbrep)$,
%%   \item
%%   then for each $\ac_k \in \symbintersect(\symbrep)$ 
%%   computes its postcondition 
%%   wrp.\ to the next statement in each thread, in the standard way,
%%   resulting in   $\symbrep_k'$
%% \item
%%   and finally produces $\symbproj(\symbrep_k')$.
%% \end{inparaenum}
%% Details of these operations are described in Appendix XXX.
%% \td{What to do with: ``If the controller is triggered, then ${\tt Val}^{\tt m}$ is updated as is described in the semantic section 5''}
%% \td{What to do with
%% ``This procedure  should of course be optimized by
%% removing redundant symbolic configurations and redundant combinations of
%% symbolic configurations in standard ways.''}

