\section{Concurrent Data Structure Implementations}
\label{programs:section}

In this section, we introduce our representation of implementations of
concurrent data structures, and of their corresponding correctness criteria.

\subsection{Data Structures}
A {\it data structure} $\dstruct$ is a pair
$\tuple{\ddomain,\malphabet}$,
where $\ddomain$ is the {\it data domain} and 
$\malphabet$ is an alphabet of method names.
%
An {\it operation} 
%\todo{I have changed {\it event} to {\it operation.}}
$\op$ is of the form
$\mname(\indata,\outdata)$ where 
$\mname\in\malphabet$ is a method name and 
$\indata,\outdata\in\ddomain$ are the {\it input} resp.\ {\it output}
data values.
%
A {\it trace} of $\dstruct$ is a sequence of operations.
The standard (sequential) semantics of a data structure $\dstruct$ is provided by
a set $\denotationof\dstruct$ of allowed traces,
called the {\it legal} traces of $\dstruct$.
%
We often identify $\dstruct$ with its behavior $\denotationof\dstruct$.

For example, for the {\tt Set} data structure, the set of method names
is given by $\set{{\tt add}, {\tt rmv}, {\tt ctn}}$,
and the data domain is the  union $\intgrs\cup\boolset$
of the sets of integers and Booleans.
%
Input and output data values are in $\intgrs$ and $\boolset$
respectively.
%
For instance, the operation ${\tt add}(3,\true)$ successfully adds the value
$3$, while ${\tt ctn}(2,\false)$ is a failed search
for $2$ in the set.
\todo[inline]{Should really $\boolset$ be in the data domain?}
\todo[inline]{How to define the semantics of {\tt Set}?}
%
%(see Appendix~\ref{dstruct:sepcs:section} for a formal specification
%of $\denotationof{\tt Set}$, $\denotationof{\tt Queue}$, and $\denotationof{\tt Stack}$.)

%% \begingroup
%% \setlength\intextsep{-8pt}
%% \setlength{\columnsep}{4pt}
%% \begin{wrapfigure}{R}{6cm}
%% \center 


\subsection{Concurrent Data Structure Implementations}
A (concurrent) implementation of a data structure assigns, to each
method name, a method which performs operations
%% In this section, we introduce programs that consist
%% of arbitrary numbers of concurrently executing threads
%% that access a concurrent data structure.
%(Appendix~\ref{planguage:section} contains
%the language syntax used in our tool).
%
%% Each thread executes a method that performs an operation
on the data structure,
Each data structure implementation also comes with an initialization method,
named {\tt init}, which initializes its shared state.

%\todo{What is the difference between methods here in 
%Section~\ref{linearizability:section}?}
%
Each method declares local variables
  and a method body.
%
Variables are either pointer variables
(to heap cells), 
or data variables, assuming values from an infinite (ordered)
domain,
or from some finite set $\fset$ that includes the Boolean values $\boolset$.
%
We assume w.l.o.g. that the infinite set is given by the set 
$\intgrs$ of integers.
%
The body is built in the standard way
from atomic commands, using standard control
flow constructs (sequential composition, selection, and loop constructs).
%
Each statement is equipped with a unique label.
%
We assume that the set of local variables include the parameter of the method
in addition to the program counter {\tt pc} whose value is the label of the next
statement to be executed.
%
Method execution is terminated by executing a {\tt return} command,
which may return a value.
The global variables can be
accessed by all threads, whereas local variables can be accessed only
by the thread which is invoking the corresponding method.
%
We assume that all global variables are pointer variables.

%
Heap cells have a fixed set $\fieldset$ of fields, namely
data fields that assume values in
$\intgrs$ or $\fset$, and lock fields.
Furthermore, each cell has one or several named pointer fields.
For instance, in data structure implementations based on singly linked lists,
each heap cell has a pointer field named {\tt next}, in implementations
based on skip lists, there is an array of pointer fields named
{\tt next[k]} where {\tt k} ranges from $1$ to the maximum level of the
skip list.
%% heap cells are organized into singly-linked lists.%\todo{Possibly with a 
%cycle?}
We use the term $\intgrs$-field for a data field that
assumes values in $\intgrs$, and the terms $\fset$-field and lock field
with analogous meaning.
Atomic commands include assignments between data variables, 
pointer variables, or fields of cells pointed to by a pointer variable.
%
The command {\tt new Node()} allocates a new structure of type
{\tt Node} on the heap, and returns a reference to it.
%
The compare-and-swap command {\tt CAS(\&a,b,c)} atomically
compares the values of {\tt a} and {\tt b}.
If  equal, it assigns the value of
{\tt c} to {\tt a}  and returns {\tt true}, 
otherwise, it leaves {\tt a} unchanged and returns {\tt false}. 
%
We assume a memory management mechanism, which automatically collects
garbage, and also ensures that a new cell is fresh, i.e., has
not been used before by the program; this avoids the so-called
ABA problem (e.g.,~\cite{MS:QueueAlgorithms}).


\subsection{Data Structure Correctness and Linearizability}
In a concurrent
data structure implementation, an operation $\mname(\indata,\outdata)$
gives rise to two {\it actions}, namely
a {\it call action} $\call\eid\mname\indata$ and 
a {\it return action} $\return\eid\mname\outdata$, where $\eid\in\nat$ is an
{\it action identifier}.
%
A {\it history} $\history$ is a sequence of actions such that
for each return action $\action_2$ in $\history$ there is a unique
{\it matching}  call action $a_1$ where the action identifiers
in $\action_1$ and $\action_2$ are identical and different from the action
identifiers of all other actions in $\history$ and such that
$\action_1$ occurs before $\action_2$ in $\history$.
%
A call action carries always the same method name as its matching return
action.
A call action which does not match any return action in $\history$ is said
to be {\em pending}.
A history without pending call actions is said to be {\em complete}.
A {\em completed extension} of $\history$ is a complete history
$\history'$ obtained from $\history$ by
  appending (at the end) zero or more return actions that are matched by
  pending call actions in $\history$, and
  thereafter removing pending call actions.
%
For action identifiers $\eid_1,\eid_2$, we write
$\eid_1\occursbefore\history\eid_2$ to denote that
the return action with identifier $\eid_1$ occurs before
the call action with identifier $\eid_2$ in $\history$.
We say that a history is {\it sequential} if
it is of the form
$\action_1\action'_1\action_2\action'_2\cdots\action_\nn\action'_\nn$
%% where $\action'_\nn$ can be absent and
where $\action'_\ii$ is the matching action of $\action_\ii$ 
for all $\ii:1\leq\ii\leq\nn$, i.e., each call action 
is immediately followed by the matching return action. 
%
We identify a sequential history of the above form with
the corresponding trace 
$\op_1\op_2\cdots\op_\nn$ where
$\op_\ii=\mname(\indata_\ii,\outdata_\ii)$,
$\action_\ii=\call{\eid_\ii}{\mname}{\pindata\ii}$, and
$\action_\ii=\return{\eid_\ii}{\mname}{\poutdata\ii}$,
i.e., we merge each call action together with the matching return action
into one operation.
%
A complete history $\history'$ 
is a {\it linearization} of $\history$ if
(i) $\history'$ is a permutation of $\history$,
(ii) $\history'$ is sequential, 
and
(iii) $\eid_1\occursbefore{\history'}\eid_2$
if $\eid_1\occursbefore{\history}\eid_2$
for each pair of action identifiers $\eid_1$ and $\eid_2$.
%
We say that a sequential history $\history'$ is {\it valid} wrt.\ $\dstruct$ if
the corresponding trace is in $\denotationof{\dstruct}$.
%
We say that $\history$ is {\it linearizable} wrt.\ $\dstruct$ if there is
a completed extension of $\history$, which has
a linearization that is valid wrt.\ $\dstruct$.
%
A set $\histories$ of histories is {\it linearizable} wrt.\ $\dstruct$ if 
all members of $\histories$ are  {\it linearizable} wrt.\ $\dstruct$.

\subsection{Verification of Correctness}
We verify the correctness of a concurrent data structure implementation by
verifying that any combination of method executions satisfies the
appropriate correctness criterion, e.g., of linearizability.

To this end, define a {\em program} to consist 
of an arbitrary number of concurrently executing threads
that access a concurrent data structure.
%(Appendix~\ref{planguage:section} contains
%the language syntax used in our tool).
%
Each thread executes a method that performs an operation
on the data structure. We assume that the data structure has been initialized
by the {\tt init} method prior to the start of program execution.

To verify correctness of the data structure, we must verify that the
sequence of call and return actions satisfy the criterion imposed by
linearizability. Linearizability states that each operation on the data
structure can be considered as being
performed atomically at some point, called the {\em linearization point (LP)},
between its invocation and return.
It is difficult to use this definition of linearizability directly, therefore
a number of techniques have been developed for reducing it to other criteria,
whose verification is more tractable. The general pattern of these criteria
is to check that for each program execution,
the ordering of call and return actions, and possibly other statement
executions in methods, satisfy certain criteria. For completeness, let us
survey some of these techniques.

The most common technique is to annotate methods with LPs. More precisely,
each method is instrumented to announce precisely when the linearization point
occurs during each method invocation. In many cases, the linearization point
can be associated with a particular statement in the method code (so called
{\em fixed linearization points}). Linearizability can then be verified by
checking that the occurrence of linearization points is allowed by the
sequential semantics of the particular data structure that is implemented.
There are also a number of
algorithms for which the linearization point depends on the relative order of
other method executions.
\bjcom{Provide examples}
For some of these algorithms, linearizability can be verified by checking
that the relative ordering of call and return actions satisfy certain
criteria. 

\begin{figure}{R}{6cm}
\center 

\begin{tikzpicture}[]

\node(s0)[draw,line width=1pt,fill=blue!10,circle,text=black]
{$s_0$};

\node(s1)[draw,line width=1pt,fill=blue!10,circle,text=black,anchor=center]
at ($(s0.center)+(80pt,0pt)$) {$s_1$};


\node(s2)[draw,line width=1pt,fill=blue!10,circle,text=black,scale=0.9,anchor=center ]
at ($(s0.center)+(40pt,-61pt)$) {$s_2$};

\node[draw,line width=1pt,circle,text=black,anchor=center,minimum size=5mm]
at ($(s0.center)+(40pt,-61pt)$) {};


\draw[->,>=stealth,out=45,in=135,line width=1pt] (s0) to 
node[above=-1pt,pos=0.5] {\footnotesize${\tt add}(x,\true)$} 
(s1);

\draw[->,>=stealth,out=225,in=315,line width=1pt] (s1) to node[above=0pt,pos=0.5] {\footnotesize${\tt rmv}(x,\true)$} 
(s0);

\draw [->,>=stealth,thick,out=270,in=180,line width=1pt] (s0) 
to 
node[left,pos=0.2] {\footnotesize${\tt add}(x,\false)$} 
node[left,pos=0.45] {\footnotesize${\tt rmv}(x,\true)$} 
node[left,pos=0.8] {\footnotesize${\tt ctn}(x,\true)$} 
 (s2);

\draw [->,>=stealth,thick,out=270,in=0,line width=1pt] (s1) 
to 
node[right,pos=0.2] {\footnotesize${\tt add}(x,\true)$} 
node[right,pos=0.45] {\footnotesize${\tt rmv}(x,\false)$} 
node[right,pos=0.8] {\footnotesize${\tt ctn}(x,\false)$} 
 (s2);




\end{tikzpicture}
\caption{Set observer.}
\label{set:observer:fig}
\end{figure}

The checking of such ordering constraints can be done by 
{\em observers}, 
as introduced in~\cite{AHHR:integrated}. 
%
Observers are
finite automata extended with a finite set of {\em registers}
that assume values in $\intgrs$. 
%
%
At initialization,
the registers are nondeterministically
assigned arbitrary values, which never change
during a run of the observer. 
%
Formally, an observer $\observer$ is a tuple
$\otuple$ where $\ostateset$ is a finite set 
of {\it observer states} including the 
{\it initial state} $\oinitstate$ and
the {\it accepting state} $\oaccstate$, 
a finite set $\ovarset$  
of {\it registers}, and $\otransitionset$ is a finite
set of {\it transitions}.
%
%
Transitions are of the form 
$\tuple{\ostate_1,\mname(\inxvar,\outxvar),\ostate_2}$ where 
$\inxvar$ and $\outxvar$ are either registers or constants, i.e.,
transitions are labeled by 
operations whose input or output data may be parameterized on registers.
%
%% The observer processes a trace one operation at a time.
%% %
%% If there is a transition, whose label, after replacing registers by their
%% values, matches the operation, such a transition is performed. 
%% %
%% If there is no
%% such transition, the observer remains in its current state.
%
The observer accepts a trace if it can  be processed in such a way that
an accepting state is reached.
%
Observers can be used to give {\it exact} specifications of
the behaviors of data structures such as sets, queues, and stacks.
%
The observer is defined in such a way that it accepts precisely those
traces that do {\em not} belong to the behavior
of the data structure.
%
This is best illustrated by an example. Fig.~\ref{set:observer:fig}
depicts an observer that accepts the
sequences of operations that are {\em not} in $\denotationof{\tt Set}$.
%
%Other
%observers for stacks and queues can be found in~\cite{AHHR:integrated}
%and Appendix~\ref{dstruct:sepcs:section}.











