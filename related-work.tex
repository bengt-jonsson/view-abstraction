\paragraph{Related Work}
\bjcom{Still to be written}
\cite{GBCS:pldi07} presents a thread-modular shape analysis for multi-threaded
programs in which locks protect portions of the heap, and threads have to
acquire a lock before accessing the corresponding portion of the heap.
Fine-grained concurrent algorithms do not follow this pattern.

Thread-modular approaches for verifying fine-grained concurrent algorithms
have been presented in several works. 
\cite{AHHR:integrated} introduce observers to specify the semantics of
data structures. It verifies program using the thread-modular approach, where
heaps are specified by reachability constraints between cells pointed to
by program variables. The work is applied to concurrent stack and
queue implementaitons based on singly-linked lists.
\cite{Holik:sas17} present a more efficient way to handle the expensive
interference steps for common programming idioms in lock-free data structures.
\cite{Quy:sas16} extend this approach to be able to verify linearizability
for a large class of concurrent data structures based on singly-linked lists.
It used a specifically designed finitary abstraction of unbounded singly-linked
list segments.

\bjcom{We should say something about to CAVE}

\bjcom{We should have a paragraph on verification of skiplists, and on verification of the TS queue/stack}

\todo[inline]{Below is related work from the SAS paper}
Much previous work has been devoted to
the {\it manual} verification of linearizability for
concurrent programs. Examples
include~\cite{LF:pldi13,Aaron:logical:linearizability}.
%% %
%% For instance, Turon {\it et al.} \cite{Aaron:logical:linearizability}
%% propose a semantic model, 
%% based on Kripke logical relations, that can be used for proving linearizability.
%% %
%% Liang {\it et al.} \cite{LF:pldi13}
%% propose a program logic with a lightweight instrumentation mechanism 
%% for verifying linearizability for algorithms with non-fixed LPs.
%
In \cite{OHearnlist}, O'Hearn {\it et al.}  define a
{\it hindsight lemma} that
provides a non-constructive evidence for linearizability. 
%
The lemma is used to prove linearizability of an optimistic variant of 
the lazy set algorithm.
%% by identifying the hindsight property of the algorithm. 
Vafeiadis \cite{Vafeiadis:Thesis}
uses forward and backward simulation relations together
with history or prophecy variables to prove linearizability.
%
These approaches are manual, and without
tool implementations.
{\it Mechanical} proofs of linearizability, using interactive theorem
provers, have been reported in 
\cite{Colvin:Lazy-List,Derrick:fm14,SWD:cav12,SDW:tcl14}.
%
For instance, Colvin {\it et al.} \cite{Colvin:Lazy-List}
verify the lazy set algorithm in  PVS,
using a combination of forward and backward simulations.

Several techniques for verifying linearizability are based on establishing
commutation properties between atomic actions inside each method.
Such methods can be partly automated~\cite{EQSST:tacas10,singh:issre16}
or part of automated program analysis~\cite{LMP:cav14}.

There are several works on {\it automatic} verification of linearizability.
%
In \cite{Vafeiadis:cav10}, Vafeiadis
develops an automatic tool for proving  linearizability that
employs instrumentation to verify logically pure executions.
%
However, this work can handle non-fixed LPs only for read-only methods,
i.e, methods that do not modify the heap.
%
This means that the method cannot handle
 algorithms like the {\it Elimination} queue ~\cite{Shavit:ElimQueue}, {\it HSY} stack~\cite{HSYstack}, {\it CCAS}~\cite{Harris:CAS}, 
{\it RDCSS}~\cite{Harris:CAS} and {\it HM} set~\cite{ArtOfMpP} that we consider in this paper. In addition, their shape abstraction is not powerful enough to handle algorithms like {\it Harris} set~\cite{Harris:list} and {\it Michael} set~\cite{Michael:list} that are also handled by our method.
%
%For instance, in the {\it HSY} stack algorithm both
%the {\tt push} and {\tt pop} operations modify
%the heap and help each other~\cite{HSYstack}. 
%
Chakraborty {\it et al.} \cite{HSV:concur13}
describe an ``aspect-oriented'' method for  modular verification 
of concurrent queues that they use to prove linearizability of the Herlihy/Wing queue.
Bouajjani et al. \cite{BEEH:icalp15} extended this work to show that verifying 
linearizability for certain
fixed abstract data types, including queues and stacks, is reducible to 
control-state reachability. 
%
We can incorporate this technique into
our framework by a suitable construction of observers.
The method can not be applied to sets.
%
The most recent work of Zhu {\it et al.} \cite{Poling}
describe a tool that is applied for specific set, queue, and stack  
algorithms. For queue algorithms, their technique can handle queues with helping mechanism except for {\it HW} queue~\cite{HeWi:linearizability} which is handled by our paper.
%
For set algorithms, the authors can only handle those that perform an optimistic contains (or lookup) operation by applying the {\it hindsight lemma} from 
\cite{OHearnlist}. 
%
Hindsight-based proofs provide only {\it non-constructive} 
evidence of linearizability.
%
Furthermore, some algorithms (e.g., the unordered list algorithm
considered in Sec.~\ref{section:experiments} of this paper)
do not contain the code patterns required by the hindsight method. Algorithms with non-optimistic contains (or lookup) operation like {\it HM}~\cite{ArtOfMpP}, {\it Harris}~\cite{Harris:list} and {\it Michael}~\cite{Michael:list} sets cannot be verified by their technique.  
%
Vechev {\it et al.}~\cite{Vechev:spin09}
check linearizability with user-specified non-fixed LPs,
using a tool for finite-state verification.
Their method assumes a bounded number of threads, and
they report state space explosion when having more than two threads.
%
Dragoi  {\it et al.} \cite{Henzinger:CAV13} describe a method for proving
linearizability that is applicable to algorithms with non-fixed LPs.
%
However, their method needs to rewrite the implementation so that all operations 
have linearization points within the rewritten code.
%
\v{C}ern{\'y} {\it et al} \cite{CernyRZCA:CAV10} show decidability of a class
of programs with a bounded number of threads operating on concurrent data structures.
%
%% This paper is about a different problem
%% Horn and Kroening \cite{Kroening-Linearizability:FORTE15}
%% have recently presented an efficient linearizability checker based on {\it P-compositionality} 
%% which can be applied when having a bounded number of threads.
%
Finally, the works~\cite{AHHR:integrated,BLMRS:cav08,Vafeiadis:vmcai09}
all require fixed linearization points.

We have not found any report in the literature of a
verification method that is sufficiently powerful to
automatically verify the class of concurrent set
implementations based on sorted and non-sorted
singly-linked lists having non-optimistic contains (or lookup) operations we consider. For instance %For instance,
%the shape abstraction of the CAVE implementation
%\cite{Vafeiadis:cav10}
%is not powerful enough to handle
the lock-free sets of {\it HM}~\cite{ArtOfMpP},
{\it Harris}~\cite{Harris:list}, or {\it Michael}~\cite{Michael:list},
or unordered set of~\cite{Zhang:unorderedlist},
%reported in
%Section~\ref{section:experiments} (as also confirmed by the CAVE
%tool implementors).
%The same applies to
%the tool in~\cite{Poling}.



