\paragraph{Related Work}
A large number of techniques have been developed for representing heap structures
in automated analysis, including,
e.g., separation logic and various related graph formalisms
\cite{InvaderCAV08,thor10,rival11,dudka13}, other logics \cite{Sagiv02,pale97},
automata \cite{habermehl:forest}, or graph grammars \cite{juggrnaut10}. 
Most works apply these to sequential programs.

Approaches for automated verification of concurrent algorithms are limited to the
case of singly-linked
lists~\cite{AHHR:integrated,meyer:vmcai16,Quy:sas16,Sagiv:correlation,Vafeiadis:cav10}.
Furthermore, many of these techniques impose additional restrictions on the considered verification problem, such as bounding the number of accessing
threads~\cite{Amit:comparisonAbstraction,Vechev:spin09,CernyRZCA:CAV10}.

In~\cite{AHHR:integrated}, concurrent programs operating on SLLs are analyzed
using an adaptation of a transitive closure logic~\cite{BiRa:vmcai06}, combined with
tracking of simple sortedness properties between data elements; the approach does
not allow to represent patterns observed by threads when following sequences of
pointers inside the heap, and so has not been applied to concurrent set
implementations.
In our recent work~\cite{Quy:sas16}, we extended this approach to handle SLL implementations
of concurrent sets by adapting a
well-known abstraction of singly-linked lists ~\cite{MYRS:Canonical} for concurrent programs.
The resulting technique is specifically tailored for singly-links.
Our fragment abstraction is significantly simpler conceptually, and can therefore be  adapted
also for other classes of heap structures.
The approach of~\cite{Quy:sas16} is the only one with a shape representation strong enough to
verify  concurrent set
implementations based on sorted and non-sorted
singly-linked lists having non-optimistic contains (or lookup) operations we consider, such as
the lock-free sets of {\it HM}~\cite{ArtOfMpP},
{\it Harris}~\cite{Harris:list}, or {\it Michael}~\cite{Michael:list},
or unordered set of~\cite{Zhang:unorderedlist}. As shown in
Section~\ref{section:experiments}, our fragment abstraction can handle them
as well as also algorithms employing skiplists and arrays of singly-linked lists.

There is no previous work on automated verification of skiplist-based concurrent algorithms.
Verification of {\em  sequential} algorithms have been addressed under restrictions, such as limiting the
number of levels to two or three~\cite{boxes13,Quy:atva13:journal}. The work~\cite{Sanchez:skiplists}
generates verification conditions for statements in sequential skiplist implementations. All these
works assume that skiplists have the well-formedness property that any higher-level lists is a
sublist of any lower-level list, which is true for sequential skiplist algorithms, but false for
several concurrent ones, such as~\cite{ArtOfMpP,Linden:opodis13}.

Concurrent algorithms based on arrays of SLLs, and including timestamps, e.g., addressing
the challenge of verifying the algorithms in~\cite{ts-stack} have been verified using manual techniques in~\cite{Khyzha:esop17,singh:issre16}.

Our fragment abstraction is related in spirit to other formalisms that
abstract dynamic graph structures by defining some form of equivalence on
its nodes (e.g.,~\cite{spotlight07,Sagiv02,habermehl:forest}). These have
not been applied to verify functional correctness fine-grained concurrent
algorithms,  except for a limited number of SLL-based algorithms. Fragment
abstraction's representation of both local an global information allows to
extend the applicability of this class of techniques.



