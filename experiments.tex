%--------------------------------------------------------
\section{Experimental Results}
\label{section:experiments}
%--------------------------------------------------------
\newcommand{\emm}{(emm)}
%\begin{table}
%\footnotesize
%\begin{tabular}{|c | c | }
%  \hline
% \textsf{{{Algorithms}}} &  \textsf{{{Time(s)}}} \\
%\hline
%%\textsf{TS stack  ~\cite{MS:QueueAlgorithms}}\;\;\;\;\;\;  & \textsf{76.02} \\
%\textsf{TS stack  ~\cite{ts-stack}}\;\;\;\;\;\;  & \textsf{176} \\
%%\textsf{TS queue  ~\cite{MS:QueueAlgorithms}}& \textsf{11.07} \\
%\textsf{TS queue  ~\cite{ts-stack}}& \textsf{101} \\
%\hline
%\end{tabular} aaa \;\;\;\;\;\; \begin{tabular}{|c | c | }
%  \hline
% \textsf{{{Algorithms}}} &  \textsf{{{Time(s)}}} \\
%\hline
%\textsf{Lock-free set   ~\cite{ArtOfMpP}} & \textsf{1992} \\
%
%\textsf{Optimistic set ~\cite{MS:QueueAlgorithms}}& \textsf{500} \\
% 
%\textsf{Priority queue 1 ~\cite{Shavit:ElimQueue}}  &  \textsf{1320} \\
%
%\textsf{Priority queue 2~\cite{Shavit:ElimQueue}}  &  \textsf{599} \\
%\hline
%\end{tabular}
%
%\vspace*{1cm}
%
%\begin{tabular}{|c | c | }
%  \hline
%   \textsf{{{Algorithms}}} &  \textsf{{{Time (s)}}} \\
%\hline
%\textsf{Treiber stack  ~\cite{Treiber:stack}}\;\;\;\;\;\; & \textsf {18} \\
%
%\textsf{MS lock-free queue  ~\cite{MS:QueueAlgorithms}}& \textsf{21} \\
%
%\textsf{DGLM queue  ~\cite{Doherty:lockfree}}&  \textsf {16} \\
%
%\textsf{Vechev-CAS set  ~\cite{Vechev:list}}  & \textsf{86} \\
%
%\textsf{Vechev-DCAS set  ~\cite{Vechev:list}}   & \textsf{16}  \\
%
%\textsf{Michael lock-free set ~\cite{Michael:list}}  & \textsf{178}  \\
%
%\textsf{Pessimistic set  ~\cite{ArtOfMpP}}&\textsf{30} \\
%
%\textsf{Optimistic set ~\cite{ArtOfMpP}}& \textsf{25} \\
%
%\textsf{Lazy set ~\cite{Lazyset}  }  & \textsf {34} \\
%
%\textsf {O$\textquotesingle $Hearn set  ~\cite{OHearnlist}}     & \textsf{88} \\
%
%\textsf{HM lock-free set  ~\cite{ArtOfMpP} } & \textsf{120} \\
%\hline
%\end{tabular}
%
%\end{table}



\begin{figure}[]
%% \vspace*{-1cm}
\center
\footnotesize


\begin{tabular}{|c | c | }
  \hline
   \textsf{{\textbf{Algorithms}}} &  \textsf{{\textbf{Time (s)}}} \\
\hline
\textsf{{{ARRAY OF SLLs}}} & \\
\hline
%\textsf{TS stack  ~\cite{MS:QueueAlgorithms}}\;\;\;\;\;\;  & \textsf{76.02} \\
\textsf{TS stack  ~\cite{ts-stack}}\;\;\;\;\;\;  & \textsf{176} \\
%\textsf{TS queue  ~\cite{MS:QueueAlgorithms}}& \textsf{11.07} \\
\textsf{TS queue  ~\cite{ts-stack}}& \textsf{101} \\
\hline
 \textsf{{{SKIP-LISTS}}} &   \\
\hline
\;\;\;\;\textsf{Lock-free skip-list   ~\cite{ArtOfMpP}}\;\;\;\;\;\;  & \textsf{1992} \\

\textsf{Optimistic skip-list ~\cite{MS:QueueAlgorithms}}& \textsf{500} \\
 
\textsf{Priority queue skip-list 1 ~\cite{priority-queue-skiplist}}  &  \textsf{1320} \\

\textsf{Priority queue skip-list 2~\cite{jonatan:priorityqueue}}  &  \textsf{599} \\
\hline
   \textsf{{{SINGLY-LINKED LISTS}}} &   \\
\hline
\textsf{Treiber stack  ~\cite{Treiber:stack}}\;\;\;\;\;\; & \textsf {18} \\

\textsf{MS lock-free queue  ~\cite{MS:QueueAlgorithms}}& \textsf{21} \\

\textsf{DGLM queue  ~\cite{Doherty:lockfree}}&  \textsf {16} \\

\textsf{Vechev-CAS set  ~\cite{Vechev:list}}  & \textsf{86} \\

\textsf{Vechev-DCAS set  ~\cite{Vechev:list}}   & \textsf{16}  \\

\textsf{Michael lock-free set ~\cite{Michael:list}}  & \textsf{178}  \\

\textsf{Pessimistic set  ~\cite{ArtOfMpP}}&\textsf{30} \\

\textsf{Optimistic set ~\cite{ArtOfMpP}}& \textsf{25} \\

\textsf{Lazy set ~\cite{Lazyset}  }  & \textsf {34} \\

\textsf {O$\textquotesingle $Hearn set  ~\cite{OHearnlist}}     & \textsf{88} \\

\textsf{HM lock-free set  ~\cite{ArtOfMpP} } & \textsf{120} \\
\hline
\end{tabular}

%\vspace{10mm}
%\vspace*{-0.3cm}
\caption{Experimental results for verifying concurrent programs}
%\vspace*{-0.7cm}
\label{Experiments:fig}
\end{figure}
%\vspace{-0.2cm}
Based on our framework, we have implemented a tool in OCaml, and used
it for
verifying concurrent algorithms (both lock-based and lock-free)
including timestamps stack and queue, skip-list sets and priority queues as well as singly-linked lists algorithms (stacks, queues, sets). The experiments were performed on a desktop 2.8 GHz processor with 8GB memory. The results are presented in Fig.~\ref{Experiments:fig}, where running times are given in seconds. 
All experiments start from the initial heap,  
and end either when the analysis reaches the fixed point or when a violation of safety properties or linearizability is detected. 

\paragraph{Linearizability.}

%Our tool is capable of verifying linearizability by forming  as in the automata-theoretic approach [Vardi and Wolper 1986], also adopted in [~\cite{Quy:sas16}, the cross-product of the program and the corresponding observer, where the observer synchronizes with the program on call and return actions that are recognized by the observer. This reduces the problem of checking each of the above constraints to the problem of checking that the corresponding observer cannot reach an accepting state.
Our tool is capable of verifying linearizability by forming  as in the automata-theoretic approach ~\cite{VW:modelchecking}, and using the technique in ~\cite{Quy:sas16} in which they use observers to specify the sequential semantics of data structures and define controllers that specify the linearization policies. This reduces the problem of checking linearizability to the problem of checking that the corresponding observer cannot reach an accepting state.
\paragraph{Safety Properties.}
Our tool is also capable of verifying memory related safety properties such as the absence of null pointer dereferencing,  dangling pointers,  double-freeing, cycles, and dereferencing of freed nodes, as well as sortedness. In fact, for each algorithm, the time reported in Fig.~\ref{Experiments:fig} is the sum of the 
times taken to show linearizability and all the properties mentioned above.

\paragraph{Array of SLLs benchmarks} For array of SLLs benchmarks, we verified both TS stack and TS queue in ~\cite{ts-stack} which is one of the most recent challenging problems in linearization verification. We use observers with call and return actions in ~\cite{BEEH:icalp15}, and also adopted in ~\cite{Quy:sas16}, where the observer synchronizes with the program on call and return actions that are recognized by the observer. To the best of our knowledge, our tool is the first tool which can handle both TS stack and queue automatically.
\paragraph{Skip-List benchmarks} The concurrent Skip-List algorithms pose a hard challenge to automated
verification, since they employ many pointer relations, whose interrelation must be discovered during
verification. Several approaches to this problem, collectively termed shape analyses, have been developed.
We have verified four popular Skip-List algorithms including two sets and two priority queues. Note that, in the Skip-List set in ~\cite{ArtOfMpP}, the property where higher-level lists are sub-list of lower-level lists is not preserved. This problem make this Skip-List set much harder to handle. We use the set observer in ~\cite{Quy:sas16} to specify the sets, and priority queue observers in ~\cite{AHHR:integrated:rep} to specify the priority queues. To the best of our knowledge, there has not been any previous work that can
automatically verify correctness of Skip-List algorithms with unbounded number of levels. We are the first who verify these algorithms automatically.

\paragraph{Singly-linked list benchmarks} We also verified singly-linked list algorithms including stack, queue, and set algorithms. These algorithms are also handled by ~\cite{Quy:sas16}. 

\paragraph{Running Times.}
As can be seen from the table, the running times
vary in the different examples.
%
This is due to the types of shapes that are produced during the analysis.
%
For instance, skip-lists algorithm have much longer running times. This is due to 
the number of pointer variables and their complicated shapes. Whereas, other algorithms 
produce simple shape patterns and hence they have shorter running times.
\paragraph{Error Detection}
 In addition to establishing correctness of the original versions of the
benchmark algorithms, we tested our tool with intentionally inserted bugs. For example, we emitted setting time statement in line 5 of the {\tt push} method in TS stack algorithm. The tool, as expected, successfully detected and reported the bug.
 



