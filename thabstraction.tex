\section{Thread Abstraction}
\label{thabstraction:section}
By Theorem~\ref{thm:soundness}, linearizability can be verified by
establishing a 
reachability property for the product $\system=\augprog\compose\monitor$.
This verification must handle the challenges of
an unbounded number of threads, an unbounded heap, and unbounded data domain.
In this section, we describe
how we handle an unbounded number of threads by adapting
the thread-modular approach~\cite{BLMRS:cav08}, and extending it to
handle global transitions where several threads synchronize,
due to interaction between controllers. 
%% Handling of unbounded heap and data is described in
%% Section~\ref{sec:annotations}.

\todo[inline]{I believe we do not need to define thread abstraction explicitly:
  it only complicates the treatment. We can go directly to
  fragment abstraction, and not burden the reader with the below}
In this section, we describe how the global program state is projected onto
a particular thread. Note that this projection is different from the
projection defined in Section~\ref{semantics:section}.
\bjcom{We need to fix the terminology of view/projection, etc.}


Let a heap cell $\cell$ be {\em accessible} from a thread $\thread$ if
it is reachable (directly or via sequence of
next-pointers) from a global variable or local variable of $\thread_1$.
A {\it thread abstracted} configuration of $\system$
is  a pair $\taconftuple$, where
$\taconf \in \confsetof{\system}$ with
$\athreadsof{\taconf} = \set{\thread_1}$, such that
each cell in the heap of $\taconf$ is accessible from $\thread_1$
(i.e., the heap has no garbage), and where $\private$ is a predicate over the
cells in the heap of $\taconf$.
For a system configuration $\sconf$,
%% thread abstracted configuration $\tuple{\sconf_1,\private}$,
let $\thabstractionof{\sconf}$ be the set of 
thread abstracted configurations $\taconftuple$, such that
%% we write $\sconf \models \taconftuple$ if
\begin{inparaenum}[(i)]
\item
  $\taconf$ can be obtained by
  removing all threads from $\sconf$ except one,
  renaming that thread
  to $\thread_1$, and thereafter removing all heap cells that are not
  accessible from $\thread_1$,
\item
  $\private(\cell)$ is true iff in $\sconf$, the heap cell
  $\cell$ is accessible only from
  the thread that is not removed.
\end{inparaenum}
For  a set $\confs$ of system configurations, define
$\thabstractionof{\confs} = \mathop\bigcup_{\sconf \in \confs}\thabstractionof{\sconf}$.
%% $ be the set of 
%% define its (thread-modular) abstraction 
%% $\thabstractionof\confs$ as the set of thread abstracted configurations
%% $\taconftuple$ such that $\sconf \models \taconftuple$
%% for some $\sconf \in \confs$.
Conversely, for a set $\confs$ of thread abstracted configurations,
define
$\thconcretizationof\confs=
\setcomp{\sconf}{\thabstraction(\{\confs\})\subseteq\confs}$.
%
Define the postcondition operation $\thpost$ on sets of
thread abstracted configurations in the standard way by
\(
\thpostof{\confs} =
\thabstraction(\postof{\thconcretization(\confs)})
\).
By standard arguments, $\thpost$ then soundly overapproximates
$\post$.
%% in the sense that
%% \(
%% \thabstractionof{\postof{\confs}} \subseteq \thpostof{\thabstractionof{\confs}}
%% \).

%% \begin{lemma}
%% For sets of concrete configurations $\confs_1,\confs_2$ and a set of abstract configuration ${\confs_3}$, 
%% %if\todo{Is the relation defined for sets of configurations?}
%% if $\confs_2 = \postof{\confs_1}$ and ${\confs_3} = \thpostof{\thabstractionof{\confs_1}}$, then we have $\confs_2 \subseteq \thconcretizationof{\confs_3}$

 
%% %$\thabstractionof{\confs_1}\thabsmovesto{}*\thabstractionof{\confs_2}$.
%% \end{lemma}

\begin{figure}[h]
\center
\input lazy-list-shape	
\caption{Example of (a) heap state of Lazy Set algorithm, (b) its thread-abstracted version, and (c) symbolic version.
  The observer register $x$ has value $9$.}
\label{fig:llshapes}
\end{figure}

\todo[inline]{Here, we need an example: take, e.g., something from TS Stack}

