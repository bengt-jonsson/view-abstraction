\section{Introduction}

In this paper, we consider one of the most difficult current challenges in
software verification, namely to automate its
application to algorithms with an unbounded number of
threads that concurrently access a dynamically allocated shared state.
Such algorithms are of central importance in concurrent programs. 
Thye are widely used in libraries,
such as the Intel Threading Building Blocks or
the \url{java.util.concurrent} package,
to provide efficient concurrent realizations of
simple interface abstractions.
They are notoriously difficult to get correct and verify, since they
often employ fine-grained synchronization and avoid locking wherever
possible. A number of bugs  in published
algorithms have been reported~\cite{DDGJLMMSS:dcas,MiSc:correction}.

%% It is therefore important to develop efficient techniques for
%% verifying conformance to simple abstract specifications of overall
%% functionality, a concurrent implementation of a common
%% data type abstraction, such as a queue, should be verified to conform
%% to a simple abstract specification of a (sequential) queue.

A significant research effort has been directed towards this problem.
A major difficult is that a successful verification technique must be
able to reason about fine-grained concurrent programs that are infinite-state
in many dimensions: 
they consist of an unbounded number of concurrent threads, which
operate on an unbounded domain of data values, and use
unbounded dynamically allocated memory. 
The list of suggested approaches  is long. Many works address the
challenge by imposing some restrictions, such as
bounding the number of accessing
threads~\cite{Amit:comparisonAbstraction,Vechev:spin09,CernyRZCA:CAV10},
restricting the class of algorithms that can be verified
\cite{HSV:concur13,Vafeiadis:cav10},
or requiring auxiliary lemmas
\cite{OHearnlist,Poling}.
\todo[inline]{The last list and its description must be improved. We actually should be
more precise about what previous work has achieved, in terms of which
combinations of challenges have been overcome}
Recently, the authors of this paper have presented an approach which
overcomes these limitations.
Almost all of these approaches have been applied only to concurrent algorithms
that represent data structures by singly-linked lists. However, many concurrent
data structure implementations employ more sophisticated structures, such as
skip lists, trees, an various extensions of singly-linked lists, e.g., with
time-stamps.
There are almost no presented approaches for automatically verifying algorithms
such more complex data structure representations.

In this paper, we present a technique for automatic verification of
concurrent data structure implementations that are more complex than
just singly-linked lists.
\td{Here, there should be a precise statement of what we cover}
Our approach is the first framework that can automatically
verify concurrent data structure implementations that employ skip lists and
time-stamp queues, while also handling
an unbounded number of concurrent threads, an
unbounded domain of data values, and an unbounded heap structures.

\bjcom{Here is just some text to be developed}
In this abstraction, one first defines a
set of {\em tags}. Intuitively, a tag is a predicate on nodes in a heap,
which consists of constraints on the values of its fields, and lists
the set of pointer variables that point directly to the node, or that point
to a node that can reach or be reached from the node by a sequence of
{\tt next} pointers. Thus, for each program the set of tags is bounded.
The heap structure is then represented by a set of {\em fragments}. Each
fragment is a pair of tags, such that each pair of nodes in the
projection of the heap onto a particular thread is represented by some
fragment.


%% It is well understood how to
%% accomplish this for finite-state programs (as embodied, e.g., in
%% the SPIN tool~\cite{Holzmann:spin}),
%% but we lack approaches for handling unbounded data domains in specifications
%% and implementations in connection with an unbounded number of threads and
%% dynamically allocated data structures.

\todo[inline]{Here should be a verbose summary of contributions}

