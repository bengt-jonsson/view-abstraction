\section{Introduction}


%% One of the most difficult current challenges in
%% software verification is to automate its
%% application to 

Concurrent algorithms with an unbounded number of
threads that concurrently access a dynamically allocated shared state.
are of central importance in a large number of software systems.
They are widely used in libraries,
such as the Intel Threading Building Blocks or
the \url{java.util.concurrent} package,
to provide efficient concurrent realizations of
common interface abstractions.
They are notoriously difficult to get correct and verify, since they
often employ fine-grained synchronization and avoid locking when
possible. A number of bugs  in published
algorithms have been reported~\cite{DDGJLMMSS:dcas,MiSc:correction}.
Consequently, significant research efforts have been directed towards developing
techniques to verify correctness of such algorithms.
\bjcom{Maybe provide a concrete example of a verification task}.
A major difficulty is that a successful verification technique must be
able to reason about fine-grained concurrent algorithms that are infinite-state
in many dimensions: 
they consist of an unbounded number of concurrent threads, which
operate on an unbounded domain of data values, and use
unbounded dynamically allocated memory. 
Many presented techniques require significant {\em manual} effort for
constructing a proof of
correctness \bjinsert{references}, in some cases without any mechanical support,
in some cases with the support of an interactive theorem prover.
Automation is difficult to obtain, because of the above difficulties.
A majority of automated verification approaches impose restrictions on
the considered problem, such as bounding the number of accessing
threads~\cite{Amit:comparisonAbstraction,Vechev:spin09,CernyRZCA:CAV10},
\bjinsert{Here, we must improve over previous texts, which were:
restricting the class of algorithms that can be verified
\cite{HSV:concur13,Vafeiadis:cav10},
or requiring auxiliary lemmas
\cite{OHearnlist,Poling}.}
\todo[inline]{The last list and its description must be improved. We actually should be
more precise about what previous work has achieved, in terms of which
combinations of challenges have been overcome}
Almost all automated approaches are restricted to attention to concurrent algorithms
that represent data structures by singly-linked lists. However, many concurrent
data structure implementations employ more sophisticated structures, such as
skip lists, trees, and various extensions of singly-linked lists, e.g., with
time-stamps \bjinsert{References}.
There are almost no presented approaches for automatically verifying algorithms
that operate on shared data structure representations beyond singly-linked
lists \bjcom{Check what there is}

In this paper, we present
a technique for automatic verification of
concurrent data structure implementations that operate on data structures
more complex than just singly-linked lists.
\td{Here, there should be a precise statement of what we cover}
Our approach is the first framework that can automatically
verify concurrent data structure implementations that employ skip lists and
time-stamp queues, while also handling
an unbounded number of concurrent threads, an
unbounded domain of data values, and an unbounded heap structures.

\bjcom{Here is just some text to be developed}
In this abstraction, one first defines a
set of {\em tags}. Intuitively, a tag is a predicate on nodes in a heap,
which consists of constraints on the values of its fields, and lists
the set of pointer variables that point directly to the node, or that point
to a node that can reach or be reached from the node by a sequence of
{\tt next} pointers. Thus, for each program the set of tags is bounded.
The heap structure is then represented by a set of {\em fragments}. Each
fragment is a pair of tags, such that each pair of nodes in the
projection of the heap onto a particular thread is represented by some
fragment.


%% It is well understood how to
%% accomplish this for finite-state programs (as embodied, e.g., in
%% the SPIN tool~\cite{Holzmann:spin}),
%% but we lack approaches for handling unbounded data domains in specifications
%% and implementations in connection with an unbounded number of threads and
%% dynamically allocated data structures.

\todo[inline]{Here should be a verbose summary of contributions}

