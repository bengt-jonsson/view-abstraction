\begin{abstract}
%%   One of the most challenging tasks in software verification is to prove linearizability for concurrent threads  that  access a shared data structure. Such systems give rise to unbounded numbers of threads that operate on an bounded data domain and that access dynamically allocated memory. Furthermore, proving linearizability is harder than proving control state reachability due to  
%% existentially quantified linearization points. The problem is further complicated by the presence of advanced features such as non-fixed linearization points, speculation, and helping. In this paper, 
A major challenge in automated verification is to develop techniques that are
able to reason about fine-grained concurrent algorithms that
consist of an unbounded number of concurrent threads, which
operate on an unbounded domain of data values, and use
unbounded dynamically allocated memory. 
Existing automated techniques consider the case where
shared data is organized into singly-linked lists.
We present a novel shape analysis for automated verification of fine-grained concurrent algorithms that can handle heap structures which are more complex than just singly-linked lists, in particular
skip lists and arrays of singly linked lists, while
at the same time handling an unbounded
number of concurrent threads, an unbounded domain of data values
(including timestamps), and an unbounded shared heap.
Our technique is based on a novel shape abstraction,
which represent a set of heaps by a set of {\em fragments}.
A fragment is an abstraction of a pair of heap cells that are connected by a pointer field.
%% Our fragment abstraction is simple and uniform shape representation, which
%% anyway allows automatic verification of reachability properties.
We have implemented our approach and applied it to automatically verify
correctness, in the sense of linearizability, of 
most linearizable concurrent implementations
of sets, stacks, and queues, which emply singly-linked lists, skip lists, or
arrays of singly-linked lists with timestamps,
which are known to us in the literature.
\end{abstract}
