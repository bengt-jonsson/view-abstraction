\section{Data Structures, Observers, and Linearizability}
\label{linearizability:section}

\paragraph{Data Structure Semantics.}
A {\it data structure} $\dstruct$ is a pair
$\tuple{\ddomain,\malphabet}$ 
where $\ddomain$ is the {\it data domain} and 
$\malphabet$ is the alphabet of method names.
%
An {\it operation} 
%\todo{I have changed {\it event} to {\it operation.}}
$\op$ is of the form
$\mname(\indata,\outdata)$ where 
$\mname\in\malphabet$ is a method name and 
$\indata,\outdata\in\ddomain$ are the {\it input} resp.\ {\it output}
data values.
%
A {\it trace} of $\dstruct$ is a sequence of operations.
%
The behavior $\denotationof\dstruct$
of $\dstruct$ is the set of traces.
%%, called the {\it legal} traces of $\dstruct$.
%
We often identify $\dstruct$ with its behavior $\denotationof\dstruct$.
For example, in the {\tt Set} data structure, the set of method names
is given by $\set{{\tt add}, {\tt rmv}, {\tt ctn}}$,
and the data domain is the  union $\intgrs\cup\boolset$
of the sets of integers and Booleans.
%
Input and output data values are in $\intgrs$ and $\boolset$
respectively.
%
For instance, the operation ${\tt add}(3,\true)$ successfully adds the value
$3$, while ${\tt ctn}(2,\false)$ is a failed search
for $2$ in the set.
%
%(see Appendix~\ref{dstruct:sepcs:section} for a formal specification
%of $\denotationof{\tt Set}$, $\denotationof{\tt Queue}$, and $\denotationof{\tt Stack}$.)


\begingroup
\setlength\intextsep{-8pt}
\setlength{\columnsep}{4pt}
\begin{wrapfigure}{R}{6cm}
\center 

\begin{tikzpicture}[]

\node(s0)[draw,line width=1pt,fill=blue!10,circle,text=black]
{$s_0$};

\node(s1)[draw,line width=1pt,fill=blue!10,circle,text=black,anchor=center]
at ($(s0.center)+(80pt,0pt)$) {$s_1$};


\node(s2)[draw,line width=1pt,fill=blue!10,circle,text=black,scale=0.9,anchor=center ]
at ($(s0.center)+(40pt,-61pt)$) {$s_2$};

\node[draw,line width=1pt,circle,text=black,anchor=center,minimum size=5mm]
at ($(s0.center)+(40pt,-61pt)$) {};


\draw[->,>=stealth,out=45,in=135,line width=1pt] (s0) to 
node[above=-1pt,pos=0.5] {\footnotesize${\tt add}(x,\true)$} 
(s1);

\draw[->,>=stealth,out=225,in=315,line width=1pt] (s1) to node[above=0pt,pos=0.5] {\footnotesize${\tt rmv}(x,\true)$} 
(s0);

\draw [->,>=stealth,thick,out=270,in=180,line width=1pt] (s0) 
to 
node[left,pos=0.2] {\footnotesize${\tt add}(x,\false)$} 
node[left,pos=0.45] {\footnotesize${\tt rmv}(x,\true)$} 
node[left,pos=0.8] {\footnotesize${\tt ctn}(x,\true)$} 
 (s2);

\draw [->,>=stealth,thick,out=270,in=0,line width=1pt] (s1) 
to 
node[right,pos=0.2] {\footnotesize${\tt add}(x,\true)$} 
node[right,pos=0.45] {\footnotesize${\tt rmv}(x,\false)$} 
node[right,pos=0.8] {\footnotesize${\tt ctn}(x,\false)$} 
 (s2);






\end{tikzpicture}

\caption{Set observer.}

\label{set:observer:fig}
\end{wrapfigure}

\closeparagraph{Observers}
We specify traces of data structures by
{\em observers}, 
as introduced in~\cite{AHHR:integrated}. 
%
Observers are
finite automata extended with a finite set of {\em registers}
that assume values in $\intgrs$. 
%
%
At initialization,
the registers are nondeterministically
assigned arbitrary values, which never change
during a run of the observer. 
%
Formally, an observer $\observer$ is a tuple
$\otuple$ where $\ostateset$ is a finite set 
of {\it observer states} including the 
{\it initial state} $\oinitstate$ and
the {\it accepting state} $\oaccstate$, 
a finite set $\ovarset$  
of {\it registers}, and $\otransitionset$ is a finite
set of {\it transitions}.
%
%
Transitions are of the form 
$\tuple{\ostate_1,\mname(\inxvar,\outxvar),\ostate_2}$ where 
$\inxvar$ and $\outxvar$ are either registers or constants, i.e.,
transitions are labeled by 
operations whose input or output data may be parameterized on registers.
%
%% The observer processes a trace one operation at a time.
%% %
%% If there is a transition, whose label, after replacing registers by their
%% values, matches the operation, such a transition is performed. 
%% %
%% If there is no
%% such transition, the observer remains in its current state.
%
The observer accepts a trace if it can  be processed in such a way that
an accepting state is reached.
%
Observers can be used to give {\it exact} specifications of
the behaviors of data structures such as sets, queues, and stacks.
%
The observer is defined in such a way that it accepts precisely those
traces that do {\em not} belong to the behavior
of the data structure.
%
This is best illustrated by an example. Fig.~\ref{set:observer:fig}
depicts an observer that accepts the
sequences of operations that are {\em not} in $\denotationof{\tt Set}$.
%
%Other
%observers for stacks and queues can be found in~\cite{AHHR:integrated}
%and Appendix~\ref{dstruct:sepcs:section}.

\endgroup



%% \paragraph{Histories.}
\closeparagraph{Linearizability.}
An operation $\mname(\indata,\outdata)$ gives rise to two {\it actions}, namely
a {\it call action} $\call\eid\mname\indata$ and 
a {\it return action} $\return\eid\mname\outdata$, where $\eid\in\nat$ is an
{\it action identifier}.
%
A {\it history} $\history$ is a sequence of actions such that
for each return action $\action_2$ in $\history$ there is a unique
{\it matching}  call action $a_1$ where the action identifiers
in $\action_1$ and $\action_2$ are identical and different from the action
identifiers of all other actions in $\history$ and such that
$\action_1$ occurs before $\action_2$ in $\history$.
%
A call action carries always the same method name as its matching return
action.
A call action which does not match any return action in $\history$ is said
to be {\em pending}.
A history without pending call actions is said to be {\em complete}.
A {\em completed extension} of $\history$ is a complete history
$\history'$ obtained from $\history$ by
  appending (at the end) zero or more return actions that are matched by
  pending call actions in $\history$, and
  thereafter removing pending call actions.
%
For action identifiers $\eid_1,\eid_2$, we write
$\eid_1\occursbefore\history\eid_2$ to denote that
the return action with identifier $\eid_1$ occurs before
the call action with identifier $\eid_2$ in $\history$.
We say that a history is {\it sequential} if
it is of the form
$\action_1\action'_1\action_2\action'_2\cdots\action_\nn\action'_\nn$
%% where $\action'_\nn$ can be absent and
where $\action'_\ii$ is the matching action of $\action_\ii$ 
for all $\ii:1\leq\ii\leq\nn$, i.e., each call action 
is immediately followed by the matching return action. 
%
We identify a sequential history of the above form with
the corresponding trace 
$\op_1\op_2\cdots\op_\nn$ where
$\op_\ii=\mname(\indata_\ii,\outdata_\ii)$,
$\action_\ii=\call{\eid_\ii}{\mname}{\pindata\ii}$, and
$\action_\ii=\return{\eid_\ii}{\mname}{\poutdata\ii}$,
i.e., we merge each call action together with the matching return action
into one operation.
%
A complete history $\history'$ 
is a {\it linearization} of $\history$ if
(i) $\history'$ is a permutation of $\history$,
(ii) $\history'$ is sequential, 
and
(iii) $\eid_1\occursbefore{\history'}\eid_2$
if $\eid_1\occursbefore{\history}\eid_2$
for each pair of action identifiers $\eid_1$ and $\eid_2$.
%
We say that a sequential history $\history'$ is {\it valid} wrt.\ $\dstruct$ if
the corresponding trace is in $\denotationof{\dstruct}$.
%
We say that $\history$ is {\it linearizable} wrt.\ $\dstruct$ if there is
a completed extension of $\history$, which has
a linearization that is valid wrt.\ $\dstruct$.
%
A set $\histories$ of histories is {\it linearizable} wrt.\ $\dstruct$ if 
all members of $\histories$ are  {\it linearizable} wrt.\ $\dstruct$.


%\todo{The following should be moved somewhere else.}
%Observers can also be used to characterize
%histories that are not linearizable wrt. to data structures such as queues
%and stacks (but not sets \cite{HSV:concur13,jad:icalp}).
%
%In such a case the transitions are labeled by actions
%(rather than operations).
%\todo{put examples of additional observers in the appendix?}
